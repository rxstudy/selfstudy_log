\documentclass[12pt]{article}

\include{pythonlisting}
\usepackage{fullpage}
\usepackage{times}
\usepackage[normalem]{ulem}
\usepackage{multirow}
\usepackage{fancyhdr,graphicx,amsmath,amssymb, mathtools, scrextend, titlesec, enumitem}\usepackage[pdftex]{hyperref}
\usepackage[ruled,vlined]{algorithm2e} 
\usepackage{parskip}
\usepackage{listings}
\newcommand{\Var}{\mathrm{Var}}
\newcommand{\Cov}{\mathrm{Cov}}
\newcommand{\E}{{\rm I\kern-.3em E}}
\newcommand{\Binomial}{\mathrm{Binomial}}
\newcommand{\Beta}{\mathrm{Beta}}
\newcommand{\Poisson}{\mathrm{Poisson}}
\newcommand{\Normal}{\mathcal{N}}
\newcommand{\Uniform}{\mathrm{Uniform}}

\lstset{frame=tb,
  language=Python,
  aboveskip=3mm,
  belowskip=3mm,
  showstringspaces=false,
  columns=flexible,
  basicstyle={\small\ttfamily},
  numbers=none,
  stringstyle=\color{mauve},
  breaklines=true,
  breakatwhitespace=true,
  tabsize=3
}

\title{Chapter 1: Calculus on Euclidean Space}
\author{Ran Xie}
\begin{document}
\maketitle
\section{Dot Product}
\subsection*{1.1}
(a) $v \cdot w = 1(-1)+2(0)+(-1)3 = -4$

(b) $v \cross w = 2(3)U_1 - (3- 1)U_2 + (2)U_3 = 6U_1 - 2U_2 + 2U_3$

(c) $v/||v|| = \frac{1}{\sqrt{6}}(1, 2, -1)$.  and $w/||w|| = \frac{1}{\sqrt{10}}(-1, 0, 3)$

(d) $||v \cross w|| = \sqrt{36 + 4 + 4} = \sqrt{44}$

(e) $\cos \theta = \frac{v \cdot w}{||v||||w||} = \frac{-4}{\sqrt{6} \sqrt{10}} = - \frac{2}{\sqrt{15}}$

\subsection*{1.2}
(a) $d(p, q) = 0 \Leftrightarrow ||p-q|| = 0 \Leftrightarrow p - q = 0 \Leftrightarrow q = p$

(b) $d(p, q) = || p - q|| = |-1|||q - p| = d(q, p)$

(c) $d(p, q) + d(q, r) = ||p - q|| + ||q - r|| \geq ||p - q + q - r|| = ||p - r||$

\subsection*{1.3}
 $v = xe_1 + y e_2 + ze_3$. Then
 $$
 \begin{aligned}
 	\begin{pmatrix}
 		1/\sqrt{6} &  2/\sqrt{6} &  1/\sqrt{6} \\
 		-2/\sqrt{8} &  0 &  2/\sqrt{8} \\
 		1/\sqrt{3} &  -1/\sqrt{3} &  1/\sqrt{3} \\
 	\end{pmatrix}
 	\begin{pmatrix}
 		x \\ y \\ z\\
 	\end{pmatrix}
 	&=
 	\begin{pmatrix}
 		6 \\ 1 \\ -1
 	\end{pmatrix} \\
 \begin{pmatrix}
 	x \\ y \\ z\\
 \end{pmatrix}
&=
 \begin{pmatrix}
 	1/\sqrt{6} &  2/\sqrt{6} &  1/\sqrt{6} \\
 	-2/\sqrt{8} &  0 &  2/\sqrt{8} \\
 	1/\sqrt{3} &  -1/\sqrt{3} &  1/\sqrt{3} \\
 \end{pmatrix}^{-1}
\begin{pmatrix}
	6 \\ 1 \\ -1
\end{pmatrix} \\
 \end{aligned}
 $$

\subsection*{1.4}
(a)
$$\begin{aligned}
	u\cdot (v \cross w) &= (u_1 U_1 + u_2 U_2 + u_3 U_3) \cdot \begin{vmatrix}
		U_1 & U_2 & U_3 \\
		v_1 & v_2 & v_3 \\
		w_1 & w_2 & w_3 \\
	\end{vmatrix} \\
 &=  (u_1 U_1 + u_2 U_2 + u_3 U_3) \cdot (D_1U_1 - D_2 U_2 + D_3 U_3) \\
 &= D_1 u_1 - D_2 u_2 + D_3 u_3 \\
 &=  \begin{vmatrix}
 	u_1 & u_2 & u_3 \\
 	v_1 & v_2 & v_3 \\
 	w_1 & w_2 & w_3 \\
 \end{vmatrix}
\end{aligned}$$

(b) By (a), the product is equal to the determinant, the 3 vectors are independent iff the determinant is non zero. 

(c) By (a),  the product is equal to the determinant, swapping any two vectors is equivalent to swapping the rows in the determinant which in turn changes the sign.

(d) This is equivalent to swapping the rows even numbers of times so the sign of the determinant is unchanged.

\subsection*{1.5}
(a) Suppose $v$ and $w$ are linearly dependent, then $v\cross w =a (w \cross w) = 0$. 

Now suppose $v \cross w = 0$, then for any vector $u$, $u \dot (v \cross w) = \det(u, v, w) = 0$. This means for any $u$,  $u,v,w$ are linearly dependent. Since $\real^3$ requires 3 vectors to span the space, there exists $u$ such that $u$ is not linearly dependent with $v$ and $w$ yet the determinant of the three is 0. Therefore $v, w$ are linearly dependent.

(b) Since $v \cross w= ||v|| ||w|| \sin \theta$, by basic geometry, $||w|| \sin \theta$ is the height of the parallelogram and the $||v||$ is the base of it. Therefore cross product is the area of the parallelogram formed by $w, v$.

\subsection*{1.6}
Consider a matrix $E$, where its rows are denoted as $e_1, e_2, e_3$. Then $$E^T E = \begin{pmatrix}
	e_1 \cdot e_1  & 	e_1 \cdot e_2  & 	e_1 \cdot e_3  \\
	e_2 \cdot e_1  & 	e_2 \cdot e_2  & 	e_1 \cdot e_3  \\
	e_3 \cdot e_1  & 	e_1 \cdot e_2  & 	e_3 \cdot e_3  \\
\end{pmatrix}$$.  
If $E$ is orthogonal matrix, then the product above is the identity matrix which means the $e_1, e_2, e_3$ will need to satisfy the definition of a frame. If we take determinant on both side
$$ \det{E^TE} = (\det E)^2 = 1$$.
Therefore $\det E = \pm 1$.

\subsection*{1.7}
Take $v_1 = (v\cdot u) u$ to be the projection along $u$. Then $v = v_1 + v_2$ where $v_2$ is defined by $v - v_1$. We just need to check their dot product. 

$$ v_1 \cdot v_2 = v_1 \cdot (v - v_1) = v_1 \cdot v - ||v_1||^2 = (v \cdot u)u \cdot v - ||(v\cdot u) u||^2 = (v \cdot u)^2 (1 - ||u||^2) = 0$$ 

Since $u$ is unit vector.

\subsection*{1.8}
For a parallelepipe formed by $u, v, w$, the volume is the height, $h$ times the base parallogram area $A$, formed by $v, w$. 

$h$ can be found by projecting $u$ onto the unit vector $v\cross w / ||v\cross  w||$. So $h = u \cdot v\cross w / ||v\cross  w||$.

$A$ is simply $||v\cross  w||$. 

Therefore$$V = hA = u \cdot \frac{v\cross w}{||v\cross  w||} ||v\cross  w|| = u \cdot (v \cross w)$$

\subsection*{1.9}
(a) For any point $p$ such that $||p||<1$. There exists an $\epsilon > 0$ such that $||p|| < 1 -  \epsilon$. Then we have an open ball $B_\epsilon(p)$. For any $q$ in the open ball,  $$||q|| = ||q - p + p|| \leq ||q - p|| + ||p|| < \epsilon + (1 - \epsilon) = 1$$
Therefore the open ball is a proper subset and hence $\{p | ||p|| < 1\}$ is open.

(b) $\{p | p_3 > 0\} = \real^2 \cross H^+$. $H^+$ is open by the same argument from (a). Product of open sets is open in the induced product topology.

\subsection*{1.10}
(a) closed. Sphere boundary points are closed.

(b) Open.  $p_3\neq =0$ means $\{ p_3 > 0 \} \cup \{p_3 < 0\}$. And union of open sets is open from 1.9(b).

(c) Not open. This set is equal to the set of points on the plane constructed by $p_1=p_2$ minus the set of points on the line by $p_1=p_2 =p_3$. For example $(1,1,2)$ is a boundary point in the set. So not open.

(d) Open. Interior of a cylinder.

\subsection*{1.11}
(a) 
$$\begin{aligned}
	v \cdot (\grad{f}(p)) &= \langle \sum_i v_i U_i, \sum_i \partial_if U_i \rangle (p)\\
		&= \sum_i v_i \partial_i f (p) \\
		&= v[p] \\
		&= (df)(v)
\end{aligned}
$$

(b) For a unit vector $u$ at $p$, $u= \frac{v}{||v||}$ for some $v$. Therefore $u[f] = \langle u, \grad f \rangle \leq  \frac{1}{||v||}\left|\langle v, \grad f \rangle \right| \leq \frac{1}{||v||}||v|| ||\grad f|| = ||\grad f||$ by Cauchy Schwarz inequality. It achieves maximum when $v = \grad f $ which implies $u = \frac{v}{||v||} = \frac{\grad f}{||\grad f||}$

\subsection*{1.12}
Since $f^2 + g^2 = 1$, so $f'f + g'g = 0$. The derivative of $U$, $U'= fg' - gf'$.
Let $K(t) = (f - \cos U)^2 + (g - \sin U)^2$
$$
\begin{aligned}
	K'/2 &= (f - \cos U)(f' + U' \sin U ) + (g - \sin U) (g' -  U' \cos U) \\
	&=ff'+ fU'\sin U - f' \cos U -U' \sin U \cos U + gg' - gU'\cos U - g'\sin U + U'\sin U \cos U \\ 
	&=(ff'+g'g) + fU'\sin U - f' \cos U - gU'\cos U - g'\sin U \\
	&=  fU'\sin U - f' \cos U - gU'\cos U - g'\sin U \\
	&= U'(f \sin U - g \cos U) - (f' \cos U + g'\sin U) \\
	&= (fg' - gf')(f \sin U - g \cos U) - (f' \cos U + g'\sin U) \\
	&= f^2g'\sin U - fgg' \cos U - gf'f \sin U + g^2f' \cos U  - (f' \cos U + g'\sin U) \\
	&= f^2g'\sin U + f^2f' \cos U + g^2 g' \sin U + g^2 f' \cos U  - (f' \cos U + g'\sin U) \\
	&= f^2 (g'\sin U + f' \cos U) + g^2 (g' \sin U + f' \cos U ) - ( g'\sin U + f' \cos U) \\
	&=(g'\sin U + f' \cos U)(f^2 + g^2 - 1)\\
	&= 0
\end{aligned}
$$
The implies $K(t) = (f - \cos U)^2 + (g - \sin U)^2 = \mbox{ constant }$. Let $t=0$, $K(0) = (f(0) - \cos U_0)^2 + (g(0) - \sin U_0)^2 = 0$ since $f(0) = \cos U_0$ and $g(0) = \sin U_0$. Therefore $ (f - \cos U)^2 + (g - \sin U)^2  = 0$ for all $t$. Hence $f = \cos U$ and $g = \sin U$.\QED

\section{Curves}
\subsection*{2.1}
(a)  $\alpha(t) = (2t, t^2, t^3/3)$, $v(t) = (2, 2t, t^2)$, $|v(t)|= \sqrt{4+4t^2+t^4} = 2+t^2$ and $a(t) = (0, 2, 2t)$.  Then $v(1) = (2,2,1)$, $|v(1)| = 3$ and $a(1) = (0, 2, 2)$.

(b) $s(t) = \int^t_0 |v(u)| du =  \int^t_0 2 + u^2 du = 2t + t^3/3$

(c) Since $|v(t)| = 2+t^2$ is even function, so $s=\int^{1}_{-1}2+u^2du = 2 \int^1_0 2+u^2du = 2 s(1) = 4 + 2/3$

\subsection*{2.2}
Suppose $|\alpha'(t)| = c$, Then $ \alpha' \cdot \alpha' = c^2$. Take derivative on both side, $ 2 \alpha' \cdot \alpha '' = 0$ so they are orthogonal. 

Suppose $\alpha' \cdot \alpha '' = 0$,  then  $c = \int 0 = \int \alpha' \cdot \alpha '' = \frac{1}{2} \int \frac{d}{dt}(\alpha' \cdot \alpha' ) dt = ||\alpha||^2 /2$. \QED

\subsection*{2.3}
$a(t) = (\cosh t, \sinh t, t)$. Then $a'(t) = (\sinh t, \cosh t, 1)$. So $$s(t) = \int_0^{t}|a'(t)| dt = \int_0^t \sqrt{\sinh^2t + \cosh^2 t + 1} dt = \int_0^t \sqrt{2}\cosh t dt= \sqrt{2}\sinh t$$

So a unit lenght parameterization is $t = \sinh^{-1} \left( \frac{s}{\sqrt{2}}\right)$. $$\beta(s) = a( \sinh^{-1} \left( \frac{s}{\sqrt{2}}\right)) = \left(1 + \frac{s^2}{2},  \frac{s}{\sqrt{2}},  \sinh^{-1} \left( \frac{s}{\sqrt{2}}\right) \right)$$

\subsection*{2.4}
$a(t) = (2t, t^2, \log t)$, take $t=1$ and $t=2$ the curve passes through both points. the length between the two point is $l = \int^2_1 |a'(t)|dt = \int^2_1 \sqrt{4 + 4t^2 + 1/t^2}dt =  \int^2_1 2t + 1/t dt = 3 + \log 2$

\subsection*{2.5}
Suppose $\alpha(s)$ with unit parameterization and $\beta(s_1) = \alpha(s)$ is another unit parameterization ($s(s_1)$). Then $\frac{d\beta}{ds_1} = \frac{d\alpha}{ds} \frac{ds}{ds_1}$. Take the norm on both side, by the unit length assumption, we get $\left| \frac{ds}{ds_1} \right| = 1$. Integrating both side gives us $s = s_1 + C$. 

\subsection*{2.6}
(a) $Y(t) = -\cos tU_1  -\sin tU_2 - t U_3$

(b) $Y(t) = (-\sin t, \cos t, 1) - (-\cos t, -\sin t, 0) = (\cos t - \sin t)U_1 + (\cos t + \sin t)U_2 + U_3$

(c) $$a'(t) \cross a''(t) = \begin{vmatrix}
	U_1 & U_2 & U_3 \\
	-\sin t & \cos t & 1 \\
	-\cos t & -\sin t & 0 \\
\end{vmatrix} = \sin t U_1 - \cos t U_2 + U_3$$

Then $Y(t) = \frac{1}{\sqrt{2}}(\sin t U_1 - \cos t U_2 + U_3)$

(d) $Y(t) = a(t+\pi) - a(t) = (-\cos t, -\sin t, t+\pi) - (\cos t, \sin t, t) = -2\cos t U_1 - 2 \sin t U_2 + \pi U_3 $

\subsection*{2.7}
After parameterization, $a(h(t))$ is now defined on $t\in [c, d]$. Then the new arc length is $$ s = \int_c^d \left|\frac{da}{dt}\right|dt = \int_c^d \left|\frac{da}{dh}\right| \left|\frac{dh}{dt}\right|dt $$ 
Only when $\left|\frac{dh}{dt}\right|$ is monotone can we remove the absolute value. When we remove the absolute value, we get
$$ s = \int_c^d \left|\frac{da}{dh}\right| \left|\frac{dh}{dt}\right|dt = \pm \int_c^d \left|\frac{da}{dh}\right| \frac{dh}{dt} dt =  \pm \int_{h(c)}^{h(d)} \left|\frac{da}{dh}\right| dh  = \pm \int_{a}^{b} \left|\frac{da}{dh}\right| dh  $$ 
The last expression is exactly the definition of arc length of the original curve. \QED

\subsection*{2.8}
Let a reparameterization $h: I' \rightarrow I$ be $t = h(u)$ where $h$ is bijective. For each $u$, there exists $t$ such that $t = h(u)$. Then $Y(h(u)) = Y(h(h^{-1}t)) = Y(t)$ is a tangent vector at point $\alpha (h(u)) = \alpha (h(h^{-1}t)) = \alpha(t)$ by definition. \QED

By chain rule, $$(Y(h(t)))' =  \frac{d}{dt}\sum_i Y_i(\alpha(h(t)))U_i = \sum_i h'(t) \frac{dY_i \circ \alpha}{dt} (h(t))U_i =  h'(t)Y'(h(t))$$

\subsection*{2.9}
The integral for $\alpha$ is $$s = \int^\pi_0 \sqrt{\cos^2 t + (2t\cos t - t^2 \sin t)^2 + 4 \cos^2(2t)} dt \approx 12.9153$$

The integral for $\beta$ is $$s = \int^\pi_0 \sqrt{(2t\sin t + t^2\cos t)^2 + 4t^2 +(2t + 2t\cos t - t^2 \sin t)^2 }dt \approx 14.461$$

$\beta$ is longer.

\subsection*{2.10}
If $\alpha'$ and $\beta'$ are parallel for all $t$, then they have the same tangent vector component. $\alpha_i'(t) = \beta_i'(t)$ for all $i$. Integrating both side gives $\alpha_i(t) = \beta_i(t) + c_i$. Let $p=(c_1, c_2,c_3)$, then $\alpha(t) = \beta(t) + p$.

\subsection*{2.11}
(a) $L(\sigma) = |\sigma'(t)| = |-p + q| = d(p, q)$

(b) Perform Gram-schmidt on $u$, we get an orthornormal basis $\{u, u_2, \ldots u_n\}$. $\alpha'$ can be expressed in this new basis. $||\alpha'|| = ||\alpha_u' u + \sum_{i=2}^n \alpha_i' u_i|| =\sqrt{||\alpha_u||^2 + \sum_{i=2}^n||\alpha_i||^2 }\geq ||\alpha_u'|| = \alpha' \cdot u$.

$$ \begin{aligned}
L(\alpha) &= \int_a^b |\alpha'(t)|dt  \\
	  &\geq \int_a^b \alpha'(t) \cdot u dt \\
	  & = \int_a^b \alpha_u'(t) dt \\
	  & = \alpha_u(b) - \alpha_u(a) \\
	  & = |p - q|
\end{aligned}$$
The last equality holds because we use basis $\{u, u_2, \ldots u_n\}$. $p$ and $q$ both lies on the line $p + tu$ So $\alpha(a) = p = (\alpha_u(a), 0, 0)$ and $\alpha(b) = q = (\alpha_u(b), 0, 0)$.

(c)*

\section{The Frenet Formula}
\subsection*{3.1}
$\beta(s) = (\frac{4}{5}\cos s, 1- \sin s, -\frac{3}{5}\cos s)$

$T(s) = \beta'(s) =(-\frac{4}{5}\sin s, -\cos s, \frac{3}{5}\sin s) $

$T'(s) = (-\frac{4}{5}\cos s, \sin s, \frac{3}{5} \cos s)$

$\kappa = |T'(s)| = 1$

$N = T' / \kappa = T'$

$B = T \cross N = (-3/5, -4/5)$. Base on Frenet Formula, $B' = 0 = -\tau N \Rightarrow \tau = 0$.

$\beta$ is planar and has constant curvature. Therefore it is a circle. To find its center, note that $s$ has a period of $2\pi$ and since it is unit speed, we can find center as the midpoint of two points $s=0$ and $s=\pi$ on the circle. 

 $\beta(0) = (4/5, 1, -3/5)$ and  $\beta(\pi) = (-4/5, 1, 3/5)$. So the center is $(0, 1, 0)$. It's radius is $1$.

\subsection*{3.2}
$\beta(s) = \left(\frac{(1+s)^{3/2}}{3}, \frac{(1-s)^{3/2}}{3}, \frac{s}{\sqrt{2}}  \right)$

$T=\beta'(s) =(\sqrt{1+s}/2, -\sqrt{1 - s}/ 2, \frac{1}{\sqrt{2}})$

$T' = (\frac{1}{4\sqrt{1+s}}, \frac{1}{4\sqrt{1-s}}, 0)$

$\kappa = |T'| = \frac{1}{\sqrt{8(1+s)(1-s)}} $

$N = T'/\kappa = \left(\sqrt{(1-s)/ 2}, \sqrt{(1+s)/2}, 0 \right)$

$B = T \cross N = \left(-\sqrt{1+s}/2, \sqrt{1-s}/2, 1/\sqrt{2} \right)$

\subsection*{3.3}
Skip

\subsection*{3.4}
use $T, N, B$ are orthornmal basis which is equivalent to the $i, j, k$ canonical basis. The identities follow.

\subsection*{3.5}
$A = \tau T + \kappa B$

Using Frenet's formula and identities from exercise 3.4, we have
$$
\begin{aligned}
	&A \cross T = \tau T\cross T + \kappa B \cross T =  \kappa B \cross T = \kappa N = T' \\ 
	&A \cross B = \tau T \cross B + \kappa B \cross B = -\tau N = B' \\
	&A \cross N = \tau T \cross N + \kappa B \cross N = \tau B - \kappa T = N'
\end{aligned}
$$

\end{document}