\documentclass[12pt]{article}

\include{pythonlisting}
\usepackage{fullpage}
\usepackage{times}
\usepackage[normalem]{ulem}
\usepackage{multirow}
\usepackage{fancyhdr,graphicx,amsmath,amssymb, mathtools, scrextend, titlesec, enumitem}\usepackage[pdftex]{hyperref}
\usepackage[ruled,vlined]{algorithm2e}
\usepackage{parskip}
\usepackage{listings}
\usepackage{amsmath}
\usepackage{physics}
\usepackage{bbm}
\usepackage{titling}
\usepackage{bm}
\usepackage{geometry}
\geometry{top=1cm, left=1cm, bottom=1.5cm, right=1.5cm, margin=1cm}
\newcommand{\varvec}[2][n]{#2_1,\ldots, #2_#1}
\newcommand{\Var}{\mathrm{Var}}
\newcommand{\Bias}{\mathrm{Bias}}
\newcommand{\Cov}{\mathrm{Cov}}
\newcommand{\E}{{\rm I\kern-.3em E}}
\newcommand{\Binomial}{\mathrm{Binomial}}
\newcommand{\Bernoulli}{\mathrm{Bernoulli}}
\newcommand{\Poisson}{\mathrm{Poisson}}
\newcommand{\Normal}{\mathcal{N}}
\newcommand{\one}{\mathbbm{1}}
\newcommand{\Beta}{\text{Beta}}
\newcommand{\BetaPDF}{\text{Beta}}
\newcommand{\GammaPDF}{\text{Gamma}}
\newcommand{\Uniform}{\mathrm{Uniform}}
\newcommand{\QED}{\newline \mbox{} \hfill $\blacksquare$}
\newcommand{\Real}{\mathbb{R}}
\newcommand{\Sgn}{\mbox{Sgn}}
\renewcommand*{\arraystretch}{1.4}
\newtheorem{theorem}{Result}
\newtheorem{definition}{Definition}
\lstset{frame=tb,
	language=Python,
	aboveskip=3mm,
	belowskip=3mm,
	showstringspaces=false,
	columns=flexible,
	basicstyle={\small\ttfamily},
	numbers=none,
	stringstyle=\color{mauve},
	breaklines=true,
	breakatwhitespace=true,
	tabsize=3
}

\title{Chapter 3: Euclidean Geometry}
\author{Ran Xie}
\begin{document}
	\maketitle

\section{Isometries of $\real^3$}	
\subsection*{1}
Consider $$ \begin{aligned}
	|C(p+a) - C(p) - C(a)|^2 &= C(p+a)\cdot C(p+a) + C(p)\cdot C(p) + C(a) \cdot C(a) \\&\ \ \   - 2C(p+a)\cdot C(p) - 2C(p+a)\cdot C(a) + 2C(p)\cdot C(a)  \\
	          &= (p+a)^2 + p^2 + a^2 - 2(p+a)p - 2(p+a)a + 2pa \\
	          &= p^2 + 2pa + a^2 + p^2 + a^2 - 2p^2 - 2pa - 2pa - 2a^2 + 2pa \\
	          &= 0
\end{aligned}$$
Therefore $C(p + a) = C(p) + C(a)$. It follows that $CT_a(p) = C(p + a) = C(p) + C(a) = T_{C(a)}C(p)$ \QED


\subsection*{2}
From the result in problem 1.1
$FG = T_aA T_b B = T_aT_{A(b)}AB$ and $GF=T_bBT_aA = T_bT_{B(a)}BA$. The transnational parts are $T_{a+A(b)}$ and $T_{b+B(a)}$ respectively.

\subsection*{3}
Suppose $Cp = Cq$,  Then $$
\begin{aligned}
	&\Leftrightarrow \langle Cp - Cq, Cp - Cq\rangle = 0 \\
	&\Leftrightarrow CpCp - 2CpCq - CqCq = 0 \\
	&\Leftrightarrow p^2 - 2pq - q^2 = 0 \\
	&\Leftrightarrow p = q
\end{aligned}$$ 
$C$ is 1-1. Therefore there exists inverse $C^{-1}$. To show $C^{-1}$ is orthogonal transformation. Suppose $p, q$ such that $C^{-1}p = \tilde{p}$ and $C^{-1}q = \tilde{q}$
$$ \langle C^{-1}p, C^{-1}q\rangle =  \langle\tilde{p}, \tilde{q}\rangle = \langle C\tilde{p}, C\tilde{q}\rangle = \langle p, q\rangle $$
So $C^{-1}$ is orthogonal transformation. We can define the inverse of $F$. $F^{-1} = (T_aC)^{-1}  = C^{-1}T_{-a}$. $F^{-1}$ is isometry. 

\subsection*{4}
$$
 C = \frac{1}{3} \begin{pmatrix}
 	-2 & 2 & -1 \\
 	2 & 1 & -2 \\
 	1 & 2 & 2 \\
 \end{pmatrix}
$$
It's trivial to check orthogonality after factoring out 1/3.

$Cp = \frac{1}{3}(2, 19, -7)$ and $Cq=\frac{1}{3}(-5, -4, 7)$. Then $\langle Cp, Cq \rangle = \frac{1}{9} (-135) = -15 = \langle p, q \rangle$.

\subsection*{5}
(a) $q = F(p) = T_a C(p) = (-3\sqrt{2} + 1, 1, 5\sqrt{2} - 1)^T$

(b) $q = F^{-1}(p) = (T_aC)^{-1}(p) = C^{-1}T_{-a}(p) = C^T T_{-a}(p) = (5\sqrt{2}, -5, 4\sqrt{2})^T $

(c) $q = (CT_a)(p) = (5\sqrt{2}, 1, 2\sqrt{2})^T$

\subsection*{6}
(a) $C= \mbox{diag}(-1, -1, -1)$ and $a = (0,0,0)$.

(b) Not isometry. If $p\perp a$, then $d(F(p), 0) = d(0,0) = 0 \neq d(p, 0)$. 

(c) $C = I$, $a = (-1, -2, -3)$.

(d) $C = \mbox{diag}(1, 1, 0)$, $a=(0, 0, 1)$.

\subsection*{7}
For $F_1, F_2 \in \mbox{Iso}(3)$, $F_1F_2 = T_aC_1T_bC_2 = T_aT_{C_1(b)}C_1C_2 \in \mbox{Iso}(3)$. Associative is trivial since they are functions. Inverse exists for every $F$ as proven in problem 3.

\subsection*{8}
Only Identity is in both subgroups.

\subsection*{9}
(a) For an orthgonal matrix $\begin{pmatrix}
	a & b \\ c& d \\
\end{pmatrix}$, it satisfies 
$$\begin{cases}
	ac + bd = 0 \\
	a^2 + b^2 = 1 \\
	c^2 + d^2 = 1 
\end{cases}$$
We have a free parameter. Let $d = \pm \sin \theta$, then $$ \begin{cases}
	d = \pm \sin \theta \\
	c = \cos\theta\\
	b = \mp \cos\theta \\
	a = \sin\theta 
\end{cases} $$
So $\begin{pmatrix}
	a & b \\ c& d \\
\end{pmatrix} = \begin{pmatrix}
\sin\theta & \mp \cos\theta \\ \cos\theta&  \pm \sin\theta \\
\end{pmatrix}$

(b) $F = T_a C$. $Cp Cp = p^2 \Rightarrow c^2 p^2 = p^2 \Rightarrow c = 1$. So an isometry in $\real$ is just a displacement by a constant $a$.

\section{The tangent map of an isometry}
\subsection*{1}
Translation is an isometry, so $T(v_p) = I(v)_{T{p}} = v_{T(p)}$ which has the same Euclidean coordinates as $v_p$.

\subsection*{2}
Given isometries $G=T_gC_g, F=T_fC_f $,  $(GF)_*(v_p) = (T_gC_gT_fC_f)_*(v_p) = (T_gT_{C_g(f)} C_g C_f)_*(v_p) = C_g C_f (v)_{G\circ F(p)} = G_* F_* (v) $

\subsection*{3}
$F = T_a C$,  $p=(0, 1, 0)$, $q=(3, -1, 1)$

we have $[e] = A[u] =  \frac{1}{3}\begin{pmatrix}
	2 & 2 & 1 \\-2 & 1 & 2 \\ 1 & -2 & 2 
\end{pmatrix} [u]$ and   $[f] = B[u] =  \frac{1}{\sqrt{2}} \begin{pmatrix}
1 & 0 & 1 \\ 0 & \sqrt{2} & 0 \\ 1 & 0 & -1
\end{pmatrix} $

To transform from coordinates of $e$ to $f$.
$$C = B^tA =  \frac{1}{\sqrt{2}} \begin{pmatrix}
	1 & 0 & 1 \\ 0 & \sqrt{2} & 0 \\ 1 & 0 & -1
\end{pmatrix} \frac{1}{3}\begin{pmatrix}
2 & 2 & 1 \\-2 & 1 & 2 \\ 1 & -2 & 2 
\end{pmatrix} = \begin{pmatrix}
1 / \sqrt{2} & 0 & 1 /\sqrt{2} \\  -2 /3 & 1/3 & 2/3 \\  \sqrt{2}/6 &  2\sqrt{2}/3  & - \sqrt{2}/6
\end{pmatrix}$$

$F(p) = T_a C (p) = a + Cp =  q$.  So $ a = q - Cp = (3, -1, 1) - (0, 1/3 , 2\sqrt{2}/3) = (3, -4/3, 1 - 2\sqrt{2}/3) $

\subsection*{4}
(a)
A plane is defined by $\langle (x - p)_p , q_p \rangle = 0$. If an isometry $F=T_aC$, 
then $$
 \begin{aligned}
 	& \langle (x - p)_p , q_p \rangle = 0 \\
 	&\Leftrightarrow \langle F_* (x - p)_p ,  F_* q_p \rangle = 0 \\
 	&\Leftrightarrow \langle C (x - p)_{F(p)} ,  Cq_{F(p)}  \rangle = 0 \\
 	&\Leftrightarrow \langle C(T_{C(a)}x - T_{C(a)}p) _{F(p)} ,  Cq_{F(p)}  \rangle = 0 \\
 	&\Leftrightarrow \langle (F(x) - F(p))_{F(p)} ,  Cq_{F(p)}  \rangle = 0 \\
 \end{aligned}
$$
Note that $(T_{C(a)}x - T_{C(a)}p = x - p$ since translation is canceled out. \QED

(b)
Let $e_1 = (0, 1, 0),  e_2 = (1/\sqrt{2}, 0, - 1/\sqrt{2})$, then $e_3 = e_1 \cross e_2 = (-1/\sqrt{2}, 0, -1/\sqrt{2})$ form a frame. From $e_1$ to $e_2$, we simply need to perform a 90 degree rotation along $e_3$. The transformation is  $C_e = \begin{pmatrix}
	0 & -1 & 0 \\ 
	1 & 0 & 0 \\
	0 & 0 & 1 \\
\end{pmatrix}$ wrt to the frame. Then it is $A^t C_e A$ in the canonical frame where $A$ is the attitude matrix. We get 
$$
  C_u = A^t C_e A = \begin{pmatrix}
  	0 & 1 & 0 \\
  	1/\sqrt{2} & 0 & -1/\sqrt{2} \\
  	-1/\sqrt{2} & 0 & -1/\sqrt{2} 
  \end{pmatrix}^t
\begin{pmatrix}
	0 & -1 & 0 \\
	1 &  0 & 0 \\
	0 &  0 & 1
\end{pmatrix}
\begin{pmatrix}
	0 & 1 & 0 \\
	1/\sqrt{2} & 0 & -1/\sqrt{2} \\
	-1/\sqrt{2} & 0 & -1/\sqrt{2} 
\end{pmatrix}
= \frac{1}{2} \begin{pmatrix}
	1 & \sqrt{2} & 1 \\- \sqrt{2} & 0  & \sqrt{2} \\ 1 & - \sqrt{2} & 1 
\end{pmatrix}
$$

Since $F(1/2, -1, 0) = TC(1/2, -1, 0) = (1, -2, 1)$, we get $T = (3/4 - \sqrt{2}/2, -2 + \sqrt{2}/4 , 3/4 - \sqrt{2}/ 2) $


\end{document}