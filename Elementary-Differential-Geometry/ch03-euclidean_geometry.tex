\documentclass[12pt]{article}

\include{pythonlisting}
\usepackage{fullpage}
\usepackage{times}
\usepackage[normalem]{ulem}
\usepackage{multirow}
\usepackage{fancyhdr,graphicx,amsmath,amssymb, mathtools, scrextend, titlesec, enumitem}\usepackage[pdftex]{hyperref}
\usepackage[ruled,vlined]{algorithm2e} 
\usepackage{parskip}
\usepackage{listings}
\newcommand{\Var}{\mathrm{Var}}
\newcommand{\Cov}{\mathrm{Cov}}
\newcommand{\E}{{\rm I\kern-.3em E}}
\newcommand{\Binomial}{\mathrm{Binomial}}
\newcommand{\Beta}{\mathrm{Beta}}
\newcommand{\Poisson}{\mathrm{Poisson}}
\newcommand{\Normal}{\mathcal{N}}
\newcommand{\Uniform}{\mathrm{Uniform}}

\lstset{frame=tb,
  language=Python,
  aboveskip=3mm,
  belowskip=3mm,
  showstringspaces=false,
  columns=flexible,
  basicstyle={\small\ttfamily},
  numbers=none,
  stringstyle=\color{mauve},
  breaklines=true,
  breakatwhitespace=true,
  tabsize=3
}

\title{Chapter 3: Euclidean Geometry}
\author{Ran Xie}
\begin{document}
	\maketitle

\section{Isometries of $\real^3$}	
\subsection*{1}
Consider $$ \begin{aligned}
	|C(p+a) - C(p) - C(a)|^2 &= C(p+a)\cdot C(p+a) + C(p)\cdot C(p) + C(a) \cdot C(a) \\&\ \ \   - 2C(p+a)\cdot C(p) - 2C(p+a)\cdot C(a) + 2C(p)\cdot C(a)  \\
	          &= (p+a)^2 + p^2 + a^2 - 2(p+a)p - 2(p+a)a + 2pa \\
	          &= p^2 + 2pa + a^2 + p^2 + a^2 - 2p^2 - 2pa - 2pa - 2a^2 + 2pa \\
	          &= 0
\end{aligned}$$
Therefore $C(p + a) = C(p) + C(a)$. It follows that $CT_a(p) = C(p + a) = C(p) + C(a) = T_{C(a)}C(p)$ \QED


\subsection*{2}
From the result in problem 1.1
$FG = T_aA T_b B = T_aT_{A(b)}AB$ and $GF=T_bBT_aA = T_bT_{B(a)}BA$. The transnational parts are $T_{a+A(b)}$ and $T_{b+B(a)}$ respectively.

\subsection*{3}
Suppose $Cp = Cq$,  Then $$
\begin{aligned}
	&\Leftrightarrow \langle Cp - Cq, Cp - Cq\rangle = 0 \\
	&\Leftrightarrow CpCp - 2CpCq - CqCq = 0 \\
	&\Leftrightarrow p^2 - 2pq - q^2 = 0 \\
	&\Leftrightarrow p = q
\end{aligned}$$ 
$C$ is 1-1. Therefore there exists inverse $C^{-1}$. To show $C^{-1}$ is orthogonal transformation. Suppose $p, q$ such that $C^{-1}p = \tilde{p}$ and $C^{-1}q = \tilde{q}$
$$ \langle C^{-1}p, C^{-1}q\rangle =  \langle\tilde{p}, \tilde{q}\rangle = \langle C\tilde{p}, C\tilde{q}\rangle = \langle p, q\rangle $$
So $C^{-1}$ is orthogonal transformation. We can define the inverse of $F$. $F^{-1} = (T_aC)^{-1}  = C^{-1}T_{-a}$. $F^{-1}$ is isometry. 

\subsection*{4}
$$
 C = \frac{1}{3} \begin{pmatrix}
 	-2 & 2 & -1 \\
 	2 & 1 & -2 \\
 	1 & 2 & 2 \\
 \end{pmatrix}
$$
It's trivial to check orthogonality after factoring out 1/3.

$Cp = \frac{1}{3}(2, 19, -7)$ and $Cq=\frac{1}{3}(-5, -4, 7)$. Then $\langle Cp, Cq \rangle = \frac{1}{9} (-135) = -15 = \langle p, q \rangle$.

\subsection*{5}
(a) $q = F(p) = T_a C(p) = (-3\sqrt{2} + 1, 1, 5\sqrt{2} - 1)^T$

(b) $q = F^{-1}(p) = (T_aC)^{-1}(p) = C^{-1}T_{-a}(p) = C^T T_{-a}(p) = (5\sqrt{2}, -5, 4\sqrt{2})^T $

(c) $q = (CT_a)(p) = (5\sqrt{2}, 1, 2\sqrt{2})^T$

\subsection*{6}
(a) $C= \mbox{diag}(-1, -1, -1)$ and $a = (0,0,0)$.

(b) Not isometry. If $p\perp a$, then $d(F(p), 0) = d(0,0) = 0 \neq d(p, 0)$. 

(c) $C = I$, $a = (-1, -2, -3)$.

(d) $C = \mbox{diag}(1, 1, 0)$, $a=(0, 0, 1)$.

\subsection*{7}
For $F_1, F_2 \in \mbox{Iso}(3)$, $F_1F_2 = T_aC_1T_bC_2 = T_aT_{C_1(b)}C_1C_2 \in \mbox{Iso}(3)$. Associative is trivial since they are functions. Inverse exists for every $F$ as proven in problem 3.

\subsection*{8}
Only Identity is in both subgroups.

\subsection*{9}
(a) For an orthgonal matrix $\begin{pmatrix}
	a & b \\ c& d \\
\end{pmatrix}$, it satisfies 
$$\begin{cases}
	ac + bd = 0 \\
	a^2 + b^2 = 1 \\
	c^2 + d^2 = 1 
\end{cases}$$
We have a free parameter. Let $d = \pm \sin \theta$, then $$ \begin{cases}
	d = \pm \sin \theta \\
	c = \cos\theta\\
	b = \mp \cos\theta \\
	a = \sin\theta 
\end{cases} $$
So $\begin{pmatrix}
	a & b \\ c& d \\
\end{pmatrix} = \begin{pmatrix}
\sin\theta & \mp \cos\theta \\ \cos\theta&  \pm \sin\theta \\
\end{pmatrix}$

(b) $F = T_a C$. $Cp Cp = p^2 \Rightarrow c^2 p^2 = p^2 \Rightarrow c = 1$. So an isometry in $\real$ is just a displacement by a constant $a$.
\end{document}