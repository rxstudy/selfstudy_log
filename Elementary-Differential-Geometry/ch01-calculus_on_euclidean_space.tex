\documentclass[12pt]{article}

\include{pythonlisting}
\usepackage{fullpage}
\usepackage{times}
\usepackage[normalem]{ulem}
\usepackage{multirow}
\usepackage{fancyhdr,graphicx,amsmath,amssymb, mathtools, scrextend, titlesec, enumitem}\usepackage[pdftex]{hyperref}
\usepackage[ruled,vlined]{algorithm2e}
\usepackage{parskip}
\usepackage{listings}
\usepackage{amsmath}
\usepackage{physics}
\usepackage{bbm}
\usepackage{titling}
\usepackage{bm}
\usepackage{geometry}
\geometry{top=1cm, left=1cm, bottom=1.5cm, right=1.5cm, margin=1cm}
\newcommand{\varvec}[2][n]{#2_1,\ldots, #2_#1}
\newcommand{\Var}{\mathrm{Var}}
\newcommand{\Bias}{\mathrm{Bias}}
\newcommand{\Cov}{\mathrm{Cov}}
\newcommand{\E}{{\rm I\kern-.3em E}}
\newcommand{\Binomial}{\mathrm{Binomial}}
\newcommand{\Bernoulli}{\mathrm{Bernoulli}}
\newcommand{\Poisson}{\mathrm{Poisson}}
\newcommand{\Normal}{\mathcal{N}}
\newcommand{\one}{\mathbbm{1}}
\newcommand{\Beta}{\text{Beta}}
\newcommand{\BetaPDF}{\text{Beta}}
\newcommand{\GammaPDF}{\text{Gamma}}
\newcommand{\Uniform}{\mathrm{Uniform}}
\newcommand{\QED}{\newline \mbox{} \hfill $\blacksquare$}
\newcommand{\Real}{\mathbb{R}}
\newcommand{\Sgn}{\mbox{Sgn}}
\renewcommand*{\arraystretch}{1.4}
\newtheorem{theorem}{Result}
\newtheorem{definition}{Definition}
\lstset{frame=tb,
	language=Python,
	aboveskip=3mm,
	belowskip=3mm,
	showstringspaces=false,
	columns=flexible,
	basicstyle={\small\ttfamily},
	numbers=none,
	stringstyle=\color{mauve},
	breaklines=true,
	breakatwhitespace=true,
	tabsize=3
}

\title{Chapter 1: Calculus on Euclidean Space}
\author{Ran Xie}
\begin{document}
\maketitle

\section*{1.1.1}
(a) $fg^2 = x^2y(y\sin z)^2 = x^2y^3 \sin^2 z$

(b) $g \partial_x f + f \partial_y g = y \sin z (2xy) + x^2y (\sin z)= (2xy^2 + x^2y)\sin z$

(c) $\partial^2_{yz} (fg) =\partial^2_{yz} (x^2y^2\sin z) = 2x^2 y \cos z $

(d) $\partial_y (\sin f) =\partial_y \sin (x^2y) = x^2 \cos(x^2y) $

\section*{1.1.2}
(a) $0$

(b) $3^2(-1)- (1) 0.5 = -9 - 0.5 = 9.5$

(c) $a^2 -(1-a) = a^2 + a - 1$

(d) $t^2 t^2 - t^4 t^3 = t^4-t^7$

\section*{1.1.3}
(a) $\partial_x (x\sin (xy) + y \cos (xz)) = \sin(xy) + xy\cos(xy) - yz\sin(xz)$

(b) $$\partial_x f = \frac{\partial f}{\partial g}\frac{\partial g}{\partial h}\frac{\partial h}{\partial x} = (\cos g)(e^h)(2x)= 2xe^{x^2+y^2+z^2}\cos(e^{x^2+y^2+z^2}) $$

\section*{1.1.4}

Since $h = x^2- yz$, so $h(g_1, g_2, g_3) = g_1^2 - g_2g_3 $
$$\begin{aligned}
	\frac{\partial f}{\partial x} &=\frac{\partial h}{\partial g_1}\frac{\partial g_1}{\partial x} + \frac{\partial h}{\partial g_2}\frac{\partial g_2}{\partial x} + \frac{\partial h}{\partial g_3}\frac{\partial g_3}{\partial x} \\
	&= 2g_1\frac{\partial g_1}{\partial x} -g_3 \frac{\partial g_2}{\partial x} -g_2 \frac{\partial g_3}{\partial x}
\end{aligned} $$

(a) $2(x+y)(1) - (x+z)(0) - y^2(1) = 2(x+y) - y^2$

(b) $2e^z(0)  - e^x (e^{x+y})   -e^{x+y} (e^x) = -2e^x e^{x+y}$

(c) $2x (1) - x (-1)   + x (1) = 0$

\section*{1.2.1}
(a) $3v_p - 2w_p =3(-2,1,-1) - 2(0,1,3) = (-6, 1, -9) = -6U_1 + U_2 -9U_3$

\section*{1.2.2}
$W-xV = 2x^2U_2 - U_3 - x(x^2U_1 + xyU_2) = -x^3U_1 + (2x^2 -x^2y)U_2 - U_3$

At $p=(-1,0,2)$, $$(W-xV)(p) =  -(-1)^3 U_1(p) + (2(-1)^2 -(-1)^2 0)U_2(p) - U_3(p) = U_1(p) + 2U_2(p) - U_3(p) = (1, 2, -1)$$

\section*{1.2.3}
(a) $V = \frac{1}{7} (2z^2 U_1 - xyU_3) = \frac{2z^2}{7} U_1 - \frac{xy}{7}U_3$

(b) $V  = p_1U_1 + (p_3 - p_1)U_2$

(c) $V = xU_1 + 2y U_2 + xy^2 U_3$

(d) $V = (1+p_1, p_2p_3, p_2) - (p_1, p_2, p_3) = (1, p_2(p_3 - 1), p_2 - p_3) = U_1 + p_2(p_3 - 1) U_2 + (p_2 - p_3) U_3$

(e) $V = 0 - p = -p_1 U_1 - p_2 U_2 - p_3 U_3$

\section*{1.2.4}
If $f V + g W = f (y^2 U_1 - x^2 U_3) + g(x^2 U_1 - zU_2)$, the coefficient for $U_1$ is $f y^2 + gx^2 = 0$.


\section*{1.2.5}
(a) Suppose $a V_1 + b V_2 + c V_3 = 0$,  then $ (a + cx) U_1 + bU_2 + (-ax + c) U_3 = 0$. By independence of the natural basis, We can see $b = 0$. Moreover  $a +cx = 0$ and $ax - c = 0$ for all $x$. Take $x=0$, we get $c=0$ and $a=0$ follows. Therefore they are linearly independent at each point. 

(b) We can write $V$ in terms of $U$ in matrix form.
$$
 \begin{pmatrix}
 	V_1 \\  V_2  \\  V_3
 \end{pmatrix}
=
\begin{pmatrix}
	1 & 0 & -x \\
	0 & 1 & 0 \\
	x & 0 & 1  \\
\end{pmatrix}
\begin{pmatrix}
	U_1 \\  U_2  \\  U_3
\end{pmatrix}
$$
Then $$
 \begin{pmatrix}
	U_1 \\  U_2  \\  U_3
\end{pmatrix}
=
\begin{pmatrix}
	1 & 0 & -x \\
	0 & 1 & 0 \\
	x & 0 & 1  \\
\end{pmatrix}^{-1}
\begin{pmatrix}
	V_1 \\  V_2  \\  V_3
\end{pmatrix}
$$
Then 
$$
\begin{aligned}
xU_1 + yU_2 + zU_3 = \begin{pmatrix}
	x & y & z
\end{pmatrix}
\begin{pmatrix}
	U_1 \\ U_2 \\ U_3
\end{pmatrix}
&= \begin{pmatrix}
	x & y & z
\end{pmatrix} 
\begin{pmatrix}
	1 & 0 & -x \\
	0 & 1 & 0 \\
	x & 0 & 1  \\
\end{pmatrix}^{-1}
\begin{pmatrix}
	V_1 \\  V_2  \\  V_3
\end{pmatrix}  \\
&= \frac{1}{1+x^2} \begin{pmatrix}
	x & y & z
\end{pmatrix} 
\begin{pmatrix}
	-x^2 & 0 & x \\
	0 & 1 & 0 \\
	-x & 0 & 1  \\
\end{pmatrix}
\begin{pmatrix}
	V_1 \\  V_2  \\  V_3
\end{pmatrix}  \\
&= -\frac{x^3 + xz}{1+x^2} V_1+\frac{y}{1+x^2} V_2 + \frac{x^2+z}{1+x^2}V_3
\end{aligned}
$$
\QED

\section*{1.3.1}
(a) $v_p[f] = \frac{d}{dt}f(p + tv) |_{t=0} = \frac{d}{dt}f(2+2t, -t, -1+3t) |_{t=0} = \frac{d}{dt}(t^2(3t-1)) |_{t=0} = 0$

(b) $v_p[f] = \frac{d}{dt}f(p+tv)|_{t=0} =  \frac{d}{dt}f(2+2t, -t, -1+3t) |_{t=0} = 14(2+2t)^6 |_{t=0} = 896$

(c) $v_p[f] = \frac{d}{dt}f(p+tv)|_{t=0} =  \frac{d}{dt}f(2+2t, -t, -1+3t) |_{t=0} = \frac{d}{dt} e^{2+2t}\cos t |_{t=0} = 2e^{2+2t}\cos t |_{t=0} = 2e^2$


\section*{1.3.2}
(a) $v_p[f] = \partial_x f (p) v_1 + \partial_y f (p) v_2 + \partial_z f (p) v_3 = 0 + 2yz(p) (-1) + y^2(p) (3) = 0$

(b) $v_p[f] = \partial_x f (p) v_1 + \partial_y f (p) v_2 + \partial_z f (p) v_3 = 7x^6(p) 2 = 896$

(c) $v_p[f] = e^x\cos y |_p v_1 - e^x \sin y |_p v_2 = 2e^2$

\section*{1.3.3}
Note that $V = y^2U_1 - xU_3 = y^2 \partial_x - x \partial_z $

(a) $V[f] = y^2 \partial_x (xy) - x \partial_z (xy) = y^3$

(b) $V[g] = y^2 \partial_x z^3 - x \partial_z z^3 = 3xz^2$

(c) $V[fg] =  y^2 \partial_x (xyz^3) - x \partial_z (xyz^3) = y^2 y z^3 - x(3xyz^2) = y^3z^3 - 3x^2yz^2$

(d) $fV[g] - gV[f] = xy(3xz^2) - z^3 y^3 = 3x^2yz^2 - y^3z^3$

(e) $V[f^2 + g^2] = V[f^2] + V[g^2] = 2f V[f] + 2g V[g] = 2xy (y^3) + 2z^3 (3xz^2) = 2xy^4 + 6xz^5$

(f) $V[V[f]] = V[y^3] = y^2 \partial_x y^3 - x \partial_z y^3 = 0$

\section*{1.3.4}
For any point $p$, $V_p = \sum_i v_i(p)U_i(p)$. Then $V_p[x_j] = \sum_i v_i(p) U_i(p)[x_j] = \sum_i v_i(p) \delta_{ij} = v_j(p)$. \QED

\section*{1.3.5}
Note that $V= \sum_i v_i U_i$ and $W = \sum_i w_i U_i$. Since $V[f]= W[f]$ for every $f$,  take $f= x_j$, we get $v_j = w_j$ for every $j$. Hence $V = W$ \QED

\section*{1.4.1}
Since $\alpha(t) = (1 + \cos t, \sin t, 2\sin (t/2))$. $\alpha'(t) = (\sin t, \cos t, \cos(t/2)).$

$t = 0, \alpha'(0) = (0, 1, 1)$ 

$t = \frac{\pi}{2}, \alpha'(\pi/2) = (1, 0, \sqrt{2}/2)$

$t = \pi, \alpha'(\pi) = (0, -1, 0)$

\section*{1.4.2}
$\alpha(t) = \int \alpha'(t) dt = (t^3/3, t^2/2, e^t) + C$.

Since $\alpha(0) = (0, 0, 1) + C = (1, 0, 5)$, so $C = (1, 0, 4)$.

$\alpha(t) = (t^3/3 + 1, t^2/2, e^t + 4)$

\section*{1.4.3}
Since $\alpha(t) = (1 + \cos t, \sin t, 2\sin (t/2))$ and $h(s) = \cos^{-1} s$. Therefore $$\beta(s) = \alpha(h(s)) = \left(1 + s, \sin \cos^{-1} s, 2 \sin(\frac{\cos^{-1} s}{2}) \right) $$

Note that $s \in (0, 1)$ meaning $\cos^{-1}s = h$ can be positive or negative. So $ \sin \cos^{-1} s  = \pm \sqrt{1 - s^2}$. Similarly, $2 \sin(\frac{\cos^{-1} s}{2}) = \pm 2\sqrt{\frac{1 - s}{2}}$ by half angle formula of $\sin$. By restricting $h \geq 0$, we get 
$$\beta(s) = \left(1 + s, \sqrt{1 - s^2}, 2\sqrt{\frac{1 - s}{2}} \right)$$

\section*{1.4.4}
$\beta = \alpha(h(s)) = (s, s^{-1}, \sqrt{2}\log s)$. Then $\beta'(s) = (1, -s^{-2}, \sqrt{2} s^{-1})$.

By lemma 4.5 $\beta'(s) = (dh/ds) \alpha'(h(s)) = s^{-1}(e^t, -e^{-t}, \sqrt{2})|_{t=h(s)} = (1, -s^{-2}, \sqrt{2} s^{-1})$

\section*{1.4.5}
$l_1: (1, -3, -1) + t(6-1, 2+3, 1+1) = (5t+1, 5t-3, 2t-1)$.

$l_2: (-1, 1, 0) + s(-5+1, -1 - 1, -1) = (-4s-1, -2s+1, -s)$.

Suppose they meet, then we have 
\begin{align}
	5t + 1 &= -4s - 1 \\
	5t - 3 &= -2s + 1 \\
	2t - 1 &= -s \\
\end{align}
Solving the first two equation, we have $t=2, s= -3$. Putting them into the 3rd equation, we get $3 = 3$ which is consistent. They do meet.

\section*{1.4.6}
For any curve $\alpha(t)$ with initial velocity of $v_p$. Then $\alpha'(t)[f] = \frac{d(f(\alpha))}{dt} (t)$ by Lemma 4.6. Evaluating at 0, we get $$\begin{aligned}
	\alpha'(0)[f] = v_p [f] &= \frac{d(f(\alpha))}{dt}(0) \\
	   &= \sum_i \frac{\partial f}{\partial x_i}(\alpha(0)) \frac{d \alpha_i}{dt}(0) \\
	   &= \sum_i \frac{\partial f}{\partial x_i}(\alpha(0)) \alpha_i'(0) \\
	   &= \sum_i \frac{\partial f}{\partial x_i}(p)[v_p]_i \\
	   &= \sum_i \frac{\partial f}{\partial x_i}(p)\frac{d([p+tv_p]_i)}{dt} \\ 
	   &= \sum_i \frac{\partial f}{\partial x_i}(p+ tv_p)\frac{d([p+tv_p]_i)}{dt} \bigg|_{t=0} \\
	   &= \frac{df}{dt}(p + tv_p) \bigg|_{t=0}
\end{aligned}$$
\section*{1.4.7}
(a) $\frac{d}{dt}(t, 1 + t^2, t)\bigg|_{t=0} =(1, 2(0), 1) = (1, 0, 1)$ at point $(0, 1, 0)$.

$\frac{d}{dt}(\sin t, \cos t, t)\bigg|_{t=0} = (1, 0, 1)$ at point $(0, 1, 0)$.

$\frac{d}{dt}(\sinh t, \cosh t, t) \bigg|_{t=0} = (\cosh(0), \sinh(0), 1) = (1, 0, 1)$ at point $(0, 1, 0)$.

(b) $f = x^2 - y^2 + z^2$

$f(t) = f(t, 1+t^2, t) = t^2 - (1+t^2)^2 + t^2$. Then $\frac{df}{dt}\bigg|_0 = 4t-4t(1+t^2)\bigg|_0 =  0$

$f(t) = f(\sin t, \cos t, t) = \sin^2 t - \cos^2 t + t^2$. Then $\frac{df}{dt}\bigg|_0 = 2\sin^t \cos t + 2 \cos t \sin t + 2t \bigg|_0 = 0$.

$f(t) = f(\sinh t, \cosh t, t) = \sinh^2 t - \cosh^2 t + t^2$. Then $\frac{df}{dt}\bigg|_0 = 2\sinh t \cosh t - 2 \cosh t\sinh t + 2t \bigg|_0= 2t \bigg|_0 = 0$

\section*{1.4.8}
(a)$x=\frac{1}{2} \cos t$, $y=\sin t$.

(b) $x = t$, $y= (1 -3t)/4$.

(c) $x = t$, $y=e^t$.


\section*{1.4.9}
$\alpha(t) = (2\cos t, 2 \sin t, t)$,
$\alpha'(t) = (-2\sin t, 2\cos t, 1)$.

Line at $0$ is $u \rightarrow (2, 0, 0) + u(0, 2, 1)$

Line at $\pi/4$ is $v \rightarrow (\sqrt{2}, \sqrt{2}, \pi/4) + v(-\sqrt{2}, \sqrt{2}, 1)$

\section*{1.5.1}
 $p = (0,-2, 1)$, $v_p=(1,2, -3)$.
  
(a) $(y^2dx)(v_p) = y^2(p)dx(v_p) = (-2)^2 (1)= 4$

(b) $(zdy-ydz)(v_p) = z(p)dy(v_p) - y(p)dz(v_p) = (1)(2) - (-2)(-3) = -4 $

(c) $[(z^2-1)dx - dy + x^2dz](v_p) = (z^2-1)(p)dx(v_p) - dy(v_p) + x^2(p) dz(v_p) = -2$

\section*{1.5.2}
For any point $p$, $$\begin{aligned}
	\phi(V_p) &= (\sum_i f_i dx_i)(V_p) \\ 
	      &=\sum_i f_i(p) dx_i(V_p) \\ 
	      &= \sum_i f_i(p) dx_i(\sum_j v_j U_j) \\ 
	      &= \sum_i f_i(p) \sum_j v_j dx_i(U_j) \\ 
	      &= \sum_i f_i(p)\sum_j v_j \delta_{ij} \\ 
	      &= \sum_i f_i(p)v_i
\end{aligned} $$
Hence $\phi(V) = \sum_i f_i v_i$.

\section*{1.5.3}
$\phi = x^2dx - y^2dz$

(a) For $V=xU_1 + yU_2 + zU_3$,  $\phi(V) = x^3 - y^2z$.

(b) For $W= xy(U_1 - U_3) + yz(U_1 - U_2) = (xy+yz)U_1 - yzU_2 - xyU_3$, $\phi(W) = x^2y(x+z) +xy^3$

(c) For $T=(1/x)V + (1/y)W$,  $$\phi(T) = \phi((1/x)V + (1/y)W) = (1/x)\phi(V) + (1/y)\phi(W) = \frac{x^3-y^2z}{x} + x^2(x+z) + xy^2 $$

\section*{1.5.4}
(a) Since $df^2 = 2fdf$. Suppose $df^n = nf^{n-1}df$,  $$df^{n+1} = d(ff^n) = f^ndf + fd(f^n) = f^ndf + f(nf^{n-1}df) = (n+1)f^ndf$$. Therefore by induction, $df^n = nf^{n-1}df$.  As a result,
$$df^4 = 4f^3df$$

(b) $df = d(\sqrt{f} \sqrt{f}) =2 \sqrt{f}d (\sqrt{f})$ Then $$d (\sqrt{f}) = \frac{1}{2\sqrt{f}}df$$

(c) $d(\log(1+f^2)) = \frac{1}{1+f^2} d(1+f^2) = \frac{2f}{1+f^2} df$

\section*{1.5.5}
(a) $$df = \sum_i \partial_i f dx_i = \frac{xdx + ydy + zdz}{\sqrt{x^2+y^2+z^2}}$$

(b) $$ df = \frac{1}{1 + (y/x)^2} (-\frac{y}{x^2}dx + \frac{1}{x} dy) = \frac{1}{x^2 + y^2} (-ydx + xdy)$$

\section*{1.5.6}
$p = (0,-2, 1)$, $v_p=(1,2, -3)$.

(a) $df = y^2dx + (2xy-z^2) dy -2yzdz$. Then $df(v_p) = (-2)^2(1) + (-1)(2) -2(-2)(-3) = 10$

(b) $df = \exp(yz)(dx + xzdy + xydz)$. Then $df(v_p) = \exp(-2)$.

(c) $df = (y \cos(xy)\cos(xz) - z\sin(xy)\sin(xy))dx + x\cos(xy)\cos(xz)dy - x\sin(xy)\sin(xz)dz$. Then $df(v_p) = y(p)dx(v_p) = (-2)1 = -2$.

\section*{1.5.7}
$\phi(v_p) = \sum_i f_i(p) v_i$ for $\phi = \sum_i f_i dx_i$.

(a) Yes. $f_1(p) = 1, f_3(p) = -1$,  $\phi = dx - dz$

(b) No.  Because $dx_i(v_p)$ must involve $v_i$.

(c) Yes. $f_1(p) = p_3, f_2(p) = p_1$. So $\phi = zdx + xdy$.

(d) Yes. $df(v_p) = v_p(f) = v_p[x^2+y^2]$. Therefore $f = x^2+y^2$.  $\phi = df = 2xdx + 2ydy$.

(e) Yes. $\phi = 0$.

(f) No. Because $dx_i(v_p)$ must involve $v_i$.

\section*{1.5.8}
By definition of $d$. $df(v_p) = v_p[f]$. Then by theorem 3.3
$$ d(fg)(v_p) = v_p(fg) = f(p)v_p(g) + g(p) v_p(f) = f(p)dg(v_p) + g(p) df(v_p)$$

\section*{1.5.9}
Since $df = -2xydx + (1-x^2-2yz)dy + (1-y^2)dz$,  we need to find points such that $xy =0,  1-x^2-2yz =0, 1-y^2= 0$. From the last equation, we get $y=\pm 1$. Then the first equation gives $x=0$. Putting the values into the 2nd equations, we have $z = \frac{1-x^2}{2y} = \pm \frac{1}{2}$. So the critical points are $(0, 1, 1/2)$ and $(0, -1, -1/2)$

\section*{1.5.10}
Suppose $p$ is a local maxima of $f$ and $p$ is not a critical point of $f$, then there exists a tangent vector $v_p$ such that $$df(v_p) = \frac{df}{dt}(p+tv_p)\bigg|_0 = \lim_{t \rightarrow 0} \frac{f(p+tv_p) - f(p)}{t} > 0$$
This is true in general as we can always take the opposite direction $-v_p$ if $df(v_p) < 0$. Then we have $\frac{f(p+t_0v_p) - f(p)}{t_0} > \epsilon$ for some $\epsilon>0$ and $t_0$ which contradicts with the assumption that $p$ is local maxima of $f$. Same argument applies to local minima of $f$. \QED

\section*{1.5.11}
(a)
$(df)(v_p) = \sum_i \frac{\partial f}{\partial x_i} v_i$.  Taylor expanding $f(u)$ around $p$, we get $f(u) = f(p) + \sum_i \partial_i f (p)(u_i - p_i) + O(u^2)$. Let $u=p+v$, then $$f(p+v) = f(p) + \sum_i \partial_i f (p)v_i + O_2 = f(p) + df(v_p) + O_2$$.
Where $O_2$ is second order error. Therefore $f(p+v) - f(p) \approx df(v_p)$ in the first order. \QED

(b) $$df = (2xy/z) dx + (x^2/z) dy - (x^2y/z^2)dz$$. $p = (1,1.5,1)$ and $v_p=(-0.1, 0.1, 0.2)$.
We get $df(v_p) = 2(1.5)(-0.1) +0.1 - (1.5)(0.2) =-0.3 + 0.1 - 0.3 = -0.5$. 

Direct calculation gives $f(0.9, 1.6, 1.2) - f(1, 1.5, 1) = 1.08 - 1.5 = -0.42$

\section*{1.6.1}
Given $\phi = yzdx + dz$, $\psi = \sin zdx + \cos zdy$, $\xi = dy+zdz$.

(a)

 $\phi \wedge \psi = (yzdx + dz) \wedge ( \sin zdx + \cos zdy) = yz\cos z dx \wedge dy - \sin z dx \wedge dz - \cos z dy \wedge dz $

$ \psi \wedge \xi = (\sin zdx + \cos zdy) \wedge ( dy+zdz)  = \sin z dx \wedge dy + z \sin z dx \wedge dz + z \cos z dy \wedge dz $

$ \xi \wedge \phi = ( dy+zdz) \wedge ( yzdx + dz) = -yzdx \wedge dy + dy \wedge dz - yz^2 dx \wedge dz$

(b)

 $d\phi = d(yz)\wedge dx + d1 \wedge dz = (zdy + ydz)\wedge dx = -z dx \wedge dy - y dx \wedge dz$
 
 $d\psi = d(\sin z) \wedge dx + d(\cos z) \wedge dy = \cos z dz \wedge dx - \sin z dz \wedge dy$
 
 $d\xi = d1 \wedge dy+dz \wedge dz = 0$
 
\section*{1.6.2}
Given $\phi = dx /y $, $\psi = z dy$. Then $\phi \wedge \psi = (z/y) dx \wedge dy$. Directly computing the differential gives $$d(\phi \wedge \psi) = d(z/y) \wedge dx \wedge dy = (1/y dz - z/y^2 dy) \wedge dx \wedge dy = 1/y dx dy dz$$ Using theorem 6.4, 
$$ \begin{aligned}
	d(\phi \wedge \psi &= d\phi \wedge \psi - \phi \wedge d\psi \\
	  &= (- 1/y^2 dy \wedge dx) \wedge zdy - 1/y dx \wedge (dz \wedge dy) \\
	  &= 0 - 1/y dx dz dy \\
	  &= 1/y dx dy dz
\end{aligned}$$ 

\section*{1.6.3}
For any function $f$,  $df = \sum_i \partial_i f_i dx_i$. 
$$
  \begin{aligned}
  	d(df) &= \sum_i d(\partial_i f) \wedge dx_i  \\
  		  &= \sum_i (\sum_j \partial_i \partial_j f dx_j) \wedge dx_i \\
  		  &= \sum_{i\neq j} \partial_i \partial_j f dx_j \wedge dx_i \\
  		  &= - \sum_{i < j} \partial_i \partial_j f dx_i \wedge dx_j + \sum_{j < i} \partial_j \partial_i f dx_j \wedge dx_i \\
  		  &= 0
  \end{aligned}
$$

$d(fdg) = df \wedge dg + f d(dg) = df \wedge dg$.

\section*{1.6.4}
(a) $d(fdg + gdf) = df\wedge dg + dg \wedge df = 0$

(b) $d((f -g)(df + dg))= d(f-g) \wedge (df + dg) = (df - dg) \wedge (df + dg) = df \wedge dg - dg \wedge df = 2 df \wedge dg$

(c) $d(fdg \wedge g df) = d(fdg) \wedge gdf - fdg \wedge d(gdf) = df \wedge dg \wedge g df - f dg \wedge dg \wedge df = 0$

(d) $d(gfdf) + d(fdg) = d(gf) \wedge df + df \wedge dg = (fdg + gdf)\wedge df + df \wedge dg = (1 - f)df \wedge dg$

\section*{1.6.5}
$ \phi_i = \sum_j f_{ij} dx_j$.  If we do the wedge product, 
$$ \begin{aligned}
	\bigwedge_i \phi_i &= \bigwedge_i \sum_j f_{ij} dx_j \\
					   &= \left[\sum_{\sigma} \prod_i f_{i, \sigma_i} \right](dx_{\sigma_1} \wedge \cdots \wedge dx_{\sigma_n}) \\
					   &= \left[\sum_{\sigma} \mbox{sgn}(\sigma) \prod_i f_{i, \sigma_i} \right](dx_1 \wedge \cdots \wedge dx_n) \\
					   &= \det|f_{i,j}| dx_1 \wedge \cdots \wedge dx_n
\end{aligned} $$
Each term in the expanded expression is a product of picking a term in $\phi_i$ for each $i$. Since $dx_j \wedge dx_j$ vanishes, so we need to pick $n$ terms with different $j$ from $\phi_1$ to $\phi_n$ and hence the expression above. 

\section*{1.6.6}
In cylindrical coordinate, $x = r\cos \theta, y=r\sin \theta, z = z$. The volume element in canonical coordinate is $dx \wedge dy \wedge dz$. Then $$
\begin{aligned}
	dx \wedge dy \wedge dz &= (\cos\theta dr - r\sin \theta d\theta) \wedge (\sin \theta dr + r\cos\theta d\theta) \wedge dz \\
	  &= (r\cos^2 theta + r\sin^2\theta) dr \wedge d\theta \wedge dz   \\
	  &= r dr \wedge d\theta \wedge dz
\end{aligned}
$$

\section*{1.6.7}
Given a one-form $\phi = \sum_i f_i dx_i$, then  $$d(d\phi) = d(d(\sum_i f_i dx_i)) = d(\sum_i df_i \wedge dx_i) = \sum_i d(1)\wedge df_i \wedge dx_i = 0$$

\section*{1.6.8}
(a) $df = \sum_i \partial_i f dx_i$. $dx_i$ 1-1 to $U_i$. Therefore $df$ 1-1 to $\grad f$.

(b) $\{dx_i\}$ is 1-1 with $\{U_i\}$, for each $\phi = \sum_i f_i dx_i$ there exists $V= \sum_i f_i U_i$.  Then $$\begin{aligned}
	d\phi = \sum_i df_i \wedge dx_i &= \sum_i (\sum_j \partial_j f_i dx_j ) \wedge dx_i \\
	  &= \sum_i \sum_j \partial_j f_i dx_j \wedge dx_i \\
	  &= \sum_{i\neq j} \partial_j f_i dx_j \wedge dx_i \\
	  &= - \sum_{i < j} \partial_j f_i dx_i \wedge dx_j  +  \sum_{j < i} \partial_j f_i dx_j \wedge dx_i \\ 
	  &= - \sum_{i < j} \partial_j f_i dx_i \wedge dx_j  +  \sum_{i < j} \partial_i f_j dx_i \wedge dx_j \ \ \ \mbox{(swap i and j)}\\ 
	  &= \sum_{i < j} (\partial_i f_j - \partial_j f_i )dx_i \wedge dx_j  \\ 
\end{aligned} $$
This is equal to $\curl \phi$ when the dimension is 3 because there exists 1-1 mapping between $U_1,U_2,U_3$ and $dx_1\wedge dx_2, dx_2 \wedge dx_3, dx_1 \wedge dx_3$.

(c) For a vector field $V = f_1U_1+f_2U_2+f_3U_3$, there exists $\eta = f_3 dx_1dx_2  - f_2dx_1dx_2 + f_1dx_2dx_3$ by correspondence. Then 
$$
 \begin{aligned}
 	d\eta &= df_3 dx_1 dx_2 - df_2 dx_1 dx_2 + f_1 dx_2 dx_3 \\
 		  &= \partial_3 f_3 dx_3 dx_1 dx_2 - \partial_2 f_2 dx_2 dx_1 dx_3 + \partial_1 f_1 dx_2 dx_3 \\
 		  &= (\partial_3 f_3 + \partial_2 f_2 + \partial_1 f_1) dx_1 dx_2 dx_3 \\
 		  &= \div V dx_1 dx_2 dx_3
 \end{aligned}
$$


\section*{1.6.9}
$df = \partial_x f dx + \partial_y f dy$ and $dg = \partial_x g dx + \partial_y g dy$.

$ df\wedge dg = \partial_xf \partial_y g dx \wedge dy + \partial_y f\partial_x g dy \wedge dx = ( \partial_xf \partial_y g - \partial_y f\partial_x g) dx \wedge dy = 
\begin{vmatrix}
	\partial_x f & \partial_y f \\
	\partial_x g & \partial_y g \\
\end{vmatrix}
dx \wedge dy
$

\section*{1.7.1}
$F(u, v) = (u^2 -v^2, 2uv)$.

(a) $F(p)=(0, 0)$ then $(f_1(p), f_2(p)) = (p_1^2 - p_2^2, 2p_1p_2) = (0, 0)$. Therefore $p_1 = 0, p_2 = 0$.

(b) $F(p) = (8, 6)$ then $p_1^2 - p_2^2 = 8$ and $2p_1p_2=6$. Putting together two equations and factoring it, we have $(p_2^2 + 9)(p_2^2 - 1) = 0$. Therefore $p_1 = \pm 3, p_2 = \pm 1$.

(c) $F(p) = p$. Then $p_1^2 - p_2^2 = p_1$ and $2p_1p_2 = p_2$.

\section*{1.7.3}
$$\begin{aligned}
	 F_*(v) &= \frac{d}{dt}\big|_{t=0} F(p+ tv) \\ 
	   &= \frac{d}{dt}\big|_{t=0} ((p_1 + tv_1)^2 - (p_2 + tv_2)^2, 2(p_1+tv_1)(p_2+tv_2))  \\
	   &= ( 2v_1(p_1 + tv_1) - 2v_2(p_2 + tv_2), 2v_1(p_2+tv_2) + 2(p_1+tv_1)v_2)\big|_{t=0} \\
	   &= 2(v_1p_1 - v_2p_2, v_1p_2 + p_1v_2)
\end{aligned}$$

\section*{1.7.4}
$$
 F_{*, p} = \begin{pmatrix}
 	2u & 2v \\
 	-2v & 2u
 \end{pmatrix}
$$
$F_{*, p}$ only vanishes at the origin. 

\section*{1.7.5}
$F_*(v_p) = \frac{d}{dt}\big|_{t=0} F(p + tv) = \frac{d}{dt}\big|_{t=0}(F(p) + tF(v)) = F(v)$

\section*{1.7.6}
(a) Take $m=n=1$, for any mapping that is a joint of two line segments with different slope, the mapping is not differential at the point of the joints hence it is not a diffeomorphism.

(b) Suppose $F: \real^n \rightarrow \real^n$ is 1-1 and onto, by inverse function theorem, we can take the open set as the entire $\real^n$ since $F$ is onto. Then $F$ is diffeomorphism.

\section*{1.7.7}
$$
\begin{aligned}
	v_p[g(F)] &= \sum^n_{i} v_i U_i[g(F)] \\ 
		&= \sum^n_{i} v_i \frac{\partial}{\partial x_i}[g(F)] \\ 
	   &= \sum^n_{i} v_i \sum^m_j \frac{\partial g}{\partial f_j} \frac{\partial f_j}{\partial x_i} \\
	   &= \sum^m_j  \left[ \sum^n_{i} v_i \frac{\partial f_j}{\partial x_i} \right] \frac{\partial g}{\partial f_j}  \\
	   &= \sum^m_j  \left[ F_*(v_p) \right]_j \frac{\partial g}{\partial f_j}  \\
	   &= F_*(v_p)[g]  \\
\end{aligned}
$$

\section*{1.7.8}
Given any curve $\alpha(t)$ such that $\alpha'(0) = v_p$.
$$
\begin{aligned}
	F_*(v_p) = \frac{d}{dt}\big|_{t=0}F(\alpha(t)) = \sum_i^n \frac{\partial F}{\partial \alpha_i}\frac{d\alpha_i}{dt}\bigg|_{t=0} = \sum_i^n [v_p]_i \frac{\partial F}{\partial \alpha_i}
\end{aligned}
$$
By proof of proposition 7.5, the expression above can give back the line definition of the push forward of $F$.

\section*{1.7.9}
(a) $GF = (g_i \circ f_i)$. Since $g_i, f_i$ are differentiable, so $g_i \circ f_i$ are differentiable. Hence $GF$ is a mapping.

(b) $$[(GF)_*]_{ij} = \frac{\partial g_i \circ f_i }{\partial x_j} = \sum_k^m \frac{\partial g_i}{\partial f_k} \frac{\partial f_k}{\partial x_j} = [G_*]_{ik}[F_*]_{kj}$$

(c) $F$ is diffeomorpohism means $F^{-1}$ exists and is differentiable. Then since $F$ is differentiable itself, $F^{-1}$ also has a differentiable inverse therefore $F^{-1}$ is also a diffeomorpishm.

\section*{1.7.10}
(a) Suppose $F(u_1, v_1) = F(u_2, v_2)$,  then $ v_1 \exp u_1 = v_2 \exp u_2$ and $2 u_1 = 2u_2$. It is obvious that $v_1 = v_2$ as well. So $F$ is 1-1. 

For any point $(y_1, y_2)$ in $\real^2$,  Let $u = y_2 / 2$, $v = y_1/\exp(-y_2 / 2)$, then $F(u, v) = (y_1, y_2)$. So $F$ is onto.

$|F_*| = \begin{vmatrix}
	ve^u & e^u \\
	2 & 0 \\
\end{vmatrix} = 2\exp(u) \neq 0$ So $F$ is regular. These properties hold in $\real^2$ so by inverse function theorem,  $F$ is diffeomorphism. 

(b) $F(u, v) = (x, y)$. Then $v e^u = x$ and $2u = y$. Expressing $u, v$ in terms of $x,y$,  $u = y/2, v = x \exp(- y/2)$. We get the inverse 
$$ F^{-1}(x, y) = (0.5y, x \exp(-y/2))$$

$F^{-1}\circ F (u, v) = F^{-1}(ve^u, 2u) = ((0.5)2u, v\exp(u) \exp(- 2u / 2)) = (u, v) $

$F \circ F^{-1}(x, y) = F(0.5y, x \exp(-y/2)) = (x\exp(-y/2) \exp(0.5y), 0.5y (2)) = (x, y)$
\end{document}