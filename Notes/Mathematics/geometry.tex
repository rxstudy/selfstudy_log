\documentclass[12pt]{article}
\usepackage{multicol}
\usepackage[margin=0.4in]{geometry}
\include{pythonlisting}
\usepackage{fullpage}
\usepackage{times}
\usepackage[normalem]{ulem}
\usepackage{multirow}
\usepackage{fancyhdr,graphicx,amsmath,amssymb, mathtools, scrextend, titlesec, enumitem}\usepackage[pdftex]{hyperref}
\usepackage[ruled,vlined]{algorithm2e} 
\usepackage{parskip}
\usepackage{listings}
\newcommand{\Var}{\mathrm{Var}}
\newcommand{\Cov}{\mathrm{Cov}}
\newcommand{\E}{{\rm I\kern-.3em E}}
\newcommand{\Binomial}{\mathrm{Binomial}}
\newcommand{\Beta}{\mathrm{Beta}}
\newcommand{\Poisson}{\mathrm{Poisson}}
\newcommand{\Normal}{\mathcal{N}}
\newcommand{\Uniform}{\mathrm{Uniform}}

\lstset{frame=tb,
  language=Python,
  aboveskip=3mm,
  belowskip=3mm,
  showstringspaces=false,
  columns=flexible,
  basicstyle={\small\ttfamily},
  numbers=none,
  stringstyle=\color{mauve},
  breaklines=true,
  breakatwhitespace=true,
  tabsize=3
}
\title{Geometry Note}
\author{Ran Xie}
\begin{document}
\maketitle

\twocolumn
\section{Definitions}

\textbf{Space of linear function} $L(V, W)$ vector space of linear functions from $V$ to $W$.

\textbf{Dual Space} $V^* = L(V, R)$. For each basis $\{e_i\}$ of $V$, there exists unique $\{e^i\}$ of $V^*$ such that $e^i (e_j) = \delta^i_j$

\textbf{Tensor Space} $T^r_s = \underbrace{V \otimes V}_\text{r times} \otimes \underbrace{V^* \otimes V^*}_\text{s times}  $  is space of multilinear functions on $$\underbrace{V^* \cross \ldots \cross V^*}_\text{r times} \cross \underbrace{V \cross \ldots \cross V}_\text{s times}$$

\textbf{Tensor Product} between $A$ of $(r,s)$ and $B$ of $(t, u)$,  is $$\begin{aligned}  A \otimes B &(\tau^1,\ldots, \tau^{r+t}, v_1,\ldots, v_{s+u}) \\ & = A(\tau^1, \ldots, \tau^r, v_1, \ldots, v_s) \\ &B(\tau^{r+1}, \ldots , \tau^{r+t}, v_{s+1}, \ldots, v_{s+u})\end{aligned}$$

\textbf{Vector Field} $X$ on coordinate neighborhood $U$ of a manifold $M$,  with coordinate $x^i$. For each point $p$, $X = X^i \partial_i$.  $X[f] = X^i \partial_i f$

\textbf{Change of Coordinates} If $Y$ has coordinate neighborhood $V$ of $y^i$, then $Y^i = X^j \frac{\partial y^i }{\partial x^j}$

\textbf{Map Differential(Pushforward)} $F_*$ is induced map $F_*: TM \rightarrow TN$ of $C^\infty$ map $F: M\rightarrow N$. $F_* (v_p) = (F_*v)_{F(p)}$. With coordinate, $F_* = [ \partial_j ( y^i \circ F ) ]$, the Jacobian of $F$. Note that $y^i \circ F = F^i(x^1, \ldots, x^m)$

\textbf{Tensor Bundle} $T^r_s M$  of type $(r,s)$ is the union of all tensor spaces $M^r_s(p)$ at each point $p \in M$.  

\textbf{Tangent Bundle} $TM=T^1_0M$, 

\textbf{Scalar Bundle} $T^0_0M = M \cross \real$, 

\textbf{Cotangent Bundle/ Differentials / Phase space} $T^0_1M$

\textbf{Tensor Field} $T$ of type $(r,s)$, $T(p) \in T^r_s M (p)$ for each $p$. $(1,0)$  is vector field, $(0,0)$ gives real-valued function. $(0, 1)$ gives differential.

\textbf{Tensor Coordinate} of $T^r_s$ wrt coordinate $x^i$ are $d^{r+s}$ real-valued functions $$T^{i_1\ldots i_r}_{j_1\ldots j_s} = T(dx^{i_1}, \ldots dx^{i_r}, \partial_{j_1}, \ldots , \partial_{j_s})$$

\textbf{Tensor Product}

\textbf{Exterior Product}

\textbf{Differential forms} p-form is $C^\infty$ skew-symmetric covariant tensor field of degree $p$ (type $(0,p)$). Local basis has ${d \choose p}$ p-forms $dx^{i_1}\cdots dx^{i_p}$ where $(i_1,\ldots,i_p)$ is increasing.

\onecolumn
\section*{Case Study 1: Surface of a sphere}
The surface of sphere of radius 1 is a manifold $$S^2 = \{(x, y, z) \in \Real^3 | x^2+y^2+z^2 = 1 \}$$
We can define a chart $(U, \psi)$ for $S^2$ where $U\subseteq M$ with spherical coordinate.  Let $$U = \left\{(\theta, \phi) \in [0, 2\pi]\times [0, \pi] \right\} $$
and $$
\psi(x,y,z): \begin{cases}
	\theta = \arccos(z) \\
	\phi = \text{sng}(y) \arccos \frac{x}{\sqrt{x^2 + y^2}}
\end{cases}, 
\psi^{-1}(\theta, \phi): 
\begin{cases}
	x &= \sin\theta \cos\phi \\
	y &= \sin\theta \sin\phi \\
	z &= \cos \theta
\end{cases} 
$$
Then $\psi(U) \subseteq \Real^2$ is a homeomorphism from $U$ to $\psi(U)$. $\psi$ is called a \textbf{Locale coordinate map}. And the component functions $(\theta, \phi)$ defined by $\psi(p) = (\theta(p), \phi(p))$ for $p \in S^2$ are called \textbf{local coordinates} on $U$.

One can think of this as giving a temporary identification between $U$ and $\psi(U)$. When we work in this chart, we can think of $U$ as an open subsets of the manifold and as an open subset of $\Real^2$. Thus, we can represent a point $p \in U \subseteq S^2$ by its coordinate $(\theta, \phi) = \psi(p)$ and think of it as being the point $p$. We say $(\theta, \phi)$ is the local coordinate for $p$ or $p = (\theta, \phi)$ in local coordinates. \textit{(See Lee's Smooth Manifold Local Coordinate Representations section)}

Given the same chart, the coordinate vectors $\partial_\theta, \partial_\phi$ form a basis for $T_pS^2$. If $v \in T_pS^2$, then  $$v = v^1 \frac{\partial}{\partial \theta}\bigg\rvert_p + v^2 \frac{\partial}{\partial \phi}\bigg\rvert_p = v^1\partial_\theta + v^2 \partial_\phi = v^i \partial_i $$

The dual space to $T_pS^2$ is $T^*_pS^2$, if $w \in T^*_pS^2$, $$w = w_1 d\theta + w_2 d\phi = w_i dx^i \text{(in generic coordinates)}$$
and $w(v) = w_i v^i$

$S^2$ is Riemannian with symmetric metric tensor defined as 
$$
\begin{aligned}
	g &= g_{ij}dx^i \otimes dx^j  \\
	  &= g_{11}d\theta\otimes d\phi + g_{12}d\theta \otimes d\phi + g_{21}d\phi \otimes d\theta + g_{22} d\phi \otimes d\phi \\
	  &= g_{11}(d\theta)^2 + \frac{1}{2}(g_{12} + g_{21}) d\theta \otimes d\phi +  \frac{1}{2}(g_{21} + g_{12})  d\phi\otimes d\theta +  g_{22} (d\phi)^2 \qquad ,(g_{12} = g_{22}) \\
	  &= g_{11}(d\theta)^2 + \frac{g_{12}}{2}(d\theta \otimes d\phi + d\phi\otimes d\theta) + \frac{g_{21}}{2}(d\phi\otimes d\theta + d\theta \otimes d\phi) + g_{22} (d\phi)^2 \\
	  &= g_{11}(d\theta)^2 + g_{12}d\theta d\phi + g_{21}d\phi d\theta + g_{22} (d\phi)^2 \\
	  &= g_{ij}dx^i dx^j
\end{aligned}$$

We will now compute $g$. Since $(\theta, \phi)$ are local coordinate of $S^2$, we can introduce a smooth immersion map $\iota = \psi^{-1} = (\sin\theta \cos\phi, \sin\theta\sin\phi, \cos\theta)$ into $\Real^3$. Since $\Real^3$ has Euclidean metric $\bar{g} = (dx)^2 + (dy)^2 + (dz)^2$, then $g$ is the pullback of $\bar{g}$,
$$\begin{aligned} 
	g &= \iota^*\bar{g} \\
	  &= (d(\sin\theta \cos\phi))^2 + (d(\sin\theta\sin\phi))^2 + (d(\cos\theta))^2\\
	  &= (\cos\theta \cos\phi d\theta - \sin\theta \sin\phi d\phi)^2 + (\cos\theta\sin\phi d\theta + \sin\theta\cos\phi d\phi)^2 + (\sin\theta d\theta))^2\\
	  &= (d\theta)^2 + \sin^2\theta (d\phi)^2
\end{aligned} 
$$


\section*{Case Study 2: Relativistic length contraction}
Given a stationary frame, it has Minkowski flat metric of $(d\tau)^2 = (dt)^2 - (dx)^2$.
A measure of length between location $A$ and $B$ along $x$-axis in a stationary frame are the distance between two simultaneous events $E_A = (t_0, x_a)^T$ and $E_B = (t_0, x_b)^T$. We want to calculate the distance with respect to a moving frame with constant velocity.

Let $S$ be the stationary frame with axis $(t, x)$ and $S'$ with axis $(\bar{t}, \bar{x})$ be the moving frame in the $x$-direction with speed $v$. The line element $d\tau$ is invariant in different frames, therefore $(dt)^2 - (dx)^2 = (d\tau)^2 = (d\bar{t})^2 - (d\bar{x})^2$. So we have $(t)^2 - (x)^2 = (\bar{t})^2 - (\bar{x})^2$. The solution is given by $$
 \begin{aligned}
 	t &= \bar{t}\cosh \theta + \bar{x} \sinh\theta   & \\
 	x &= \bar{t}\sinh \theta + \bar{x} \cosh\theta   & 
 \end{aligned}
$$
The trajectory of the origin of $S'$ along x axis is $x(t) = vt$ in frame $S$ but $\bar{x}(\bar{t}) = 0$ in $S'$ after the above transformation. So the above transformation is mapping $(\bar{t}, 0)$ to $(t, vt)$. Substituting those in the solution, then we have $v = \tanh\theta \equiv \beta$. From that, we have $\cosh \theta = \frac{1}{\sqrt{1 - v^2}} \equiv \gamma$, $\sinh\theta = \frac{v}{\sqrt{1 - v^2}} = \gamma\beta$ and . We arrive at Lorentz transform from $S \rightarrow S'$: $$\Lambda = \begin{pmatrix}
	\cosh\theta & \sinh\theta \\
	\sinh\theta & \cosh\theta \\ 
\end{pmatrix}^{-1}
= \begin{pmatrix}
	\cosh\theta & - \sinh\theta \\
	-\sinh\theta & \cosh\theta \\ 
\end{pmatrix}
= \begin{pmatrix}
	\gamma & -\gamma\beta \\
	-\gamma \beta & \gamma \\
\end{pmatrix}
$$

Now consider the world line of $A$ and $B$ in $S$,  
$$W_A(t) = \begin{pmatrix}
	t \\ x_a
\end{pmatrix},  
W_B(t) =  \begin{pmatrix}
t \\  x_b
\end{pmatrix}$$. After Lorentz transform, $$
\widehat{W}_A(t) = \begin{pmatrix}
 \gamma t - \gamma \beta x_a \\
 - \gamma\beta t + \gamma x_a
\end{pmatrix},
\widehat{W}_B(t) = \begin{pmatrix}
	\gamma t - \gamma \beta x_b \\
	- \gamma\beta t + \gamma x_b
\end{pmatrix}
$$
As we can see when $E_A$ and $E_B$ that are simultaneous in $S$ is not simultaneous in $S'$ because their time component is not the same. To measure the distance in $S'$, we will choose two events along the world line of $A$ and $B$ with the same time component in $S'$ that is $\widehat{W}_A^{(0)}(t_a) = \widehat{W}_B^{(0)}(t_b)$. Therefore we have $\gamma t_a  -  \gamma\beta x_a = \gamma t_b - \gamma \beta x_b$. We can choose $t_a = \beta x_a$ and $t_b = \beta x_b$ which is their  $\bar{x}$-intercept in $S'$.

Then the distance measured in $S'$ is $$\begin{aligned}
\widehat{L} & =  \widehat{W}_B^{(1)}(\beta x_b) - \widehat{W}_A^{(1)}(\beta x_a)\\
   & = -\gamma\beta^2 x_b + \gamma x_b + \gamma \beta^2 x_a - \gamma x_a  = \gamma(1 - \beta^2)(x_b - x_a) \\
   &= \frac{x_b - x_a}{\gamma}\\
   &= \frac{L}{\gamma}
\end{aligned}$$

\end{document}