\documentclass[12pt]{article}
\usepackage{multicol}
\usepackage[margin=0.4in]{geometry}
\include{pythonlisting}
\usepackage{fullpage}
\usepackage{times}
\usepackage[normalem]{ulem}
\usepackage{multirow}
\usepackage{fancyhdr,graphicx,amsmath,amssymb, mathtools, scrextend, titlesec, enumitem}\usepackage[pdftex]{hyperref}
\usepackage[ruled,vlined]{algorithm2e}
\usepackage{parskip}
\usepackage{listings}
\usepackage{amsmath}
\usepackage{physics}
\usepackage{bbm}
\usepackage{titling}
\usepackage{bm}
\usepackage{geometry}
\geometry{top=1cm, left=1cm, bottom=1.5cm, right=1.5cm, margin=1cm}
\newcommand{\varvec}[2][n]{#2_1,\ldots, #2_#1}
\newcommand{\Var}{\mathrm{Var}}
\newcommand{\Bias}{\mathrm{Bias}}
\newcommand{\Cov}{\mathrm{Cov}}
\newcommand{\E}{{\rm I\kern-.3em E}}
\newcommand{\Binomial}{\mathrm{Binomial}}
\newcommand{\Bernoulli}{\mathrm{Bernoulli}}
\newcommand{\Poisson}{\mathrm{Poisson}}
\newcommand{\Normal}{\mathcal{N}}
\newcommand{\one}{\mathbbm{1}}
\newcommand{\Beta}{\text{Beta}}
\newcommand{\BetaPDF}{\text{Beta}}
\newcommand{\GammaPDF}{\text{Gamma}}
\newcommand{\Uniform}{\mathrm{Uniform}}
\newcommand{\QED}{\newline \mbox{} \hfill $\blacksquare$}
\newcommand{\Real}{\mathbb{R}}
\newcommand{\Sgn}{\mbox{Sgn}}
\renewcommand*{\arraystretch}{1.4}
\newtheorem{theorem}{Result}
\newtheorem{definition}{Definition}
\lstset{frame=tb,
	language=Python,
	aboveskip=3mm,
	belowskip=3mm,
	showstringspaces=false,
	columns=flexible,
	basicstyle={\small\ttfamily},
	numbers=none,
	stringstyle=\color{mauve},
	breaklines=true,
	breakatwhitespace=true,
	tabsize=3
}
\title{Geometry Note}
\author{Ran Xie}
\begin{document}
\maketitle

\twocolumn
\section{Definitions}

\textbf{Space of linear function} $L(V, W)$ vector space of linear functions from $V$ to $W$.

\textbf{Dual Space} $V^* = L(V, R)$. For each basis $\{e_i\}$ of $V$, there exists unique $\{e^i\}$ of $V^*$ such that $e^i (e_j) = \delta^i_j$

\textbf{Tensor Space} $T^r_s = \underbrace{V \otimes V}_\text{r times} \otimes \underbrace{V^* \otimes V^*}_\text{s times}  $  is space of multilinear functions on $$\underbrace{V^* \cross \ldots \cross V^*}_\text{r times} \cross \underbrace{V \cross \ldots \cross V}_\text{s times}$$

\textbf{Tensor Product} between $A$ of $(r,s)$ and $B$ of $(t, u)$,  is $$\begin{aligned}  A \otimes B &(\tau^1,\ldots, \tau^{r+t}, v_1,\ldots, v_{s+u}) \\ & = A(\tau^1, \ldots, \tau^r, v_1, \ldots, v_s) \\ &B(\tau^{r+1}, \ldots , \tau^{r+t}, v_{s+1}, \ldots, v_{s+u})\end{aligned}$$

\textbf{Vector Field} $X$ on coordinate neighborhood $U$ of a manifold $M$,  with coordinate $x^i$. For each point $p$, $X = X^i \partial_i$.  $X[f] = X^i \partial_i f$

\textbf{Change of Coordinates} If $Y$ has coordinate neighborhood $V$ of $y^i$, then $Y^i = X^j \frac{\partial y^i }{\partial x^j}$

\textbf{Map Differential(Pushforward)} $F_*$ is induced map $F_*: TM \rightarrow TN$ of $C^\infty$ map $F: M\rightarrow N$. $F_* (v_p) = (F_*v)_{F(p)}$. With coordinate, $F_* = [ \partial_j ( y^i \circ F ) ]$, the Jacobian of $F$. Note that $y^i \circ F = F^i(x^1, \ldots, x^m)$

\textbf{Tensor Bundle} $T^r_s M$  of type $(r,s)$ is the union of all tensor spaces $M^r_s(p)$ at each point $p \in M$.  

\textbf{Tangent Bundle} $TM=T^1_0M$, 

\textbf{Scalar Bundle} $T^0_0M = M \cross \real$, 

\textbf{Cotangent Bundle/ Differentials / Phase space} $T^0_1M$

\textbf{Tensor Field} $T$ of type $(r,s)$, $T(p) \in T^r_s M (p)$ for each $p$. $(1,0)$  is vector field, $(0,0)$ gives real-valued function. $(0, 1)$ gives differential.

\textbf{Tensor Coordinate} of $T^r_s$ wrt coordinate $x^i$ are $d^{r+s}$ real-valued functions $$T^{i_1\ldots i_r}_{j_1\ldots j_s} = T(dx^{i_1}, \ldots dx^{i_r}, \partial_{j_1}, \ldots , \partial_{j_s})$$

\textbf{Tensor Product}

\textbf{Exterior Product}

\textbf{Differential forms} p-form is $C^\infty$ skew-symmetric covariant tensor field of degree $p$ (type $(0,p)$). Local basis has ${d \choose p}$ p-forms $dx^{i_1}\cdots dx^{i_p}$ where $(i_1,\ldots,i_p)$ is increasing.

\onecolumn
\section{Case Study 1: Surface of a sphere}
The surface of sphere of radius 1 is a manifold $$S^2 = \{(x, y, z) \in \Real^3 | x^2+y^2+z^2 = 1 \}$$
We can define a chart $(U, \psi)$ for $S^2$ where $U\subseteq M$ with spherical coordinate.  Let $$U = \left\{(\theta, \phi) \in [0, 2\pi]\times [0, \pi] \right\} $$
and $$
\psi(x,y,z): \begin{cases}
	\theta = \arccos(z) \\
	\phi = \text{sng}(y) \arccos \frac{x}{\sqrt{x^2 + y^2}}
\end{cases}, 
\psi^{-1}(\theta, \phi): 
\begin{cases}
	x &= \sin\theta \cos\phi \\
	y &= \sin\theta \sin\phi \\
	z &= \cos \theta
\end{cases} 
$$
Then $\psi(U) \subseteq \Real^2$ is a homeomorphism from $U$ to $\psi(U)$. $\psi$ is called a \textbf{Locale coordinate map}. And the component functions $(\theta, \phi)$ defined by $\psi(p) = (\theta(p), \phi(p))$ for $p \in S^2$ are called \textbf{local coordinates} on $U$.

One can think of this as giving a temporary identification between $U$ and $\psi(U)$. When we work in this chart, we can think of $U$ as an open subsets of the manifold and as an open subset of $\Real^2$. Thus, we can represent a point $p \in U \subseteq S^2$ by its coordinate $(\theta, \phi) = \psi(p)$ and think of it as being the point $p$. We say $(\theta, \phi)$ is the local coordinate for $p$ or $p = (\theta, \phi)$ in local coordinates. \textit{(See Lee's Smooth Manifold Local Coordinate Representations section)}

Given the same chart, the coordinate vectors $\partial_\theta, \partial_\phi$ form a basis for $T_pS^2$. If $v \in T_pS^2$, then  $$v = v^1 \frac{\partial}{\partial \theta}\bigg\rvert_p + v^2 \frac{\partial}{\partial \phi}\bigg\rvert_p = v^1\partial_\theta + v^2 \partial_\phi = v^i \partial_i $$



\end{document}