\documentclass[12pt]{article}

\usepackage{fullpage}
\usepackage{times}
\usepackage[normalem]{ulem}
\usepackage{multirow}
\usepackage{fancyhdr,graphicx,amsmath,amssymb, mathtools, scrextend, titlesec, enumitem}\usepackage[pdftex]{hyperref}
\usepackage[ruled,vlined]{algorithm2e}
\usepackage{parskip}
\usepackage{listings}
\usepackage{amsmath}
\usepackage{physics}
\usepackage{bbm}
\usepackage{titling}
\usepackage{bm}
\usepackage{geometry}
\geometry{top=1cm, left=1cm, bottom=1.5cm, right=1.5cm, margin=1cm}
\newcommand{\varvec}[2][n]{#2_1,\ldots, #2_#1}
\newcommand{\Var}{\mathrm{Var}}
\newcommand{\Bias}{\mathrm{Bias}}
\newcommand{\Cov}{\mathrm{Cov}}
\newcommand{\E}{{\rm I\kern-.3em E}}
\newcommand{\Binomial}{\mathrm{Binomial}}
\newcommand{\Bernoulli}{\mathrm{Bernoulli}}
\newcommand{\Poisson}{\mathrm{Poisson}}
\newcommand{\Normal}{\mathcal{N}}
\newcommand{\one}{\mathbbm{1}}
\newcommand{\Beta}{\text{Beta}}
\newcommand{\BetaPDF}{\text{Beta}}
\newcommand{\GammaPDF}{\text{Gamma}}
\newcommand{\Uniform}{\mathrm{Uniform}}
\newcommand{\QED}{\newline \mbox{} \hfill $\blacksquare$}
\newcommand{\Real}{\mathbb{R}}
\newcommand{\Sgn}{\mbox{Sgn}}
\renewcommand*{\arraystretch}{1.4}
\newtheorem{theorem}{Result}
\newtheorem{definition}{Definition}
\lstset{frame=tb,
	language=Python,
	aboveskip=3mm,
	belowskip=3mm,
	showstringspaces=false,
	columns=flexible,
	basicstyle={\small\ttfamily},
	numbers=none,
	stringstyle=\color{mauve},
	breaklines=true,
	breakatwhitespace=true,
	tabsize=3
}

\title{Probability and Statistics Notes}
\begin{document}
	\maketitle
\section{Sum of random variables}
Often we want to find the distribution of sums of two independent r.v $U = X + Y$. There are a few approaches 
\subsection{PDF change of variables}
This approach is based on the probability "volume" being equal after variable change (e.g. $f(x)dx = g(u)du \rightarrow g(u) = f(x(u)) \left| \frac{dx}{du}\right|$ )We can choose another $V=X$,  then $$g(u, v) = f(x(u, v), y(u, v)) \left| \frac{\partial(x, y)}{\partial(u, v)} \right|$$. So the pdf of $U$ is $g(u) = \int g(u, v) dv$. Note that the choice of $V$ is important as sometimes it will make $\int g(u, v) dv$ go unbounded (e.g. when $X$ and $Y$ are i.i.d exponential r.v, the choice of $V = X$ will make the integral infinite).

\subsection{Find CDF}
When it is hard to choose $V$, we can try finding the CDF of $U$ directly. $$P(U=X+Y < u) = \int\int_{(x,y) \in Q} f(x, y) dxdy $$
where $Q=\{(x, y) | x + y < u \}$. Note that $x + y = u$ is a line with negative slope that intercept $x$ and $y$ axis at $u$. Then we can describe the region of $Q$ and translate it into the integration bound. 

\subsection{Compute MGF}
If $X$ and $Y$ are two r.v and for all $t$, $M_X(t) = M_Y(t)$ then $F_X(x) = F_Y(x)$ for all values of $x$ ($X,Y$ have the same distribution). We can use this property to calculate the MGF of $X+Y$ and compared it with known MGF of other distributions. \textbf{}

\section{Hypothesis testing}
\subsection{Concepts}
\begin{enumerate}
	\item To study a population, we formulate hypothesis about the parameters $H_0$ and $H_1$. 
	\item We create test (in this case $LRT(x) = \lambda(x)$) to test the hypothesis by inputting the data into the test.
	\item We can compute power function of the test $P_\theta(X\in R) = P_\theta(\lambda(x) < c)$ for some $c$ (measures type I error). Note that the $c$ controls the rejection region $R$ here.
	\item We define the level/size of the test $\alpha$ to be $$\sup_{\theta \in \Theta_0}P_\theta(X\in R) = \sup_{\theta \in \Theta_0} P_\theta(\lambda(x) < c) = \alpha  \quad (\text{ or } \leq \alpha )$$ 
	we usually choose $\alpha = 0.05$ to suppress type I error to 5\%
	\item As we can see, $c$ is related to $\alpha$, we can find $c$ by solving the above inequality.
	\item Once we have $c$,  the rejection region is determined, we can input data into $\lambda(x)$ and test it against $c$. If $\lambda(x) < c$, we reject $H_0$.
	\item Test is chosen to minimize both type I and type II errors ($P_\theta(X\in R)$ and $P_\theta(X\in R^c)$)
\end{enumerate}



\end{document}