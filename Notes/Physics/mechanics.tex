\documentclass[12pt]{article}

\include{pythonlisting}
\usepackage{fullpage}
\usepackage{times}
\usepackage[normalem]{ulem}
\usepackage{multirow}
\usepackage{fancyhdr,graphicx,amsmath,amssymb, mathtools, scrextend, titlesec, enumitem}\usepackage[pdftex]{hyperref}
\usepackage[ruled,vlined]{algorithm2e}
\usepackage{parskip}
\usepackage{listings}
\usepackage{amsmath}
\usepackage{physics}
\usepackage{bbm}
\usepackage{titling}
\usepackage{bm}
\usepackage{geometry}
\geometry{top=1cm, left=1cm, bottom=1.5cm, right=1.5cm, margin=1cm}
\newcommand{\varvec}[2][n]{#2_1,\ldots, #2_#1}
\newcommand{\Var}{\mathrm{Var}}
\newcommand{\Bias}{\mathrm{Bias}}
\newcommand{\Cov}{\mathrm{Cov}}
\newcommand{\E}{{\rm I\kern-.3em E}}
\newcommand{\Binomial}{\mathrm{Binomial}}
\newcommand{\Bernoulli}{\mathrm{Bernoulli}}
\newcommand{\Poisson}{\mathrm{Poisson}}
\newcommand{\Normal}{\mathcal{N}}
\newcommand{\one}{\mathbbm{1}}
\newcommand{\Beta}{\text{Beta}}
\newcommand{\BetaPDF}{\text{Beta}}
\newcommand{\GammaPDF}{\text{Gamma}}
\newcommand{\Uniform}{\mathrm{Uniform}}
\newcommand{\QED}{\newline \mbox{} \hfill $\blacksquare$}
\newcommand{\Real}{\mathbb{R}}
\newcommand{\Sgn}{\mbox{Sgn}}
\renewcommand*{\arraystretch}{1.4}
\newtheorem{theorem}{Result}
\newtheorem{definition}{Definition}
\lstset{frame=tb,
	language=Python,
	aboveskip=3mm,
	belowskip=3mm,
	showstringspaces=false,
	columns=flexible,
	basicstyle={\small\ttfamily},
	numbers=none,
	stringstyle=\color{mauve},
	breaklines=true,
	breakatwhitespace=true,
	tabsize=3
}

\title{Mechanics}
\begin{document}
	\maketitle
	
\section{Basic Elementary Principles}
\subsection{Single particle}
Let $\bb{r}$ be radius vector of a particle from given origin and $\bb{v}$ its velocity vector:
$$ \bb{v} = \frac{d \bb{r}}{dt}$$

acceleration is given by $$ \bb{a} = \frac{d^2 \bb{r}}{d t^2}$$

Linear momentum $\bb{p} \equiv m \bb{v}$. Vector sum of forces exerted on the particle is total force $\bb{F}$,  
$$\bb{F} = \frac{d\bb{p}}{dt} \equiv  \bb{\dot p} \quad \quad \text{(Newton's second law)}$$

Conversation Theorem for linear momentum: If $\bb{F} = 0$, then $\bb{\dot p} = 0$ hence conserved.

Angular momentum of the particle around point $O$ is $\bb{L} = \bb{r} \times \bb{p}$. Torque $\bb{N} = \bb{r} \times \bb{F}$.
We can relationship $$ \bb{N} = \frac{d}{dt} (\bb{r} \times m\bb{v}) = \frac{d \bb{L}}{dt} \equiv \bb{\dot L}$$

Conservation Theorem for angular momentum: If $\bb{N}$ is zero then $\bb{\dot L} = 0$, hence conserved.

If force field does the same work for any possible path between point 1 and 2 ($W_{12} = \int^2_1 \bb{F}\cdot d \bb{s})$), 
$\bb{F}$ is conservative or $\oint \bb{F} \cdot d \bb{s} = 0$. As a result, 
there exists a potential scalar field $V$ such that $$\bb{F} = -\grad{V(\bb{r})}$$

Energy Conservation Theorem for a particle: If $\bb{F}$ acting on a particle is conservative, then $E = T + V$ is conserved.

\subsection{System of particles}
Let $\bb{F}^{(e)}_i$ external force acting on particle $i$-th, and $\bb{F}_{ji}$ is the force 
exerted by $j$-th particle on $i$-th particle in the system, 
$$\sum_j \bb{F}_{ji} +  \bb{F}^{(e)}_i = \bb{\dot{p}}_i$$
Summing over all particles, we get  \begin{align}
	& \frac{d^2}{dt^2} \sum_i m_i \bb{r}_i = \sum_i \bb{F}^{(e)}_i + \sum_{i \neq j} \bb{F}_{ji} \\
	\Rightarrow & M \frac{d^2 \bb{R}}{dt^2} = \sum_i \bb{F}^{(e)}_i \equiv \bb{F}^{(e)}
\end{align}
where $R = \frac{m_i \bb{r}_i}{\sum m_i} =  \frac{m_i \bb{r}_i}{M}$ is the center of mass.
The total linear momentum $\bb{P} = \sum m_i \frac{d\bb{r}_i}{dt} = M \frac{d\bb{R}}{dt}$.

Conservation theorem for the linear Momentum of a system of particles: 
if the total external force is zero, the total linear momentum is conserved.

Conservation Theorem for total angular momentum of a system of particles: $\bb{L}$ is constant in time 
if the applied external torque is zero. $\frac{d\bb{L}}{dt} = \bb{N}^{(e)}$

The total angular momentum of a system of particles is $\bb{L} = \bb{R} \times M \bb{v} + \sum_i \bb{r}_i' \times \bb{p}_i'$ 
where $\bb{r}_i' = \bb{R} - \bb{r}_i$ position relative to the center of mass.

\end{document}