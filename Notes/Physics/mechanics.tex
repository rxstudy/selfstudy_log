\documentclass[12pt]{article}

\include{pythonlisting}
\usepackage{fullpage}
\usepackage{times}
\usepackage[normalem]{ulem}
\usepackage{multirow}
\usepackage{fancyhdr,graphicx,amsmath,amssymb, mathtools, scrextend, titlesec, enumitem}\usepackage[pdftex]{hyperref}
\usepackage[ruled,vlined]{algorithm2e}
\usepackage{parskip}
\usepackage{listings}
\usepackage{amsmath}
\usepackage{physics}
\usepackage{bbm}
\usepackage{titling}
\usepackage{bm}
\usepackage{geometry}
\geometry{top=1cm, left=1cm, bottom=1.5cm, right=1.5cm, margin=1cm}
\newcommand{\varvec}[2][n]{#2_1,\ldots, #2_#1}
\newcommand{\Var}{\mathrm{Var}}
\newcommand{\Bias}{\mathrm{Bias}}
\newcommand{\Cov}{\mathrm{Cov}}
\newcommand{\E}{{\rm I\kern-.3em E}}
\newcommand{\Binomial}{\mathrm{Binomial}}
\newcommand{\Bernoulli}{\mathrm{Bernoulli}}
\newcommand{\Poisson}{\mathrm{Poisson}}
\newcommand{\Normal}{\mathcal{N}}
\newcommand{\one}{\mathbbm{1}}
\newcommand{\Beta}{\text{Beta}}
\newcommand{\BetaPDF}{\text{Beta}}
\newcommand{\GammaPDF}{\text{Gamma}}
\newcommand{\Uniform}{\mathrm{Uniform}}
\newcommand{\QED}{\newline \mbox{} \hfill $\blacksquare$}
\newcommand{\Real}{\mathbb{R}}
\newcommand{\Sgn}{\mbox{Sgn}}
\renewcommand*{\arraystretch}{1.4}
\newtheorem{theorem}{Result}
\newtheorem{definition}{Definition}
\lstset{frame=tb,
	language=Python,
	aboveskip=3mm,
	belowskip=3mm,
	showstringspaces=false,
	columns=flexible,
	basicstyle={\small\ttfamily},
	numbers=none,
	stringstyle=\color{mauve},
	breaklines=true,
	breakatwhitespace=true,
	tabsize=3
}

\title{Mechanics}
\begin{document}
	\maketitle
	
\section{Basic Elementary Principles}

\begin{theorem}(Newton's law)
	\begin{enumerate}
		\item  \textnormal{In an inertial reference frame, an object remains at rest or constant velocity unless acted upon by external force.}
		\item  $$\dot{\bb{p}} = \frac{d(m\bb{v})}{dt} = \bb{F}^{(e)}$$
		\item  \textnormal{Action and reaction: $\bb{F}_{ij} = -\bb{F}_{ji}$ (Additional condition for strong version: $\bb{r}_{ij} \times \bb{F}_{ji} = 0$)}
	\end{enumerate}
\end{theorem}
\subsection{Single particle}

 \begin{theorem}[Conservation of Linear Momentum]
 	$$\text{If } \bb{F} = 0, \text{ then } \bb{\dot p} = 0$$
 \end{theorem}

Angular momentum of the particle around point $O$ is $\bb{L} = \bb{r} \times \bb{p}$. Torque $\bb{N} = \bb{r} \times \bb{F}$. 

$$ \bb{N} = \frac{d}{dt} (\bb{r} \times m\bb{v}) = \frac{d \bb{L}}{dt} \equiv \bb{\dot L}$$

\begin{theorem}[Conservation of Angular Momentum]
	$$\textnormal{If $\bb{N} = 0$,  then $\bb{\dot L} = 0$} $$
\end{theorem}

If force field $\bb{F}$ is conservative ($\oint \bb{F} \cdot d \bb{s} = 0$ or $\int_a^b \bb{F} \cdot d\bb{s} = T_b - T_a$ is the same for any path between $a$ and $b$). 
There exists a potential scalar field $V$ such that $$\bb{F} = -\grad{V(\bb{r})}$$

\begin{theorem}[Conservation of Energy]
	 $$E = T + V \textnormal{ is conserved where } T \textnormal{ and } V \textnormal{ are kinetic energy and potential energy respectively}$$
\end{theorem}

\subsection{System of particles}
Center of mass: $$\bb{R} = \frac{\sum_i m_i \bb{r}_i}{\sum m_i} =  \frac{\sum_i m_i \bb{r}_i}{M}$$

Let $\bb{F}^{(e)}_i$ external force acting on particle $i$-th, and $\bb{F}_{ji}$ is the force 
exerted by $j$-th particle on $i$-th particle in the system, 
$$  m_i \frac{d^2(\bb{r_i})}{dt^2} = \bb{\dot{p_i}} = \sum_{j} \bb{F}_{ji} +  \bb{F}^{(e)}_i  $$
Summing over all particles, we get $$ \bb{\dot{p}} =  M \frac{d^2 \bb{R}}{dt^2} = \sum_i \bb{F}^{(e)}_i \equiv \bb{F}^{(e)}$$

\begin{theorem}[Conservation of Linear Momentum for a system of particles]
	$$\textnormal{if the total external force $\bb{F}^{(e)} = 0$, the total linear momentum $\bb{p}$ is conserved.}$$
\end{theorem}

The change of total angular momentum: $$\begin{aligned}
 \frac{d}{dt}\bb{L} &= \sum_i \bb{r_i}\times \bb{\dot{p}_i} = \sum_i \bb{r_i} \times (\bb{F}^{(e)}_i + \sum_j \bb{F}_{ji}) \\
  &= \bb{N}^{(e)}   + \sum_{i, j} \bb{r_i} \times \bb{F}_{ji} \\
  &= \bb{N}^{(e)}   + \sum_{i < j} \bb{r_i} \times \bb{F}_{ji} + \sum_{j < i} \bb{r_i} \times \bb{F}_{ji} \\
  &= \bb{N}^{(e)}   + \sum_{i < j} \bb{r_i} \times \bb{F}_{ji} + \sum_{i < j} \bb{r_j} \times \bb{F}_{ij} \\
  &= \bb{N}^{(e)}   + \sum_{i < j} \bb{r_i} \times \bb{F}_{ji} - \sum_{i < j} \bb{r_j} \times \bb{F}_{ji} \\
  &= \bb{N}^{(e)}  + \sum_{i < j} \left(\bb{r_i} - \bb{r_j} \right) \times \bb{F}_{ji} \\
  &= \bb{N}^{(e)}  + \sum_{i < j} \bb{r}_{ij} \times \bb{F}_{ji} \\
\end{aligned}$$

When the internal force abides strong version of Newton's 3rd law, internal force pair has the same direction as the two particles $\bb{r}_{ij}$. Then $\bb{r}_{ij}  \times \bb{F}_{ji} = 0$.  Forces satisfy strong Newton's 3rd law are \textbf{central}.
\begin{theorem}[Conservation of Angular Momentum for a system of particles]
   $$\textnormal{$\bb{L}$ is constant in time if applied external torque $\bb{N}^{(e)} = 0$ and internal forces are central}$$
\end{theorem}

Let $\bb{r}_i = \bb{R} + \bb{r}_i'$ where $\bb{r}_i'$ is the position of ith particle wrt to $\bb{R}$ instead of $O$. Similarly, $\bb{v}_i = \bb{v} + \bb{v}_i'$ where $\bb{v} = \dot{\bb{R}}$.  We can find $\bb{R} = \frac{\sum_i m_i \bb{r}_i}{M} = \frac{\sum_i m_i (\bb{R} + \bb{r}_i')}{M} = \bb{R} + \sum_i \bb{r}_i' $ which means $$\sum_i  \bb{r}_i' = 0, \quad\quad \sum_i  \bb{\dot{r}}_i' = \sum_i  \bb{v}_i' = 0 $$.

So the total angular momentum about $O$ is the angular momentum of center of mass + angular momentum of each particle around the center of mass. 
$$\begin{aligned}
	\bb{L} &=  \sum_i \bb{r_i}\times \bb{p_i} \\
	       &=\bb{R}\times M \bb{v} + \sum_i \bb{r}_i'\times \bb{p}_i' + \sum_i \bb{R}\times m_i \bb{v}_i' + \sum_i m_i \bb{r}_i' \times \bb{v} \\
	       &=    \bb{R}\times M \bb{v} + \sum_i \bb{r}_i'\times \bb{p}_i'
\end{aligned}
$$

In general $\bb{L}$ depends on choice of $O$ through $\bb{R}$ except when the center of mass is at rest $\bb{R} = 0$.

The total kinetic energy of the system is $$
T = \sum_i \frac{1}{2}m_i v_i^2 =  \sum_i \frac{1}{2} m_i (\bb{v} + \bb{v}_i') (\bb{v} + \bb{v}_i') = \frac{1}{2} M v^2 + \sum_i \frac{1}{2}m_i (v_i')^2 
$$

The ith particle has trajectory $\bb{r}_i(t)$ from time $a$ to $b$. Then the work is the change in kinetic energy.
$$W_i = \int_a^b \bb{F}_i  \cdot  d\bb{r_i} =  \int_a^b \frac{d(m_i v_i)}{dt}   \cdot  \frac{d \bb{r_i}}{dt} dt =  \int_a^b  m_i \bb{\dot{v}}_i  \cdot \bb{v}_i dt =  \int_a^b  d(\frac{1}{2} m_i v_i^2) = T_i(b) - T_i(a)$$

Also we can write 
$$
 W_i = \int_a^b \bb{F}_i  \cdot  d\bb{r_i}  = \int_a^b \bb{F}_i^{(e)}  \cdot  d\bb{r_i}  +  \sum_j \int_a^b \bb{F}_{ji} \cdot  d\bb{r_i}
$$


\end{document}