\documentclass[12pt]{article}

\include{pythonlisting}
\usepackage{fullpage}
\usepackage{times}
\usepackage[normalem]{ulem}
\usepackage{multirow}
\usepackage{fancyhdr,graphicx,amsmath,amssymb, mathtools, scrextend, titlesec, enumitem}\usepackage[pdftex]{hyperref}
\usepackage[ruled,vlined]{algorithm2e}
\usepackage{parskip}
\usepackage{listings}
\usepackage{amsmath}
\usepackage{physics}
\usepackage{bbm}
\usepackage{titling}
\usepackage{bm}
\usepackage{geometry}
\geometry{top=1cm, left=1cm, bottom=1.5cm, right=1.5cm, margin=1cm}
\newcommand{\varvec}[2][n]{#2_1,\ldots, #2_#1}
\newcommand{\Var}{\mathrm{Var}}
\newcommand{\Bias}{\mathrm{Bias}}
\newcommand{\Cov}{\mathrm{Cov}}
\newcommand{\E}{{\rm I\kern-.3em E}}
\newcommand{\Binomial}{\mathrm{Binomial}}
\newcommand{\Bernoulli}{\mathrm{Bernoulli}}
\newcommand{\Poisson}{\mathrm{Poisson}}
\newcommand{\Normal}{\mathcal{N}}
\newcommand{\one}{\mathbbm{1}}
\newcommand{\Beta}{\text{Beta}}
\newcommand{\BetaPDF}{\text{Beta}}
\newcommand{\GammaPDF}{\text{Gamma}}
\newcommand{\Uniform}{\mathrm{Uniform}}
\newcommand{\QED}{\newline \mbox{} \hfill $\blacksquare$}
\newcommand{\Real}{\mathbb{R}}
\newcommand{\Sgn}{\mbox{Sgn}}
\renewcommand*{\arraystretch}{1.4}
\newtheorem{theorem}{Result}
\newtheorem{definition}{Definition}
\lstset{frame=tb,
	language=Python,
	aboveskip=3mm,
	belowskip=3mm,
	showstringspaces=false,
	columns=flexible,
	basicstyle={\small\ttfamily},
	numbers=none,
	stringstyle=\color{mauve},
	breaklines=true,
	breakatwhitespace=true,
	tabsize=3
}

\title{Mechanics}
\begin{document}
	\maketitle
	
\section{Basic Elementary Principles}
\begin{theorem}(Newton's law)
	\begin{enumerate}
		\item  \textnormal{In an inertial reference frame, an object remains at rest or constant velocity unless acted upon by external force.}
		\item  $$\dot{\bb{p}} = \frac{d(m\bb{v})}{dt} = \bb{F}^{(e)}$$
		\item  \textnormal{Two particles exert forces on each other } $\bb{F}_{ij} = -\bb{F}_{ji}$
	\end{enumerate}
\end{theorem}
\subsection{Single particle}

 \begin{theorem}[Conversation Theorem for linear momentum]
 	$$\text{If } \bb{F} = 0, \text{ then } \bb{\dot p} = 0$$
 \end{theorem}

Angular momentum of the particle around point $O$ is $\bb{L} = \bb{r} \times \bb{p}$. Torque $\bb{N} = \bb{r} \times \bb{F}$. 

$$ \bb{N} = \frac{d}{dt} (\bb{r} \times m\bb{v}) = \frac{d \bb{L}}{dt} \equiv \bb{\dot L}$$

\begin{theorem}[Conservation Theorem for angular momentum]
	$$\textnormal{If } \bb{N} = 0,  \textnormal{then } \bb{\dot L} = 0$$
\end{theorem}

If force field $\bb{F}$ is conservative ($\oint \bb{F} \cdot d \bb{s} = 0$ or $\int_a^b \bb{F} \cdot d\bb{s} = T_b - T_a$ is the same for any path between $a$ and $b$). 
There exists a potential scalar field $V$ such that $$\bb{F} = -\grad{V(\bb{r})}$$

\begin{theorem}[Energy Conservation Theorem for a particle]
	 $$E = T + V \textnormal{ is conserved where } T \textnormal{ and } V \textnormal{ are kinetic energy and potential energy respectively}$$
\end{theorem}

\subsection{System of particles}
Center of mass: $$\bb{R} = \frac{\sum_i m_i \bb{r}_i}{\sum m_i} =  \frac{\sum_i m_i \bb{r}_i}{M}$$

Let $\bb{F}^{(e)}_i$ external force acting on particle $i$-th, and $\bb{F}_{ji}$ is the force 
exerted by $j$-th particle on $i$-th particle in the system, 
$$  m_i \frac{d^2(\bb{r_i})}{dt^2} = \bb{\dot{p_i}} = \sum_{j} \bb{F}_{ji} +  \bb{F}^{(e)}_i  $$
Summing over all particles, we get  \begin{align}
	& \frac{d^2}{dt^2} \sum_i m_i \bb{r}_i = \sum_i \bb{F}^{(e)}_i + \sum_{i \neq j} \bb{F}_{ji} \\
	\Rightarrow & M \frac{d^2 \bb{R}}{dt^2} = \sum_i \bb{F}^{(e)}_i \equiv \bb{F}^{(e)}
\end{align}
where $R = \frac{m_i \bb{r}_i}{\sum m_i} =  \frac{m_i \bb{r}_i}{M}$ is the center of mass.
The total linear momentum $\bb{P} = \sum m_i \frac{d\bb{r}_i}{dt} = M \frac{d\bb{R}}{dt}$.

Conservation theorem for the linear Momentum of a system of particles: 
if the total external force is zero, the total linear momentum is conserved.

Conservation Theorem for total angular momentum of a system of particles: $\bb{L}$ is constant in time 
if the applied external torque is zero. $\frac{d\bb{L}}{dt} = \bb{N}^{(e)}$

The total angular momentum of a system of particles is $\bb{L} = \bb{R} \times M \bb{v} + \sum_i \bb{r}_i' \times \bb{p}_i'$ 
where $\bb{r}_i' = \bb{R} - \bb{r}_i$ position relative to the center of mass.

\end{document}