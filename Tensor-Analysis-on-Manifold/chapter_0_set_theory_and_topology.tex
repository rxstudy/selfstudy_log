\documentclass[12pt]{article}

\include{pythonlisting}
\usepackage{fullpage}
\usepackage{times}
\usepackage[normalem]{ulem}
\usepackage{multirow}
\usepackage{fancyhdr,graphicx,amsmath,amssymb, mathtools, scrextend, titlesec, enumitem}\usepackage[pdftex]{hyperref}
\usepackage[ruled,vlined]{algorithm2e}
\usepackage{parskip}
\usepackage{listings}
\usepackage{amsmath}
\usepackage{physics}
\usepackage{bbm}
\usepackage{titling}
\usepackage{bm}
\usepackage{geometry}
\geometry{top=1cm, left=1cm, bottom=1.5cm, right=1.5cm, margin=1cm}
\newcommand{\varvec}[2][n]{#2_1,\ldots, #2_#1}
\newcommand{\Var}{\mathrm{Var}}
\newcommand{\Bias}{\mathrm{Bias}}
\newcommand{\Cov}{\mathrm{Cov}}
\newcommand{\E}{{\rm I\kern-.3em E}}
\newcommand{\Binomial}{\mathrm{Binomial}}
\newcommand{\Bernoulli}{\mathrm{Bernoulli}}
\newcommand{\Poisson}{\mathrm{Poisson}}
\newcommand{\Normal}{\mathcal{N}}
\newcommand{\one}{\mathbbm{1}}
\newcommand{\Beta}{\text{Beta}}
\newcommand{\BetaPDF}{\text{Beta}}
\newcommand{\GammaPDF}{\text{Gamma}}
\newcommand{\Uniform}{\mathrm{Uniform}}
\newcommand{\QED}{\newline \mbox{} \hfill $\blacksquare$}
\newcommand{\Real}{\mathbb{R}}
\newcommand{\Sgn}{\mbox{Sgn}}
\renewcommand*{\arraystretch}{1.4}
\newtheorem{theorem}{Result}
\newtheorem{definition}{Definition}
\lstset{frame=tb,
	language=Python,
	aboveskip=3mm,
	belowskip=3mm,
	showstringspaces=false,
	columns=flexible,
	basicstyle={\small\ttfamily},
	numbers=none,
	stringstyle=\color{mauve},
	breaklines=true,
	breakatwhitespace=true,
	tabsize=3
}

\title{Chapter 0: Set Theory and Topology}
\author{Ran Xie}
\begin{document}
\maketitle
	
\section*{Problem 0.1.2.1}
Since $A \triangle B = A \cup B - A \cap B$. Then 
$$ \begin{aligned}
	A \triangle B &= A \cup B - A \cap B \\
	&= (A \cup B) \cap (A \cap B)^c \\
	&= (A \cup B) \cap (A^c \cup B^c) \\
	&= (A \cap A^c) \cup (B \cap B^c) \cup (A \cap B^c) \cup (B \cap A^c) \\
	&= (A \cap B^c) \cup (B \cap A^c) \\
	&= (A - B) \cup (B - A)
\end{aligned}
$$

$$ \begin{aligned}
	A \cap C \triangle B \cap C &= (A \cap C - B \cap C) \cup (B \cap C - A \cap C) \\
	&= [(A\cap C) \cap (B^c \cup C^c)] \cup [ (B \cap C) \cup (A^c \cup C^c)] \\
	&= [A \cap C \cap B^c \cup A \cap C \cap C^c ] \cup [ B \cap C \cap A \cup B \cap C \cap C^c] \\
	&= [A \cap C \cap B^c \cup \emptyset ] \cup [ B \cap C \cap A \cup \emptyset] \\
	&=  A \cap B^c \cap C \cup  B \cap A^c \cap C  \\
	&= (A-B)\cap C  \cup (B - A) \cap C \\
	&=  [(A - B) \cup (B-A)] \cap C \\
	&= 	(A \triangle B) \cap C   
\end{aligned}
$$
\section*{Exercise 0.1.3.1}
$A \times B \neq B \times A$ Since Cartesian product is a set of ordered pair. 

\section*{Exercise 0.1.4.1}
Since $f : A \rightarrow B$ and There exists $g$ such that $f \circ g = i_B$.  Since the domain of $f \circ g$ is $B$. Then for each $y \in B$, $f \circ g(y) = i_B(y) = y$ which means there exists $x \in A$ such that $g(y) = x$ and $f(x) = y$. Therefore $f$ is onto.  \QED

 If there exists $y_1, y_2$ such that $g(y_1) = g(y_2)$. Then 
$$ \begin{aligned}
	f\circ g(y_1) = f \circ g(y_2) & \Leftrightarrow i_B(y_1) = i_B(y_2)  \\
	 &\Leftrightarrow	y_1 = y_2  \\
\end{aligned}
$$
Therefore $g$ is 1-1.  \QED

Let $h = f|_{gB}$, Since $f \circ g = i_B$, for each $y \in B$, $f \circ g (y) = i_B(y) = y$ which means there exists an $x \in g(B)$ such that $f(x) = y$. Therefore $h = f|_{gB}$ is onto. 

Note that $f \circ g$ can be written as $f|_{gB} \circ g = h \circ g = i_B$ since $f$ can only take on values in $g(B)$. $g$ is 1-1 means there is inverse $g^{-1}$ that is also 1-1. Hence $h = h \circ g \circ g^{-1} = i_B \circ g^{-1}$. Both $i_B$ and  $g^{-1}$ are 1-1, so $h$ is also 1-1.  \QED

Let $x \in g(B)$ and consider $g \circ h (x)$. There exists $y \in B$ such that $y = h(x)$. We know $h \circ g (y) = i_B(y) = y$. Suppose some $x_1 = g(y)$,  $h \circ g(y) = h(x_1) = y = h(x) \Rightarrow x_1 = x$ since $h$ is 1-1. So $g(y) = x$.  Therefore $g \circ h (x) = g (y) = x \Leftrightarrow g \circ h = i_{gB} \Leftrightarrow g = i_{gB} h^{-1}$  \QED

$f$ need not be 1-1. Example: $A = \{1, 2\}, B=\{3\}$. $f(1) = f(2) = 3$, $g(3) = 2$ and $h = f|_{g(B) =\{2\}}$.\QED

\section*{Exercise 0.1.4.2}
Suppose $f: A \rightarrow B$ is 1-1 and onto, then for each $y \in B$ there corresponds a unique $x \in A$ such that $f(x) = y$.  Define $g: B \rightarrow A$ such that for each $y \in B$, $g(y) = x$ where $f(x) =y$. $g$ is a function since each $y$ corresponds to a unique $x$ guaranteed by $f$. Therefore $g\circ f = i_A$ and $f \circ g = i_B$. \QED

Suppose There is a function $g : B \rightarrow A$ such that $g \circ f = i_A$ and $f \circ g = i_B$. For $x_1, x_2 \in A$ and $f(x_1) = f(x_2)$. Applying $g$ on both side, we have $x_1 = x_2$. Therefore $f$ is 1-1. 

For $y \in B$, there exists an $x \in A$ such that $g(y) = x$ since $g$ is a function. Applying $f$ to both side, we have $f(g(y)) = f(x) \Leftrightarrow i_B(y) = y = f(x)$. So we have found an $x$ for every $y$ such that $y = f(x)$.  Therefore $f$ is onto.  \QED

\section*{Exercise 0.1.5.1}
Suppose $f$ is onto, $B_1, B_2 \in P(B)$ and $f^{-1}(B_1) = f^{-1}(B_2)$. If $y \in B_1$, then there exists $x \in A$ such that $f(x) = y$ since $f$ is onto. By definition of complete inverse image map, $ x \in   f^{-1}(B_1) = f^{-1}(B_2)$ implies $ y = f(x) \in B_2$. The same argument applies to $B_2$. Then we have $B_1 = B_2$. Therefore $f^{-1}$ is 1-1. 

Suppose $f^{-1}$ is 1-1. For $\{y\} \in P(B)$, there exists a unique $\{x\} \in P(A)$ such that $f^{-1}(\{y\}) = \{x\}$. This implies for every $y \in B$ there exists $x$ such that $f(x) = y$.  \QED

\section*{Exercise 0.1.5.2}
(a) $$\begin{aligned}
	x \in f^{-1}(D_1 \cap D_2) &\Leftrightarrow \exists y \in D_1 \cap D_2, f(x) = y  \\
	&\Leftrightarrow x \in (f^{-1}D_1) \cap (f^{-1}D_2)	    
\end{aligned}
$$

(b) $$
\begin{aligned}
	x \in f^{-1}(D_1 \cup D_2) &\Leftrightarrow \exists y \in D_1 \cup D_2, f(x) = y  \\
	&\Leftrightarrow x \in (f^{-1}D_1) \mbox{ if } y \in D_1, x \in (f^{-1}D_2) \mbox{ if } y \in D_2 \\
	&\Leftrightarrow x \in (f^{-1}D_1) \cup (f^{-1}D_2)  \\
\end{aligned}
$$

(c) $$\begin{aligned}
	y \in f(C_1 \cap C_2) &\Rightarrow \exists x \in C_1 \cap C_2, f(x) = y \\
	&\Rightarrow y \in (fC_1) \cap (fC_2)
\end{aligned}
$$

(d) $$\begin{aligned}
	y \in f(C_1 \cup C_2) &\Leftrightarrow \exists x \in C_1 \cup C_2, f(x) = y \\
	&\Leftrightarrow y \in fC_1 \mbox{ if } x \in C_1 ,  y \in fC_2 \mbox{ if } x \in C_2 \\
	&\Leftrightarrow y \in (fC_1) \cup (fC_2)
\end{aligned}
$$

\section*{Exercise 0.1.5.3}
Let $A = \{1, 2\}$, $B =\{3\}$,  $f(1) = f(2) = 3$. If $C_1 = \{1\}, C_2=\{2\}$. Then $fC_1 \cap fC_2 = \{3\} \neq f(C_1 \cap C_2) = f (\emptyset)$

\section*{Exercise 0.1.5.4}
For $B, C \in P(A)$, $\Phi C = \Phi B \Rightarrow \phi_C = \phi_B$. If $\phi_C (x) = 1$,  then $\phi_B(x) = 1$ which means $x \in C$ implies $x \in B$ and vice versa. By the same argument on $\phi_C(x) = 0$, we have $B = C$. So $\Phi$ is 1-1.

For a characteristic function $\phi_D \in 2^A$. By definition $D \subset A \Rightarrow D \in P(A)$. So $\Phi$ is onto. 


\section*{Exercise 0.1.5.5}
If $A$ is finite ($A =\{a_1, \ldots, a_n \}$), there exists a bijection between $P(A)$ and $\{(b_1, \ldots, b_n) | b_i \in \{0, 1\} \}$ where $b_i$ is 0 if $a_i$ is absent in the subset, 1 if $a_i$ is present. We have two choice for each $i$ and there are $n$ of them. So $|P(A)| = 2^n$. From exercise 0.1.5.4, $|2^A| = |P(A)| = 2^n$.

\section*{Exercise 0.1.5.6}
$F: A \rightarrow 2^A$. For each $a \in A$, $(Fa)(a)$ is either 1 or 0. We can define $f \in 2^A$ such that $fa = (1 - (Fa)) a$ for all $a$. Then $fa \neq (Fa)a$ for all $a \in A$.

Now we can show $f$ is not in the range of $F$. Suppose there exists $b \in A$ such that $Fb = f$. But then $(Fb) b = 1 - fb = fb \Leftrightarrow 1 = 0$ which is a contradiction.


\section*{Exercise 0.2.1.1}
Let $X = \{a, b\}$, we have $T=\{\{a\}, \emptyset, X\}$ and $T=\{\{b\}, \emptyset, X\}$ with concrete and discrete topologies. Therefore 4 distinct topologies. \QED

Let $X = \{a, b, c\}$, 

For 2 elements topolgy, we have the concrete topology $\{X, \emptyset\}$. Total of 1.

For 3 elements topology, we have $T=\{\{a\}, \emptyset, X\}$ (3 of this kind).  $T=\{\{a, b\}, \emptyset, X\}$ (3 of this kind). Total of 6. 

For 4 elements topology, we have $T=\{\{a, b\}, \{a\}, \emptyset, X\}$ ($3 \times 2 = 6$ of this kind). $T=\{\{a, b\}, \{c\}, \emptyset, X\}$ (3 of this kind). Total of 9.

For 5 elements topology, $T=\{\{a, b\}, \{a, c\}, \{a\}, \emptyset, X\}$ (3 of this kind). Total of 3. $T=\{\{a, b\}, \{a\}, \{b\}, \emptyset, X\}$ (3 of this kind). Total of 6.

For 6 elements topology,  $T=\{\{a, b\}, \{a, c\}, \{a\}, \{b\}, \emptyset, X\}$ ($3 \times 2=6$ of this kind). Total of 6

For 8 elements topology, there's only 1 which is $P(X)$. 

Therefore $X$ has 1 + 6 + 9 + 6 + 6 + 1 = 29 distinct topologies.\QED

\section*{Exercise 0.2.1.3}
(a) If $x \in A^- \cup B^-$, then $x$ is in the all closed set that contain $A$ or all closed sets that contain $B$ which implies $x$ is in all close sets that 
contain $A \cup B$ since any close sets that contains $A \cup B$ contain $A$ and $B$. Therefore $x \in (A \cup B)^-$.

Let $x \in (A \cup B)^-$.  Suppose $x \notin A^-$ and $x\notin B^-$, then there exists closed set $D_A$ and $D_B$ containing $A$ and $B$ respectively such that $x \notin D_A$ and $x \notin D_B$. But $A \cup B \subset D_A \cup D_B$ and finite union of closed set is also closed. So $D_A \cup D_B$ is a closed set that covers $A\cup B$. So $x$ must be in $D_A \cup D_B$. We have reached contradiction that $x \in D_A$ or $x \in D_B$. 

Therefore $(A \cup B)^- = A^- \cup B^-$. \QED

(b) By definition of closure. \QED

(c) $A^-$ is closed since arbitrary intersection of closed set is closed. 
 $A^-$ is the smallest closed set that contains itself. So $A^- = (A^-)^-$ \QED
 
(d) $X = X - \emptyset$ is open, therefore $\emptyset$ is closed. By the same argument in (c). $\emptyset^- = \emptyset$. \QED

\section*{Exercise 0.2.1.2}
We state the dual proposition for interior operator $^0$. 

(a) $(A \cap B)^0 = A^0 \cap B^0$

Proof: If $x \in A^0 \cap B^0$,  then $x \in O_A \cap O_B$ for some open set $O_A \subset A$ and $O_B \subset B$. Finite intersect of open sets is open and $O_A \cap O_B \subset A \cap B$. Therefore $x \in (A \cap B)^0$.

If $x \in (A \cap B)^0$, then $x \in O$ for some open set $O \subset A \cap B$. Note that $O \subset A$ and $O \subset B$. Therefore $x \in A^0$ and $x \in B^0$. So $x \in A^0 \cap B^0$. \QED

(b) $A^0 \subset A$

Proof: Follow by definition of interior. \QED

(c) $(A^0)^0 = A^0$

Proof: The interior of $A^0$ is the largest open set that is contained in $A^0$ which is itself. \QED

(d) $\emptyset^0 = \emptyset$

Proof: $\emptyset$ is open set and by argument in (c). It follows. \QED

\section*{Exercise 0.2.4.1}
(a) The metric $d_p(x,y) = \left( \sum_i |u^i x - u^i y|^p \right)^{1/p}$ for fixed $x$ and $y$ has the form of $d(p) = f(p)^{g(p)}$ where $f(p) = \sum_i (c_i)^p,  c_i =  |u^i x - u^i y| \geq 0$ and $g(p) = \frac{1}{p}$. 
$$
\begin{aligned}
	d(p) &= f(p)^{g(p)} \\
	\ln(d(p)) &= g(p) \ln f(p) \\
	\frac{d'}{d} &= g' \ln f + \frac{gf'}{f} \\
	d' &= d \left[g' \ln f + \frac{gf'}{f} \right] \\
	d' &= d  \left[-\frac{1}{p^2} \ln \left(\sum_i (c_i)^p \right) + \frac{\sum_i (c_i)^p \ln c_i}{p \sum_i (c_i)^p} \right]  \\
	&= \frac{d}{p}  \left[\frac{\sum_i (c_i)^p \ln c_i}{\sum_i (c_i)^p} -\frac{1}{p} \ln \left(\sum_i (c_i)^p \right)  \right] \\
	&\leq \frac{d}{p}  \left[\ln c_{\max}-\frac{1}{p} \ln \left(\sum_i (c_i)^p \right)  \right] \\
	&= \frac{d}{p^2}  \left[\ln c_{\max}^p-\ln \left(\sum_i (c_i)^p \right)  \right] \\
	&= - \frac{d}{p^2} \left[\ln\left(\sum_i (c_i)^p\right) - \ln c_{\max}^p \right] \leq 0  \\
\end{aligned}
$$
Therefore $d_p(x,y)$ is non-increasing. \QED

(b) $d_1(x,y) = \sum_i c_i$ and $d_{\infty} (x,y) = c_{\max}$. Therefore $$d_1(x,y) = \sum_i c_i \leq \sum_i c_{\max} = n c_{\max} = n d_{\infty} (x,y) $$ 
\QED

Since we have $$ \begin{aligned}
	d_p(x,y) &= (\sum_i c_i^p)^{1/p} \\
			& \leq (n c_{\max}^p )^{1/p} \\
	\ln d_p(x,y) &= \frac{1}{p} \ln (n c_{\max}^p) = \frac{\ln n}{p} + \ln c_{\max} \\
	\lim_{p \rightarrow \infty}	\ln d_p(x,y) &= \lim_{p \rightarrow \infty} \frac{\ln n}{p} + \ln c_{\max}  \\
	  &= \ln c_{\max}
\end{aligned}
$$

Therefore $d_{\infty} (x,y) = \lim_{p \rightarrow \infty} d_p(x,y) = c_{\max}$ \QED

(c) First we show $d_t$ and $d_{\infty}$ are strongly equivalent. By (b), we have shown one direction. Next we shown there exists $k$ such that $k d_{\infty} \leq d_t$.

Consider $\frac{d_t^t}{d_{\infty}^t}$.
 $$\frac{d_t^t}{d_{\infty}^t} = \frac{\sum c_i^t}{\sum c_{\max}^t} = \sum^n \left( \frac{c_i}{c_{\max}} \right)^t \geq 1$$ Notice that every term is greater or equal to $0$ and there would be a term that is equal to 1 ($c_i = c_{\max})$. So the sum is greater or equal to 1 which means $k = 1$. (We can get this conclusion from (a) as well.) 
 $$ d_{\infty} \leq d_s \leq n d_{\infty}$$
\QED

Now for any $s, t$, $d_s, d_t$ are strongly equivalent. 
By definition, $d_s^s = \sum c_i^s$ and $d_t^t = \sum c_i^t$

Let $t = s + \delta$ and $\delta > 0$. So $t > s$.
$$ 
\begin{aligned}
d_t^t = d_t^{s + \delta} &= \sum c_i^{s+ \delta}  \\
d_t^s &= \frac{ \sum c_i^{s+ \delta}}{d_t^{\delta}} \\
& \geq \frac{c_{\max}^\delta \sum c_i^{s}}{d_t^{\delta}} \\ 
& = \frac{d_{\infty}^\delta \sum c_i^{s}}{d_t^{\delta}} \\
& \geq \frac{d_{\infty}^\delta \sum c_i^{s}}{(n d_{\infty})^{\delta}}, \mbox{ By (b) } \\
& = \frac{d_s^s}{n^{\delta}} \\
\mbox{ Therefore, } n^{\delta/s} d_t &\geq d_s
\end{aligned}$$

Furthermore, by (a), the metric is non-increase. We have 
$$ d_t \leq d_s \leq n^{t/s -1} d_t$$
So they are strongly equivalent. \QED

\section*{Exercise 0.2.5.1}
Suppose $(X, T_X), (Y, T_Y)$ are Hausdorff and $(x,y) \in X \times Y$. There exists neighorhoods $U_x, V_x \in T_X$ such that $U_x \cap V_x = \emptyset$ and $U_y, V_y \in T_Y$ such that $U_y \cap V_y = \emptyset$. By definition, $U_x \times U_y$ and $V_x \times V_y$ are neighorhood for $(x,y)$. Since $U_x \times U_y \cap V_x \times V_y = \emptyset$ by definition of product. So $X \times Y$ is Hausdorff. \QED

\section*{Exercise 0.2.6.1}
Let $f: X \rightarrow X$ be any function map from $(X, 2^X)$ to $(X, T)$. For any $Y \in T$, $f^{-1}(Y) \in 2^X$ by definition of $2^X$. $f^{-1}$ maps open sets to open sets hence continueous. \QED

Let $f$ be a continuous function from $(X, \{\emptyset, X\})$ to  $(X, 2^X)$ and $f$ is not constant. There exists $x_1, x_2 \in X$ and $y_1=fx_1$, $y_2=fx_2$ such that $x_1 \neq x_2$ and $y_1 \neq y_2$. Since $f$ is continous and point set are open in $2^X$, $f^{-1}(\{y_1\}) = X = f^{-1}(\{y_2\})$. This means all $x \in X$ maps to $y_1$ or $y_2$. If $y_1 \neq y_2$ then $f$ is not a function. Therefore $y_1 = y_2$ contradicts our assume. So $f$ is constant. \QED

\section*{Exercise 0.2.6.2}
$\tan$ and $\arctan$ are differentiable hence continous. They are also 1-1 and onto. Hence $\tan$ is homeomorphism.

\section*{Exercise 0.2.7.1}
Let $f: X \rightarrow Y$ be homeomorphism and suppose $A \subset X$ is connected. If $f(A)$ is not connected, then there exists disjoint open sets $D_1 \cap f(A), D_2 \cap f(A)$ in relative topology of $f(A)$ such that their union is $f(A)$. Then $$ \begin{aligned}
	f^{-1}(D_1 \cap f(A) \cup D_2 \cap f(A)) &= f^{-1}(D_1 \cap f(A)) \cup f^{-1}(D_2 \cap f(A)) \\
	&= f^{-1}(D_1)\cap f^{-1}(A) \cup f^{-1}(D_2)\cap f^{-1}(A) \\
	&= f^{-1}(A) \cap \left[ f^{-1}(D_1)\cup f^{-1}(D_2) \right] \\
	&= A
\end{aligned}$$. Since $f$ is homeomorphism, $ f^{-1}(D_1 \cap f(A)) = f^{-1}(D_1)\cap A$ and $f^{-1}(D_2) \cap A$ are two disjoint open sets in relative topology of $A$ and their union is $A$. This contradicts with the assumption that $A$ is connected. \QED

\section*{Exercise 0.2.7.2}
(a) Let $B = \{ x | x \in A, x \mbox{ polygonal connected to } a \in A \}$. Obviously $B \subset A$. To prove $B$ is open, we show that $B^0 = B$. Suppose $y \in B$ and $y \notin B^0$. Then $y$ isn't in any open set contained in $B$.There exists an open ball $U \subset A$ containing $y$ such that $U-B \neq \emptyset$. But $z \in U - B \subset A$ is polygonal connected to $y$ and hence to $a$. So the points in $U-B$ are in $B$. By contradiction, $y \in B $ implies $y \in B^0$. Hence $B^0 = B$ means $B$ is open. \QED

(b) Suppose $A$ is not polygonally connected. Then for any $x \in A$ define $U_x$ to be the set of points in $A$ that is polygonal connected to $x$ and $V_x$ to be the set of points in $A$ that is not polygonal connected to $x$. By our assumption, $V_x \neq \emptyset$ and $U_x \neq \emptyset$. Furthermore $V_x \cup U_x = A$ and  $V_x \cap U_x = \emptyset$. 

Note that $A$ is open by definition of relative topology. Then we have $U_x$ and $V_x$ being open as well from the result of (a). Therefore $A$ is not connected. By contrapositive, We have proven $A$ connected $\Rightarrow$ $A$ is polygonal connected. \QED

\section*{Exercise 0.2.8.1}
Suppose $f: X \rightarrow Y$ is homeomorphism and $A \subset X$ is compact. For any open covering $C_A$ of $A$. There exists a finite subcover $C'_A$ of $A$. Define $D_A = \{ f(C) | C \in C_A \}$. Then $D_A$ is an open covering for $f(A)$ since $f$ maps open set to open set. It follows that $D'_A = \{f(C') | C' \in C'_A \}$ is a finite subcovering for $f(A)$. Hence $f(A)$ is compact. Compactness is preserved.
\QED 

\section*{Exercise 0.2.9.1}
(a) Let $(X, T_X)$ be locally compact, and $(Y, Y\cap T_X)$ be a subspace where $Y$ is closed. For any $x \in Y$, there exists compact neighborhood $U \subset X$ of $x$.
We want to show $U \cap Y$ is the compact neighborhood for $x$ in $Y$.

Suppose $C$ is an open covering for $U \cap Y$,  then $X - Y$ is open and covers the portion $U - Y$. Therefore $C \cup \{X - Y\}$ is an open covering for $U$. Given that $U$ is compact, there is a finite subcovering $D \subset C \cup \{X - Y\}$. And $D$ works as a finite subcovering for $U \cap Y$ since $X - Y$ doesn't need to be in $D$ (it doesn't cover the portion $Y \cap U$). So $Y \cap U$ is compact. $Y$ is locally compact.  \QED 

(b) In a discrete space $X$, for any $x \in X$ and any open covering of $\{x\}$, there exists an open set $A$ in the covering such that $\{x\} \subset A$. Therefore $A$ is the finite open subcovering. So $\{x\}$ is compact neighborhood of $x$. So $X$ is locally compact. \QED 

(c) For any $x \in \Real^n$, there exists a closed ball centered at $x$. Closed ball are compact in $\Real^n$. Therefore $\Real^n$ is locally compact. \QED 

\section*{Exercise 0.2.10.1}
(a) Let $A$ be countable subset of metric space $X$ such that $A^- = X$. For $x \in X$ and a neighborhood $U_x$ containing $x$. We want to show $A \cap U_x \neq \emptyset$.

Suppose $A \cap U_x = \emptyset$, then $U_x \subset X - A = A^- - A \subset \partial A$. This implies $x \in \partial A$. Since $U_x$ is neighborhood of $x$, and by definition of boundary points, $U_x \cap A \neq \emptyset$. By contradiction, there exists some $a$ from $A$ such that $a \in U_x$.

Next we can construct an open ball $B_r(a) \subset U_x$ where $r \in \Real$ and $r > 0$. For such $r$, there exists rational number $r_0$ such that $0 < r_0 < r$. Then we have constructed an open ball with rational radii inside any neighborhood of $x \in X$ which forms a basis of neighorhoods for $X$. Hence $X$ is separable. \QED

(b) Let $A$ be the set of rational in $\Real$. Then $A^- = \Real$. By (a), $\Real$ is separable. And by Exercise 0.2.10.2 (below), product of separable spaces is separable, it follows that $\Real^n$ is separable. \QED

\section*{Exercise 0.2.10.2}
Let $X, Y$ be separable and $A_X, A_Y$ be the basis respectively. We want to show that $A_X \times A_Y$ forms a basis for $X \times Y$.

First $A_X \times A_Y$ is countable. Second, For any point $(x,y) \in X\times Y$ and any open neighbordhood $U$ containing $(x, y)$. $U$ can be written as $U_X \times U_Y$ by definition of product space. Then $U_X$, $U_Y$ are neighorhood that contains $x, y$ respectively. Since $A_X, A_Y$ are basis, there exist $V_X \in A_X, V_Y \in A_Y$ such that $x \in V_X \subset U_X$ and $y \in V_Y \subset U_Y$. Then $V_X \times V_Y \in A_X \times A_Y$ and $V_X \times V_Y \subset U$. Therefore $A_X \times A_Y$ forms a basis for $X \times Y$. $X \times Y$ is separable. \QED

\section*{Exercise 0.2.11.1}
%TODO

\section*{Exercise 0.2.11.2}
%TODO

\end{document}