\documentclass[12pt]{article}

\include{pythonlisting}
\usepackage{fullpage}
\usepackage{times}
\usepackage[normalem]{ulem}
\usepackage{multirow}
\usepackage{fancyhdr,graphicx,amsmath,amssymb, mathtools, scrextend, titlesec, enumitem}\usepackage[pdftex]{hyperref}
\usepackage[ruled,vlined]{algorithm2e}
\usepackage{parskip}
\usepackage{listings}
\usepackage{amsmath}
\usepackage{physics}
\usepackage{bbm}
\usepackage{titling}
\usepackage{bm}
\usepackage{geometry}
\geometry{top=1cm, left=1cm, bottom=1.5cm, right=1.5cm, margin=1cm}
\newcommand{\varvec}[2][n]{#2_1,\ldots, #2_#1}
\newcommand{\Var}{\mathrm{Var}}
\newcommand{\Bias}{\mathrm{Bias}}
\newcommand{\Cov}{\mathrm{Cov}}
\newcommand{\E}{{\rm I\kern-.3em E}}
\newcommand{\Binomial}{\mathrm{Binomial}}
\newcommand{\Bernoulli}{\mathrm{Bernoulli}}
\newcommand{\Poisson}{\mathrm{Poisson}}
\newcommand{\Normal}{\mathcal{N}}
\newcommand{\one}{\mathbbm{1}}
\newcommand{\Beta}{\text{Beta}}
\newcommand{\BetaPDF}{\text{Beta}}
\newcommand{\GammaPDF}{\text{Gamma}}
\newcommand{\Uniform}{\mathrm{Uniform}}
\newcommand{\QED}{\newline \mbox{} \hfill $\blacksquare$}
\newcommand{\Real}{\mathbb{R}}
\newcommand{\Sgn}{\mbox{Sgn}}
\renewcommand*{\arraystretch}{1.4}
\newtheorem{theorem}{Result}
\newtheorem{definition}{Definition}
\lstset{frame=tb,
	language=Python,
	aboveskip=3mm,
	belowskip=3mm,
	showstringspaces=false,
	columns=flexible,
	basicstyle={\small\ttfamily},
	numbers=none,
	stringstyle=\color{mauve},
	breaklines=true,
	breakatwhitespace=true,
	tabsize=3
}

\title{Chapter 0: Set Theory and Topology}
\author{Ran Xie}
\begin{document}
\maketitle
	
\section*{Problem 0.1.2.1}
Since $A \triangle B = A \cup B - A \cap B$. Then 
$$ \begin{aligned}
	A \triangle B &= A \cup B - A \cap B \\
	&= (A \cup B) \cap (A \cap B)^c \\
	&= (A \cup B) \cap (A^c \cup B^c) \\
	&= (A \cap A^c) \cup (B \cap B^c) \cup (A \cap B^c) \cup (B \cap A^c) \\
	&= (A \cap B^c) \cup (B \cap A^c) \\
	&= (A - B) \cup (B - A)
\end{aligned}
$$

$$ \begin{aligned}
	A \cap C \triangle B \cap C &= (A \cap C - B \cap C) \cup (B \cap C - A \cap C) \\
	&= [(A\cap C) \cap (B^c \cup C^c)] \cup [ (B \cap C) \cup (A^c \cup C^c)] \\
	&= [A \cap C \cap B^c \cup A \cap C \cap C^c ] \cup [ B \cap C \cap A \cup B \cap C \cap C^c] \\
	&= [A \cap C \cap B^c \cup \emptyset ] \cup [ B \cap C \cap A \cup \emptyset] \\
	&=  A \cap B^c \cap C \cup  B \cap A^c \cap C  \\
	&= (A-B)\cap C  \cup (B - A) \cap C \\
	&=  [(A - B) \cup (B-A)] \cap C \\
	&= 	(A \triangle B) \cap C   
\end{aligned}
$$
\section*{Exercise 0.1.3.1}
$A \times B \neq B \times A$ Since Cartesian product is a set of ordered pair. 

\section*{Exercise 0.1.4.1}
Since $f : A \rightarrow B$ and There exists $g$ such that $f \circ g = i_B$.  Since the domain of $f \circ g$ is $B$. Then for each $y \in B$, $f \circ g(y) = i_B(y) = y$ which means there exists $x \in A$ such that $g(y) = x$ and $f(x) = y$. Therefore $f$ is onto.  $\blacksquare$

 If there exists $y_1, y_2$ such that $g(y_1) = g(y_2)$. Then 
$$ \begin{aligned}
	f\circ g(y_1) = f \circ g(y_2) & \Leftrightarrow i_B(y_1) = i_B(y_2)  \\
	 &\Leftrightarrow	y_1 = y_2  \\
\end{aligned}
$$
Therefore $g$ is 1-1.  $\blacksquare$

Let $h = f|_{gB}$, Since $f \circ g = i_B$, for each $y \in B$, $f \circ g (y) = i_B(y) = y$ which means there exists an $x \in g(B)$ such that $f(x) = y$. Therefore $h = f|_{gB}$ is onto. 

Note that $f \circ g$ can be written as $f|_{gB} \circ g = h \circ g = i_B$ since $f$ can only take on values in $g(B)$. $g$ is 1-1 means there is inverse $g^{-1}$ that is also 1-1. Hence $h = h \circ g \circ g^{-1} = i_B \circ g^{-1}$. Both $i_B$ and  $g^{-1}$ are 1-1, so $h$ is also 1-1.  $\blacksquare$

Let $x \in g(B)$ and consider $g \circ h (x)$. There exists $y \in B$ such that $y = h(x)$. We know $h \circ g (y) = i_B(y) = y$. Suppose some $x_1 = g(y)$,  $h \circ g(y) = h(x_1) = y = h(x) \Rightarrow x_1 = x$ since $h$ is 1-1. So $g(y) = x$.  Therefore $g \circ h (x) = g (y) = x \Leftrightarrow g \circ h = i_{gB} \Leftrightarrow g = i_{gB} h^{-1}$  $\blacksquare$

$f$ need not be 1-1. Example: $A = \{1, 2\}, B=\{3\}$. $f(1) = f(2) = 3$, $g(3) = 2$ and $h = f|_{g(B) =\{2\}}$. $\blacksquare$

\section*{Exercise 0.1.4.2}
Suppose $f: A \rightarrow B$ is 1-1 and onto, then for each $y \in B$ there corresponds a unique $x \in A$ such that $f(x) = y$.  Define $g: B \rightarrow A$ such that for each $y \in B$, $g(y) = x$ where $f(x) =y$. $g$ is a function since each $y$ corresponds to a unique $x$ guaranteed by $f$. Therefore $g\circ f = i_A$ and $f \circ g = i_B$. $\blacksquare$

Suppose There is a function $g : B \rightarrow A$ such that $g \circ f = i_A$ and $f \circ g = i_B$. For $x_1, x_2 \in A$ and $f(x_1) = f(x_2)$. Applying $g$ on both side, we have $x_1 = x_2$. Therefore $f$ is 1-1. 

For $y \in B$, there exists an $x \in A$ such that $g(y) = x$ since $g$ is a function. Applying $f$ to both side, we have $f(g(y)) = f(x) \Leftrightarrow i_B(y) = y = f(x)$. So we have found an $x$ for every $y$ such that $y = f(x)$.  Therefore $f$ is onto.  $\blacksquare$


\end{document}