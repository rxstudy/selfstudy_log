\documentclass[12pt]{article}

\usepackage{fullpage}
\usepackage{times}
\usepackage[normalem]{ulem}
\usepackage{multirow}
\usepackage{fancyhdr,graphicx,amsmath,amssymb, mathtools, scrextend, titlesec, enumitem}\usepackage[pdftex]{hyperref}
\usepackage[ruled,vlined]{algorithm2e}
\usepackage{parskip}
\usepackage{listings}
\usepackage{amsmath}
\usepackage{physics}
\usepackage{bbm}
\usepackage{titling}
\usepackage{bm}
\usepackage{geometry}
\geometry{top=1cm, left=1cm, bottom=1.5cm, right=1.5cm, margin=1cm}
\newcommand{\varvec}[2][n]{#2_1,\ldots, #2_#1}
\newcommand{\Var}{\mathrm{Var}}
\newcommand{\Bias}{\mathrm{Bias}}
\newcommand{\Cov}{\mathrm{Cov}}
\newcommand{\E}{{\rm I\kern-.3em E}}
\newcommand{\Binomial}{\mathrm{Binomial}}
\newcommand{\Bernoulli}{\mathrm{Bernoulli}}
\newcommand{\Poisson}{\mathrm{Poisson}}
\newcommand{\Normal}{\mathcal{N}}
\newcommand{\one}{\mathbbm{1}}
\newcommand{\Beta}{\text{Beta}}
\newcommand{\BetaPDF}{\text{Beta}}
\newcommand{\GammaPDF}{\text{Gamma}}
\newcommand{\Uniform}{\mathrm{Uniform}}
\newcommand{\QED}{\newline \mbox{} \hfill $\blacksquare$}
\newcommand{\Real}{\mathbb{R}}
\newcommand{\Sgn}{\mbox{Sgn}}
\renewcommand*{\arraystretch}{1.4}
\newtheorem{theorem}{Result}
\newtheorem{definition}{Definition}
\lstset{frame=tb,
	language=Python,
	aboveskip=3mm,
	belowskip=3mm,
	showstringspaces=false,
	columns=flexible,
	basicstyle={\small\ttfamily},
	numbers=none,
	stringstyle=\color{mauve},
	breaklines=true,
	breakatwhitespace=true,
	tabsize=3
}

\title{Chapter 7: Point Estimation}
\begin{document}
	\maketitle
	
\section*{Exercise 7.1}
 If we take the product of $f(x|\theta)$ as the likelihood function, all 3 values of $\theta$ attain maximum likelihood of 0. 
 
 \subsubsection*{Method 1: With perturbation on likelihood function}
 We will perturb the likelihood function by $\epsilon$. Now even the probability 0 entries get a probability of $\epsilon$. Then 
 \begin{equation*}
 	\begin{split}
 		L(\theta=1 | x + \epsilon) &= \frac{\epsilon (1/3 + \epsilon )^2 (1/6 + \epsilon)^2}{(1 + 5\epsilon)^5} \\
 		L(\theta=2 | x + \epsilon) &= \frac{\epsilon (1/4 + \epsilon )^4}{(1 + 5\epsilon)^5} \\
 		L(\theta=3 | x + \epsilon) &= \frac{\epsilon^2 (1/4 + \epsilon )^2 (1/2 + \epsilon)}{(1 + 5\epsilon)^5} \\
 	\end{split}
 \end{equation*}
 We can compare the functions by taking the ratio and letting $\epsilon$ go to 0.  
 $$
 \frac{ L(\theta = 2 | x ) }{ L(\theta = 1 | x) } = \lim_{\epsilon \rightarrow 0} \frac{ L(\theta = 2 | x + \epsilon) }{ L(\theta = 1 | x + \epsilon) } = \lim_{\epsilon \rightarrow 0} \frac{(1/4 + \epsilon )^4}{(1/3 + \epsilon )^2 (1/6 + \epsilon)^2} = 1.26
 $$

 Similarly, $$
 \frac{L(\theta = 2| x)} {L(\theta = 3 | x)} = \lim_{\epsilon \rightarrow 0} \frac{(1/4 + \epsilon )^4}{\epsilon (1/4 + \epsilon )^2 (1/2 + \epsilon)} = \infty$$
 
Therefore $\theta=2$ is the MLE

 \subsubsection*{Method 2: With +1 smoothing}
 we can assume $N$ trials are performed for each $\theta$ and apply laplace smoothing (+1 smoothing). 
 
 Then the likelihood functions become 
 \begin{equation*}
 	\begin{split}
     L(\theta=1|x) &= \lim_{N \rightarrow \infty}\frac{(N/3 + 1)^2(N/6 + 1)^2}{(N + 5)^5} \\
 	L(\theta=2|x) &= \lim_{N \rightarrow \infty}\frac{(N/4 + 1)^4}{(N + 5)^5} \\
 	L(\theta=3|x) &= \lim_{N \rightarrow \infty}\frac{(N/4 + 1/ N)^2(N/2 + 1)}{(N + 5)^5} \\
 	\end{split}
 \end{equation*}
This is essentially the same as method 1.

 
\end{document}
