\documentclass[12pt]{article}

\include{pythonlisting}
\usepackage{fullpage}
\usepackage{times}
\usepackage[normalem]{ulem}
\usepackage{multirow}
\usepackage{fancyhdr,graphicx,amsmath,amssymb, mathtools, scrextend, titlesec, enumitem}\usepackage[pdftex]{hyperref}
\usepackage[ruled,vlined]{algorithm2e}
\usepackage{parskip}
\usepackage{listings}
\usepackage{amsmath}
\usepackage{physics}
\usepackage{bbm}
\usepackage{titling}
\usepackage{bm}
\usepackage{geometry}
\geometry{top=1cm, left=1cm, bottom=1.5cm, right=1.5cm, margin=1cm}
\newcommand{\varvec}[2][n]{#2_1,\ldots, #2_#1}
\newcommand{\Var}{\mathrm{Var}}
\newcommand{\Bias}{\mathrm{Bias}}
\newcommand{\Cov}{\mathrm{Cov}}
\newcommand{\E}{{\rm I\kern-.3em E}}
\newcommand{\Binomial}{\mathrm{Binomial}}
\newcommand{\Bernoulli}{\mathrm{Bernoulli}}
\newcommand{\Poisson}{\mathrm{Poisson}}
\newcommand{\Normal}{\mathcal{N}}
\newcommand{\one}{\mathbbm{1}}
\newcommand{\Beta}{\text{Beta}}
\newcommand{\BetaPDF}{\text{Beta}}
\newcommand{\GammaPDF}{\text{Gamma}}
\newcommand{\Uniform}{\mathrm{Uniform}}
\newcommand{\QED}{\newline \mbox{} \hfill $\blacksquare$}
\newcommand{\Real}{\mathbb{R}}
\newcommand{\Sgn}{\mbox{Sgn}}
\renewcommand*{\arraystretch}{1.4}
\newtheorem{theorem}{Result}
\newtheorem{definition}{Definition}
\lstset{frame=tb,
	language=Python,
	aboveskip=3mm,
	belowskip=3mm,
	showstringspaces=false,
	columns=flexible,
	basicstyle={\small\ttfamily},
	numbers=none,
	stringstyle=\color{mauve},
	breaklines=true,
	breakatwhitespace=true,
	tabsize=3
}

\title{Chapter 1: Probability Theory Exercises}
\author{Ran Xie}
\begin{document}
\maketitle

\section*{Exercise 1.1}
(a) $S=\{ ssss | s \in \{H, T\} \}$ , a string of length 4 with alphabet H and T \\

(b) $S=\mathbb{N}\cup\{0\}$ since damaged leaves are non-negative whole number\\

(c) $S=\mathbb{N}\cup\{0\}$ since we count in hours which is a non-negative whole number \\

(d) $S=\mathbb{R}^+$ since weight can be any positive real number \\

(e) $S= [0,1]$ fraction is between 0 and 1 \\


\section*{Exercise 1.4}
(a) $P(A\cup B \cup(A \cap B)) = P(A \cup B) = P(A) + P(B) - P(A \cap B)$  Note $A\cap B \subset A$ \\

(b) $$\begin{aligned}
P((A \cup B)\cap (A \cap B)^c) &= P(A\cup B) + P((A \cap B)^c) - P(A\cup B \cup (A \cap B)^c) \\
&= P(A\cup B) + 1 - P(A \cap B) - 1 \\
&= P(A) + P(B) - 2P(A\cap B) \\
\end{aligned}$$

(c) $P(A\cup B) = P(A) + P(B) - P(\cup B)$ \\

(d) $P(A) + P(B) - 2P(A\cap B)$ \\

\section*{Exercise 1.5}
(a) Note that $A \subset C$. Therefore $A \cap C = A$. Hence $A\cap B \cap C = A \cap B$ = 
$\{$ a U.S. birth results in identica twins and both being females $\}$.

(b) Let $D$ by the event of a twin birth. $$P(A \cap B \cap C) = P(A \cap B) = P(A | B, D) P(B | D ) P(D) = \frac{1}{2} \frac{1}{3} \frac{1}{90} = \frac{1}{540}  $$

\section*{Exercise 1.6}
We have $p_0 = (1-u)(1-w)$, $p_1 = u(1-w) + w(1-u)$, $p_2 = uw$. Also $p_0=p_1=p_2 = p$. Therefore we have 3 variables and 3 equations. 
$$\begin{aligned}
  uw - u - w + 1 &= p \\
  -2uw + u + w &= p \\
  uw &= p \\
\end{aligned}
$$
We get $p=\frac{1}{3}$, $uw = \frac{1}{3}$, $u + w = 1$. This only has imaginary solution. Hence there is no solution for $w$ and $u$ which satisfy the conditions.

\section*{Exercise 1.7}
(a) The dart board has area of $\pi r^2$. The probability of scoring $i$ points is
$$P(X=i) = \begin{cases}
	\frac{A - \pi r^2}{A}, \ \ &i=0 \\
	\frac{\pi r^2}{A} \left[ \left( \frac{6-i}{5} \right)^2 -  \left( \frac{5-i}{5}\right)^2  \right], \ \ &i =1,2,3,4,5
\end{cases}
$$
(b) $$ P(\mbox{i points}|\mbox{board hit}) = \frac{P(\mbox{i points and hit board})}{P(\mbox{hit board})} = \frac{\frac{\pi r^2}{A} \left[ \left( \frac{6-i}{5} \right)^2 -  \left( \frac{5-i}{5}\right)^2  \right]}{ \pi r^2/ A} = \left( \frac{6-i}{5} \right)^2 -  \left( \frac{5-i}{5}\right)^2 $$

\section*{Exercise 1.8}
(a) See 1.7 (a)

(b) The probability of scoring $i < j$ points corresponds to area of rings with area $A_i > A_j$. therefore the probability is decreasing function.

(c) Summing the distribution in 1.7 (a) shows it is equal to 1. And the probability is greater or equal to zero since areas can't be negative.


\section*{Exercise 1.13}
Note that $P(A\cup B) = P(A) + P(B) - P(A\cap B) \leq 1$. Therefore $P(A\cap B) \geq P(A) + P(B) - 1 =\frac{1}{12} \neq 0$, can't be disjoint.

\section*{Exercise 1.14}
Given $|S| = n$, we can order the elements such that $S=\{a_1,\ldots, a_n\}$, There exists a bijection from the set of binary string of length $n$, $B = \{b^n|b\in \{0,1\}\}$ to elements in power set of $S$ where $0$ at the ith position means the i-th element is not present in the subset and $1$ means otherwise.  For each bit in the binary string, there are two possible states $0$ and $1$. Therefore the total n-binary string count is $2^n$. By property of bijection, power set of $S$ also has the same number of elements.

\section*{Exercise 1.16}
(a) A person's name has 3 parts, so we have 3 character as initials, each with 26 choices. So $26^3$ possibilities.

(b) If a person can have only 1 given name, then we have $26^2$ for 2 character initials. The total possibilities together with (a) is $26^2 + 26^3$.

(c) By the same argument, for 3 given name, we have $26^4$. In total $26^2+26^3+26^4$.

\section*{Exercise 1.17}
Assume one piece can only have 2 different numbers and they are not order, we just need to choose 2 numbers from $n$. Therefore the number of combination is ${n \choose 2} = \frac{n(n-1)}{2}$

Now consider the case when a piece have both side the same number, there are $n$ possibilities.
Therefore the total is $\frac{n(n-1)}{2} + n  = \frac{n(n+1)}{2}$

\section*{Exercise 1.19}
Taking $r$th partial derivatives of a $n$ variable $f$, is just choosing $r$ variables without order with replacement from $n$ variables, which is ${n-1 + r \choose r}$

\section*{Exercise 1.21}
For no matching shoes, we can only choose 1 shoe from a pair, therefore we need to choose $2r$ pairs so $2r$ must be less than $n$. First we choose $2r$ shoes from $n$ pairs: ${n \choose 2r}$. For each pair, we can choose the left shoe or right shoe and we have $2r$ chosen pairs: $2^{2r}$. So the total way to choose non matching shoes is ${n \choose 2r} 2^{2r}$.  Total way of choosing is ${2n \choose 2r}$. So the probability is the ${n \choose 2r} 2^{2r} / {2n \choose 2r}$.

\section*{Exercise 1.26}
Let $T$ be the number of toss until a $6$ appears.
$P(T > 5) = 1- P(T <= 5) = 1 - \sum_{t=1}^5 \frac{5^{t-1}}{6^t} \approx 0.40$

\section*{Exercise 1.52}
Integrating $g(x)$, we have
\begin{equation*}
 G(x) = \int_{-\infty}^x g(t) = \begin{cases}
       \frac{F(x) - F(x_0)}{1-F(x_0)}, x \geq x_0 \\
       0, x < x_0 \\
     \end{cases}
\end{equation*}

$\lim\limits_{x \rightarrow -\infty} G(x) = 0$ and $\lim\limits_{x \rightarrow \infty} G(x) = \lim\limits_{x \rightarrow \infty} \frac{F(x) - F(x_0)}{1-F(x_0)} = \frac{1 - F(x_0)}{1-F(x_0)} = 1 $

Since $F(x_0)<1$ and $F(x)$ is right continuous, so $G(x)$ is also right continuous.


\section*{Exercise 1.53}
$\lim\limits_{y\rightarrow -\infty}F_Y(y) = \lim\limits_{y\rightarrow 1} (1- \frac{1}{y^2}) = 1 -1 = 0$ , 

$\lim\limits_{y\rightarrow \infty}F_Y(y) = \lim\limits_{y\rightarrow 1} (1- \frac{1}{y^2}) = 1 - 0 = 1$ 

$ (1 - 1/x^2) - (1 - 1/y^2) = 1/y^2 - 1/x^2 > 0 $ for $ x > y$ Therefore $F_Y$ is non-decreasing.

$1 - 1/y^2$ is smooth on $[1,\infty]$ hence right continuous. Therefore $F_Y$ is a cdf.\\

$f_Y(y) = \frac{d F_y}{dy} = \frac{2}{y^3}$

When $Z=10(Y-1)$, $$F_Z(z) = P(Z\leq z) = P(10(Y-1) \leq z) = P(Y \leq \frac{z}{10} + 1) = 1 - \frac{1}{(0.1z + 1)^2}$$ ,where $0 \leq z < \infty$. 0 otherwise. 

\section*{Exercise 1.54}
$c \int_0^{\pi/2}\sin{x} = c(- 0 + 1) = 1$ which gives $c = 1$

$c \int_{-\infty}^{\infty} \exp{-|x|} = 2c \int_{0}^{\infty}\exp{-x} = 2c  = 1$ which gives $c = \frac{1}{2}$

\section*{Exercise 1.55}
$F_T(t) = \int_{0}^{t} 1/1.5 \exp(-s/1.5)ds = 1 - \exp(-t/1.5)$

Note that $V\in [5, \infty)$

$F_V(5) = P(V \leq 5)= P(V = 5) = P(T < 3) = 1 -\exp(-2)$

When $v \in [5, 6)$,  $t \in [2.5, 3)$, Therefore $P(5 < V < 6) = 0$. By Cdf property, $P(V < 6) = P(V=5) = 1 -\exp(-2)$

When $v \in [6, \infty)$, $t \in [3, \infty)$, therefore $F_V(v) = P(V < v) =  P(2T \leq v) = P(T \leq v/2) = 1 -\exp(-v/3)$

Note that the cdf is continuous at $V=6$.


\end{document}