\documentclass[12pt]{article}
\usepackage{fullpage}
\usepackage{times}
\usepackage[normalem]{ulem}
\usepackage{fancyhdr,graphicx,amsmath,amssymb, mathtools, scrextend, titlesec, enumitem}\usepackage[pdftex]{hyperref}
\usepackage[ruled,vlined]{algorithm2e} 
\usepackage{parskip}
\include{pythonlisting}

\title{Chapter 1: Probability Theory Exercises}
\author{Ran Xie}
\begin{document}
\maketitle

\section*{Exercise 1.1}
(a) $S=\{ ssss | s \in \{H, T\} \}$ , a string of length 4 with alphabet H and T \\

(b) $S=\mathbb{N}\cup\{0\}$ since damaged leaves are non-negative whole number\\

(c) $S=\mathbb{N}\cup\{0\}$ since we count in hours which is a non-negative whole number \\

(d) $S=\mathbb{R}^+$ since weight can be any positive real number \\

(e) $S= [0,1]$ fraction is between 0 and 1 \\

\section*{Exercise 1.4}
(a) $P(A\cup B \cup(A \cap B)) = P(A \cup B) = P(A) + P(B) - P(A \cap B)$  Note $A\cap B \subset A$ \\

(b) $$\begin{aligned}
P((A \cup B)\cap (A \cap B)^c) &= P(A\cup B) + P((A \cap B)^c) - P(A\cup B \cup (A \cap B)^c) \\
&= P(A\cup B) + 1 - P(A \cap B) - 1 \\
&= P(A) + P(B) - 2P(A\cap B) \\
\end{aligned}$$

(c) $P(A\cup B) = P(A) + P(B) - P(\cup B)$ \\

(d) $P(A) + P(B) - 2P(A\cap B)$ \\

\section*{Exercise 1.6}
We have $p_0 = (1-u)(1-w)$, $p_1 = u(1-w) + w(1-u)$, $p_2 = uw$. Also $p_0=p_1=p_2 = p$. Therefore we have 3 variables and 3 equations. 
$$\begin{aligned}
  uw - u - w + 1 &= p \\
  -2uw + u + w &= p \\
  uw &= p \\
\end{aligned}
$$
We get $p=\frac{1}{3}$, $uw = \frac{1}{3}$, $u + w = 1$. This only has imaginary solution. Hence there is no solution for $w$ and $u$ which satisfy the conditions.

\section*{Exercise 1.13}
Note that $P(A\cup B) = P(A) + P(B) - P(A\cap B) \leq 1$. Therefore $P(A\cap B) \geq P(A) + P(B) - 1 =\frac{1}{12} \neq 0$, can't be disjoint.

\section*{Exercise 1.14}
Given $|S| = n$, we can order the elements such that $S=\{a_1,\ldots, a_n\}$, There exists a bijection from the set of binary string of length $n$, $B = \{b^n|b\in \{0,1\}\}$ to elements in power set of $S$ where $0$ at the ith position means the i-th element is not present in the subset and $1$ means otherwise.  For each bit in the binary string, there are two possible states $0$ and $1$. Therefore the total n-binary string count is $2^n$. By property of bijection, power set of $S$ also has the same number of elements.

\section*{Exercise 1.19}
Taking $r$th partial derivatives of a $n$ variable $f$, is just choosing $r$ variables without order with replacement from $n$ variables, which is ${n-1 + r \choose r}$

\section*{Exercise 1.21}
For no matching shoes, we can only choose 1 shoe from a pair, therefore we need to choose $2r$ pairs so $2r$ must be less than $n$. First we choose $2r$ shoes from $n$ pairs: ${n \choose 2r}$. For each pair, we can choose the left shoe or right shoe and we have $2r$ chosen pairs: $2^{2r}$. So the total way to choose non matching shoes is ${n \choose 2r} 2^{2r}$.  Total way of choosing is ${2n \choose 2r}$. So the probability is the ${n \choose 2r} 2^{2r} / {2n \choose 2r}$.

\section*{Exercise 1.26}
Let $T$ be the number of toss until a $6$ appears.
$P(T > 5) = 1- P(T <= 5) = 1 - \sum_{t=1}^5 \frac{5^{t-1}}{6^t} \approx 0.40$

\section*{Exercise 1.52}
Integrating $g(x)$, we have
\begin{equation*}
 G(x) = \int_{-\infty}^x g(t) = \begin{cases}
       \frac{F(x) - F(x_0)}{1-F(x_0)}, x \geq x_0 \\
       0, x < x_0 \\
     \end{cases}
\end{equation*}

$\lim\limits_{x \rightarrow -\infty} G(x) = 0$ and $\lim\limits_{x \rightarrow \infty} G(x) = \lim\limits_{x \rightarrow \infty} \frac{F(x) - F(x_0)}{1-F(x_0)} = \frac{1 - F(x_0)}{1-F(x_0)} = 1 $

Since $F(x_0)<1$ and $F(x)$ is right continuous, so $G(x)$ is also right continuous.


\section*{Exercise 1.53}
$\lim\limits_{y\rightarrow -\infty}F_Y(y) = \lim\limits_{y\rightarrow 1} (1- \frac{1}{y^2}) = 1 -1 = 0$ , 

$\lim\limits_{y\rightarrow \infty}F_Y(y) = \lim\limits_{y\rightarrow 1} (1- \frac{1}{y^2}) = 1 - 0 = 1$ 

$ (1 - 1/x^2) - (1 - 1/y^2) = 1/y^2 - 1/x^2 > 0 $ for $ x > y$ Therefore $F_Y$ is non-decreasing.

$1 - 1/y^2$ is smooth on $[1,\infty]$ hence right continuous. Therefore $F_Y$ is a cdf.\\

$f_Y(y) = \frac{d F_y}{dy} = \frac{2}{y^3}$

When $Z=10(Y-1)$, $$F_Z(z) = P(Z\leq z) = P(10(Y-1) \leq z) = P(Y \leq \frac{z}{10} + 1) = 1 - \frac{1}{(0.1z + 1)^2}$$ ,where $0 \leq z < \infty$. 0 otherwise. 

\section*{Exercise 1.54}
$c \int_0^{\pi/2}\sin{x} = c(- 0 + 1) = 1$ which gives $c = 1$

$c \int_{-\infty}^{\infty} \exp{-|x|} = 2c \int_{0}^{\infty}\exp{-x} = 2c  = 1$ which gives $c = \frac{1}{2}$

\section*{Exercise 1.55}
$F_T(t) = \int_{0}^{t} 1/1.5 \exp(-s/1.5)ds = 1 - \exp(-t/1.5)$

Note that $V\in [5, \infty)$

$F_V(5) = P(V \leq 5)= P(V = 5) = P(T < 3) = 1 -\exp(-2)$

When $v \in [5, 6)$,  $t \in [2.5, 3)$, Therefore $P(5 < V < 6) = 0$. By Cdf property, $P(V < 6) = P(V=5) = 1 -\exp(-2)$

When $v \in [6, \infty)$, $t \in [3, \infty)$, therefore $F_V(v) = P(V < v) =  P(2T \leq v) = P(T \leq v/2) = 1 -\exp(-v/3)$

Note that the cdf is continuous at $V=6$.


\end{document}