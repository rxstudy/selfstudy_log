\documentclass[12pt]{article}

\include{pythonlisting}
\usepackage{fullpage}
\usepackage{times}
\usepackage[normalem]{ulem}
\usepackage{multirow}
\usepackage{fancyhdr,graphicx,amsmath,amssymb, mathtools, scrextend, titlesec, enumitem}\usepackage[pdftex]{hyperref}
\usepackage[ruled,vlined]{algorithm2e} 
\usepackage{parskip}
\usepackage{listings}
\newcommand{\Var}{\mathrm{Var}}
\newcommand{\Cov}{\mathrm{Cov}}
\newcommand{\E}{{\rm I\kern-.3em E}}
\newcommand{\Binomial}{\mathrm{Binomial}}
\newcommand{\Beta}{\mathrm{Beta}}
\newcommand{\Poisson}{\mathrm{Poisson}}
\newcommand{\Normal}{\mathcal{N}}
\newcommand{\Uniform}{\mathrm{Uniform}}

\lstset{frame=tb,
  language=Python,
  aboveskip=3mm,
  belowskip=3mm,
  showstringspaces=false,
  columns=flexible,
  basicstyle={\small\ttfamily},
  numbers=none,
  stringstyle=\color{mauve},
  breaklines=true,
  breakatwhitespace=true,
  tabsize=3
}

\title{Chapter 3: Common Families of Distribution}
\begin{document}
\maketitle

\section*{Exercise 3.1}
There are $N_1 - N_0 +1$ numbers, therefore $P(x = n) = \frac{1}{N_1-N_0+1}$.

$EX = \frac{N_1+N_0}{2}$ which is just the midpoint.

Let $b=N_1, a=N_0$
$$
\begin{aligned}
\Var &X=EX^2 - (EX)^2 \\
    &= \frac{1}{b-a+1} \sum^{b}_{a}x^2  - (EX)^2  \\
   &= \frac{1}{b-a+1}\left[\sum^{b}_{1}x^2 - \sum^{a-1}_{1}x^2 \right] - (EX)^2 \\
   &= \frac{1}{b-a+1}\left[\frac{b(b +1)(2b+1)}{6} - \frac{(a-1)a(2a-1)}{6}\right] -  \frac{(b+a)^2}{4} \\
   &= \frac{2b(b +1)(2b+1)- 2(a-1)a(2a-1)-3(b-a+1)(b+a)^2}{12(b-a+1)} \\
   &= \frac{2b(b -a +1 + a)(2b+1) + 2a(b - a + 1 - b)(2a-1)-3(b-a+1)(b+a)^2}{12(b-a+1)} \\
   &= \frac{2b(b -a +1)(2b+1) + 2a(b - a + 1)(2a-1)-3(b-a+1)(b+a)^2 -4ab(b-a+1)}{12(b-a+1)} \\
   &= \frac{2b(2b+1) + 2a(2a-1)-3(b+a)^2 -4ab}{12} \\
   &= \frac{a^2+b^2 -2ab +2b-2a}{12} \\
   &= \frac{(b-a)(b-a+2)}{12} \\
   &= \frac{(N_1-N_0)(N_1-N_0+2)}{12} \\
\end{aligned}
$$

\section*{Exercise 3.2}
(a)
Let $X$ be the number of defective part in $K$ samples and $M$ be the total defective parts in 100 parts. Then
$$P(X=0 | M > 5) = \frac{{100 -M \choose K}}{{100 \choose K}}$$ 
is the probability of accepting a defective product given $M > 5$. 
To bound $K$, we can set $M = 6$ since defect parts becomes more prevalent which increases the chance for them to be sampled, setting $M=6$ maximizes the false positive rate $P(X=0|M)$.

Then
$$P(X=0 | M=6) = \frac{{94 \choose K}}{{100 \choose K}} < 0.1$$
Solving for $K$ numerically (polynomial of the 5th power), we get $K >31$. We can choose $K=32$.

(b)
The false positive rate is now 
$$P(X\leq 1 | M = 6) = P(X=0|M=6) + P(X=1|M=6)=  \frac{{94 \choose K}}{{100 \choose K}} +  \frac{{6 \choose 1}{94 \choose K-1}}{{100 \choose K}} < 0.1 $$
Solving for $K$ numerically (same as above except there's an additional term $1 + \frac{6K}{95-K}$), We get $K > 50.24$ which means $K=51$.

\section*{Exercise 3.4}
(a) Without eliminating the wrong key, every trial is independent with probability of $\frac{1}{n}$ for succeeding. Let $X$ be the number of tries before succeeding. 
$$P(X=k) = \left(1 - \frac{1}{n}\right)^{k-1} \frac{1}{n}$$
It is geometric distribution therefore $EX = \frac{1}{1/n} = n$

(b) By eliminating the wrong key, at $k$-th trial, we will be selecting from $n - k + 1$ remaining keys, the success probability is $\frac{1}{n-k+1}$. 
$$P(X=k) = \frac{1}{n-k+1} \prod^{k-1}_{i=1} \left(1- \frac{1}{n-i+1} \right)$$
Then
$$\begin{aligned}
EX &= \sum_{x=1}^{n} xP(x) \\
   &= \sum_{k=1}^{n} \frac{k}{n-k+1} \prod^{k-1}_{i=1} \left(1- \frac{1}{n-i+1} \right) \\
   &= \sum_{k=1}^{n} \frac{k}{n-k+1} \prod^{k-1}_{i=1} \frac{n -i}{n-i+1} \\
   &= \sum_{k=1}^{n} \frac{k}{n-k+1} \frac{n - k + 1}{n} \\
   &= \sum_{k=1}^{n} \frac{k}{n} \\
   &= \frac{n+1}{2}
\end{aligned}$$

\section*{Exercise 3.8}
Let $X$ be the number of people who choose theatre 1. Then when $X \leq N$ will be the event theatre 1 not turning away customers and $1000 - X \leq N$ will be when theatre 2 not turning away customers.
We will find the reverse: $N$ such that the probability of both theatre not turning away customers is greater than 99%.  $P(1000 -N \leq X \leq N) > 0.99)$

The binomial distribution for $X$ is $$ P(X = k) = {1000 \choose k} 0.5^{1000}$$
From $1000 -N \leq X \leq N$, we get $N \geq 500$ (If total seats of two theatres < customers, one of them will turn away customers)
Then
$$ 0.5^{1000} \sum^{N}_{1000-N} {1000 \choose k} > 0.99, \ \ N \geq 500$$
\begin{lstlisting}
 from scipy.special import comb
 def p(N):
   return sum([comb(1000, i, exact=True) for i in range(1000-N, N+1)]) / (2**1000)
\end{lstlisting}
We get $p(540) \approx 0.9896$ and $p(541) \approx 0.9913$. Therefore we can take $N = 540$.

\section*{Exercise 3.20}
(a) $$ \E X = \int^{\infty}_{0} xf(x) dx = \frac{2}{\sqrt{2\pi}} \int^{\infty}_{0} x e^{-x^2/2} dx
 = - \frac{2}{\sqrt{2\pi}} e^{-x^2/2} |^{\infty}_{0} =  \frac{2}{\sqrt{2\pi}}$$
 
 $$\E X^2 = \int^{\infty}_{0} x^2f(x) dx =  \frac{2}{\sqrt{2\pi}} \int^{\infty}_{0} x^2 e^{-x^2/2} dx$$
 
To calculate the integral, use integration by part $dg = x e^{-x^2/2}dx$, $f = x$. Therefore $g = -e^{-x^2/2}$ and $df=dx$. 

$$\int^{\infty}_{0} x^2 e^{-x^2/2} dx = -xe^{-x^2/2}|^{\infty}_{0} + \int^{\infty}_{0} e^{-x^2/2} dx = \sqrt{\frac{\pi}{2}}$$

So $$EX^2 = \frac{2}{\sqrt{2\pi}}  \sqrt{\frac{\pi}{2}} = 1$$
Therefore $\Var X = \E X^2 - (\E X)^2 = 1 - \left( \frac{2}{\sqrt{2\pi}}\right)^2 = 1 - \frac{2}{\pi}$ 

(b) Let $y=sx^2$, by change of variable,
$$ f_Y(y) = f_X(x(y)) \left| \frac{dx}{dy} \right| = \frac{1}{\sqrt{2\pi s}} y^{1/2} e^{-\frac{y}{2s}}$$
Now we compare it again the gamma distribution $$ f_Y(y|\alpha, \beta) = \frac{1}{\Gamma(\alpha)\beta^{\alpha}} y^{\alpha - 1} e^{-y/\beta}$$
First we conclude $\beta = 2s$, $\alpha = 1/2$. Then we have $\Gamma(\alpha)\beta^{\alpha} = \Gamma(1/2)\sqrt{2s} = \sqrt{2\pi s}$ which is consistent with the above. Therefore the change of variable is $y = \frac{\beta}{2} x^2$, and $Y \sim gamma(\alpha=1/2, \beta > 0)$
\end{document}
