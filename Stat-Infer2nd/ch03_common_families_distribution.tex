\documentclass[12pt]{article}

\include{pythonlisting}
\usepackage{fullpage}
\usepackage{times}
\usepackage[normalem]{ulem}
\usepackage{multirow}
\usepackage{fancyhdr,graphicx,amsmath,amssymb, mathtools, scrextend, titlesec, enumitem}\usepackage[pdftex]{hyperref}
\usepackage[ruled,vlined]{algorithm2e}
\usepackage{parskip}
\usepackage{listings}
\usepackage{amsmath}
\usepackage{physics}
\usepackage{bbm}
\usepackage{titling}
\usepackage{bm}
\usepackage{geometry}
\geometry{top=1cm, left=1cm, bottom=1.5cm, right=1.5cm, margin=1cm}
\newcommand{\varvec}[2][n]{#2_1,\ldots, #2_#1}
\newcommand{\Var}{\mathrm{Var}}
\newcommand{\Bias}{\mathrm{Bias}}
\newcommand{\Cov}{\mathrm{Cov}}
\newcommand{\E}{{\rm I\kern-.3em E}}
\newcommand{\Binomial}{\mathrm{Binomial}}
\newcommand{\Bernoulli}{\mathrm{Bernoulli}}
\newcommand{\Poisson}{\mathrm{Poisson}}
\newcommand{\Normal}{\mathcal{N}}
\newcommand{\one}{\mathbbm{1}}
\newcommand{\Beta}{\text{Beta}}
\newcommand{\BetaPDF}{\text{Beta}}
\newcommand{\GammaPDF}{\text{Gamma}}
\newcommand{\Uniform}{\mathrm{Uniform}}
\newcommand{\QED}{\newline \mbox{} \hfill $\blacksquare$}
\newcommand{\Real}{\mathbb{R}}
\newcommand{\Sgn}{\mbox{Sgn}}
\renewcommand*{\arraystretch}{1.4}
\newtheorem{theorem}{Result}
\newtheorem{definition}{Definition}
\lstset{frame=tb,
	language=Python,
	aboveskip=3mm,
	belowskip=3mm,
	showstringspaces=false,
	columns=flexible,
	basicstyle={\small\ttfamily},
	numbers=none,
	stringstyle=\color{mauve},
	breaklines=true,
	breakatwhitespace=true,
	tabsize=3
}

\title{Chapter 3: Common Families of Distribution}
\begin{document}
\maketitle

\section*{Exercise 3.1}
There are $N_1 - N_0 +1$ numbers, therefore $P(x = n) = \frac{1}{N_1-N_0+1}$.

$EX = \frac{N_1+N_0}{2}$ which is just the midpoint.

Let $b=N_1, a=N_0$
$$
\begin{aligned}
\Var &X=EX^2 - (EX)^2 \\
    &= \frac{1}{b-a+1} \sum^{b}_{a}x^2  - (EX)^2  \\
   &= \frac{1}{b-a+1}\left[\sum^{b}_{1}x^2 - \sum^{a-1}_{1}x^2 \right] - (EX)^2 \\
   &= \frac{1}{b-a+1}\left[\frac{b(b +1)(2b+1)}{6} - \frac{(a-1)a(2a-1)}{6}\right] -  \frac{(b+a)^2}{4} \\
   &= \frac{2b(b +1)(2b+1)- 2(a-1)a(2a-1)-3(b-a+1)(b+a)^2}{12(b-a+1)} \\
   &= \frac{2b(b -a +1 + a)(2b+1) + 2a(b - a + 1 - b)(2a-1)-3(b-a+1)(b+a)^2}{12(b-a+1)} \\
   &= \frac{2b(b -a +1)(2b+1) + 2a(b - a + 1)(2a-1)-3(b-a+1)(b+a)^2 -4ab(b-a+1)}{12(b-a+1)} \\
   &= \frac{2b(2b+1) + 2a(2a-1)-3(b+a)^2 -4ab}{12} \\
   &= \frac{a^2+b^2 -2ab +2b-2a}{12} \\
   &= \frac{(b-a)(b-a+2)}{12} \\
   &= \frac{(N_1-N_0)(N_1-N_0+2)}{12} \\
\end{aligned}
$$

\section*{Exercise 3.2}
(a)
Let $X$ be the number of defective part in $K$ samples and $M$ be the total defective parts in 100 parts. Then
$$P(X=0 | M > 5) = \frac{{100 -M \choose K}}{{100 \choose K}}$$ 
is the probability of accepting a defective product given $M > 5$. 
To bound $K$, we can set $M = 6$ since defect parts becomes more prevalent which increases the chance for them to be sampled, setting $M=6$ maximizes the false positive rate $P(X=0|M)$.

Then
$$P(X=0 | M=6) = \frac{{94 \choose K}}{{100 \choose K}} < 0.1$$
Solving for $K$ numerically (polynomial of the 5th power), we get $K >31$. We can choose $K=32$.

(b)
The false positive rate is now 
$$P(X\leq 1 | M = 6) = P(X=0|M=6) + P(X=1|M=6)=  \frac{{94 \choose K}}{{100 \choose K}} +  \frac{{6 \choose 1}{94 \choose K-1}}{{100 \choose K}} < 0.1 $$
Solving for $K$ numerically (same as above except there's an additional term $1 + \frac{6K}{95-K}$), We get $K > 50.24$ which means $K=51$.

\section*{Exercise 3.3}
Pedestrian takes 3 seconds to cross the road if there is no car during the 3 seconds. If a pedestrian waits exact 4 seconds to cross, the process is 7 seconds. The last 3 seconds there needs to have no cars so the probability is $(1-p)^3$. For the pedestrian to cross after 4th seconds, the 4th second must have car (otherwise, event such as (1,1,1,0,0,0,0) will have the pedestrian wait only 3 seconds).The only case that pedestrian will cross without waiting for 4 seconds even when the 4th second has car is (0,0,0,1) for the first 4 seconds. Therefore the probability of user waiting exactly 4 seconds is $1-p(1-p)^3$. The probability of user crossing after 4th second is $(1-p)^3$. The probability of the entire process is the product: $[1-p(1-p)^3](1-p)^3$

\section*{Exercise 3.4}
(a) Without eliminating the wrong key, every trial is independent with probability of $\frac{1}{n}$ for succeeding. Let $X$ be the number of tries before succeeding. 
$$P(X=k) = \left(1 - \frac{1}{n}\right)^{k-1} \frac{1}{n}$$
It is geometric distribution therefore $EX = \frac{1}{1/n} = n$

(b) By eliminating the wrong key, at $k$-th trial, we will be selecting from $n - k + 1$ remaining keys, the success probability is $\frac{1}{n-k+1}$. 
$$P(X=k) = \frac{1}{n-k+1} \prod^{k-1}_{i=1} \left(1- \frac{1}{n-i+1} \right)$$
Then
$$\begin{aligned}
EX &= \sum_{x=1}^{n} xP(x) \\
   &= \sum_{k=1}^{n} \frac{k}{n-k+1} \prod^{k-1}_{i=1} \left(1- \frac{1}{n-i+1} \right) \\
   &= \sum_{k=1}^{n} \frac{k}{n-k+1} \prod^{k-1}_{i=1} \frac{n -i}{n-i+1} \\
   &= \sum_{k=1}^{n} \frac{k}{n-k+1} \frac{n - k + 1}{n} \\
   &= \sum_{k=1}^{n} \frac{k}{n} \\
   &= \frac{n+1}{2}
\end{aligned}$$

\section*{Exercise 3.5}
We want to first assume that if new drug has the same effectiveness of the old drug and see how often would we observe the result of 85 effective cases out of 100 to make sure the experiment result is not a coincident. (Or how often standard drug can produce the same trial result as the new drug).

Let $X$ be the number of effective cases out of 100. Then $X \sim \Binomial (100, 0.8)$.

$$P(X \geq 85) = \sum^{100}_{k=85}{100 \choose 85} 0.8^{85} 0.2^{15} \approx 0.1285$$

So there are about $13\%$ chance that the standard drug can produce the same result. We can't conclude the new drug is better.

\section*{Exercise 3.6}
(a) We can use binomial distribution $\Binomial (2000, 0.01)$. Then
$$ P(X= k) = {2000 \choose k} 0.01^k  0.99^{2000 -k }$$

(b) $$ P(X < 100) = \sum^{99}_{k=0} {2000 \choose k} 0.01^k  0.99^{2000 -k } $$

(c) A conservative rule to check if a normal approximation is good is $\min (np, n(1-p)) = 20 \geq 5$. Therefore we can approximate it with $\Normal (\mu=20, \sigma = \sqrt{20(0.99)}) $.
$$P ( X < 100) = P \left(\frac{X - 20}{\sqrt{20(0.99)}} \leq (\frac{100 - 20}{\sqrt{20(0.99)}}\right) = P(Z < 17.98) \approx 1 $$

\section*{Exercise 3.7}
Let $X$ be the number of chocolate chips in the cookie and $X \sim \Poisson (\lambda)$. Then 
$$P(X \geq 2 ) = 1 - P(X = 0) - P(X = 1) = 1 - e^{-\lambda} - \lambda e^{-\lambda} > 0.99$$
We get $\lambda \approx 6.64$.

\section*{Exercise 3.8}
(a) Let $X$ be the number of people who choose theatre 1. Then when $X \leq N$ will be the event theatre 1 not turning away customers and $1000 - X \leq N$ will be when theatre 2 not turning away customers.
We will find the reverse: $N$ such that the probability of both theatre not turning away customers is greater than 0.99 .  $P(1000 -N \leq X \leq N) > 0.99)$

The binomial distribution for $X$ is $$ P(X = k) = {1000 \choose k} 0.5^{1000}$$
From $1000 -N \leq X \leq N$, we get $N \geq 500$ (If total seats of two theatres is less than number of customers, one of them will certainly turn away customers).

We have
$$ 0.5^{1000} \sum^{N}_{1000-N} {1000 \choose k} > 0.99, \ \ N \geq 500$$
\begin{lstlisting}
 from scipy.special import comb
 def p(N):
   return sum([comb(1000, i, exact=True) for i in range(1000-N, N+1)]) / (2**1000)
\end{lstlisting}
We get $p(540) \approx 0.9896$ and $p(541) \approx 0.9913$. Therefore we can take $N = 541$.

(b) Since $X \sim \Binomial (1000, 1/2)$. We can approximate with a normal distributed $Y \sim \Normal (1000 \times 0.5, \sqrt(1000 \times 0.5 \times 0.5)) = \Normal (500, \sqrt{250})$
$$ P(1000- N \leq X \leq N) \approx P( 1000 - N \leq Y \leq N) = P(- \frac{N - 500}{\sqrt{250}} \leq Z \leq \frac{N - 500}{\sqrt{250}}) > 0.99$$

This means we need to look for a $z$ score where the probability is greater than 0.995 by symmetry. So $z \geq 2.58$. So $N \geq 2.58 (\sqrt{250}) + 500 = 540.8 \approx 541$

\section*{Exercise 3.9}
(a) we can think of it as $\Binomial (60, 1/90)$ for a success being having a twin birth (It is natural to assume the twin will sign up for the same elementary school). Therefore having 5 or more twin birth can be approximately
$$ P(X \geq 5) = 1 - \sum^{4}_{k=0}{60 \choose k} (1/90)^k (89/90)^{60-k} \approx 0.0006 $$

(b) Within New York State, we can calculate the probability of at least one such school. $Y \sim \Binomial (62(5)=310, 0.0006)$
$$ P(Y \geq 1) = 1- P(Y=0) = 1 - (1-0.0006)^{310} = 0.17$$
Not a rare occurrence.

(c) For 50 states in even one year, there are 50 independent trial. The probability of having at least 1 success is then $$ 1 - (1 - 0.17)^{50} \approx 0.9999$$ 
which is close to certain occurrence.

\section*{Exercise 3.20}
(a) $$ \E X = \int^{\infty}_{0} xf(x) dx = \frac{2}{\sqrt{2\pi}} \int^{\infty}_{0} x e^{-x^2/2} dx
 = - \frac{2}{\sqrt{2\pi}} e^{-x^2/2} |^{\infty}_{0} =  \frac{2}{\sqrt{2\pi}}$$
 
 $$\E X^2 = \int^{\infty}_{0} x^2f(x) dx =  \frac{2}{\sqrt{2\pi}} \int^{\infty}_{0} x^2 e^{-x^2/2} dx$$
 
To calculate the integral, use integration by part $dg = x e^{-x^2/2}dx$, $f = x$. Therefore $g = -e^{-x^2/2}$ and $df=dx$. 

$$\int^{\infty}_{0} x^2 e^{-x^2/2} dx = -xe^{-x^2/2}|^{\infty}_{0} + \int^{\infty}_{0} e^{-x^2/2} dx = \sqrt{\frac{\pi}{2}}$$

So $$EX^2 = \frac{2}{\sqrt{2\pi}}  \sqrt{\frac{\pi}{2}} = 1$$
Therefore $\Var X = \E X^2 - (\E X)^2 = 1 - \left( \frac{2}{\sqrt{2\pi}}\right)^2 = 1 - \frac{2}{\pi}$ 

(b) Let $y=sx^2$, by change of variable,
$$ f_Y(y) = f_X(x(y)) \left| \frac{dx}{dy} \right| = \frac{1}{\sqrt{2\pi s}} y^{1/2} e^{-\frac{y}{2s}}$$
Now we compare it again the gamma distribution $$ f_Y(y|\alpha, \beta) = \frac{1}{\Gamma(\alpha)\beta^{\alpha}} y^{\alpha - 1} e^{-y/\beta}$$
First we conclude $\beta = 2s$, $\alpha = 1/2$. Then we have $\Gamma(\alpha)\beta^{\alpha} = \Gamma(1/2)\sqrt{2s} = \sqrt{2\pi s}$ which is consistent with the above. Therefore the change of variable is $y = \frac{\beta}{2} x^2$, and $Y \sim gamma(\alpha=1/2, \beta > 0)$
\end{document}
