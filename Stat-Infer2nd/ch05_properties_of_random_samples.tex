\documentclass[12pt]{article}

\include{pythonlisting}
\usepackage{fullpage}
\usepackage{times}
\usepackage[normalem]{ulem}
\usepackage{multirow}
\usepackage{fancyhdr,graphicx,amsmath,amssymb, mathtools, scrextend, titlesec, enumitem}\usepackage[pdftex]{hyperref}
\usepackage[ruled,vlined]{algorithm2e}
\usepackage{parskip}
\usepackage{listings}
\usepackage{amsmath}
\usepackage{physics}
\usepackage{bbm}
\usepackage{titling}
\usepackage{bm}
\usepackage{geometry}
\geometry{top=1cm, left=1cm, bottom=1.5cm, right=1.5cm, margin=1cm}
\newcommand{\varvec}[2][n]{#2_1,\ldots, #2_#1}
\newcommand{\Var}{\mathrm{Var}}
\newcommand{\Bias}{\mathrm{Bias}}
\newcommand{\Cov}{\mathrm{Cov}}
\newcommand{\E}{{\rm I\kern-.3em E}}
\newcommand{\Binomial}{\mathrm{Binomial}}
\newcommand{\Bernoulli}{\mathrm{Bernoulli}}
\newcommand{\Poisson}{\mathrm{Poisson}}
\newcommand{\Normal}{\mathcal{N}}
\newcommand{\one}{\mathbbm{1}}
\newcommand{\Beta}{\text{Beta}}
\newcommand{\BetaPDF}{\text{Beta}}
\newcommand{\GammaPDF}{\text{Gamma}}
\newcommand{\Uniform}{\mathrm{Uniform}}
\newcommand{\QED}{\newline \mbox{} \hfill $\blacksquare$}
\newcommand{\Real}{\mathbb{R}}
\newcommand{\Sgn}{\mbox{Sgn}}
\renewcommand*{\arraystretch}{1.4}
\newtheorem{theorem}{Result}
\newtheorem{definition}{Definition}
\lstset{frame=tb,
	language=Python,
	aboveskip=3mm,
	belowskip=3mm,
	showstringspaces=false,
	columns=flexible,
	basicstyle={\small\ttfamily},
	numbers=none,
	stringstyle=\color{mauve},
	breaklines=true,
	breakatwhitespace=true,
	tabsize=3
}

\title{Chapter 5: Properties of Random Samples}
\begin{document}
\maketitle

\section*{Exercise 5.1}
The samples are drawn from Bernoulli trial with success rate $0.01$. The probability of $n$ samples not containing color-blind is $0.99^n$.  We want to find $N$ such that for $n \geq N$,  $0.99^n \leq 1- 0.95$. $N \approx 299$. \QED

\section*{Exercise 5.2}
(a) Let $T$ be the number of years until the first year's rainfall is exceeded. Then 
$$\begin{aligned}
P(T=k) &= P(X_2 \leq X_1, \ldots , X_{k-1} \leq X_1, X_k > X_1) \\
&= \int_{x}  P(X_2 \leq x, \ldots , X_{k-1} \leq x, X_k > x | X_1 = x)f(x) dx \\
&= \int_{x} P(X_k > x) f(x) \prod_{i=2}^{k-1} P(X_i \leq x) dx  \\
&= \int_{x} (1- F(x)) f(x) F(x)^{k-1} dx  \\
&= \int_{x}  F(x)^{k-1}f(x) dx - \int_{x}  F(x)^k f(x) dx  \\
&= \frac{1}{k} F(x)^k|^{\infty}_{-\infty} - \frac{1}{k+1} F(x)^{k+1}|^{\infty}_{-\infty} \\
&= \frac{1}{k} - \frac{1}{k+1} \\
&= \frac{1}{k(k+1)}
\end{aligned} 
$$
\QED

(b) $$\E T = \sum_k k P(T=k) = \sum_k \frac{1}{k+1} = \infty $$  \QED

\section*{Exercise 5.3}
Since $\{X_i\}$ are i.i.d $\sim F_X(x)$ and $Y_i$ is hierachical wrt to $X_i$. $Y_i \sim \Bernoulli(P(X_i > \mu)) | X_i$. So $Y_i$ are i.i.d. Therefore the sum of $Y_i \sim \Binomial (n, P(X_i > \mu)) = \Binomial (n, 1 - F_X(\mu)) $.

$$ P(Y_i = k) = {n \choose k} (1 - F_X(\mu))^k F_X(\mu)^{n-k}$$      
\QED

\section*{Exercise 5.4}
(a) $X_i | P \sim \Bernoulli(P)$ are i.i.d and $P \sim \Uniform(0,1)$. Let $ T = \sum_{i=1}^k X_i$.
$$ \begin{aligned}
	P(X_1=x_1,\ldots, X_k=x_k) &= \int^1_0 P(X_1=x_1,\ldots, X_k=x_k |P=p)f(p)dp \\
	&= \int_0^1 \prod_{i=1}^k P(X_i=x_i |P=p)f(p) dp, \mbox{ Since } X_i|P \mbox{ are i.i.d} \\
	&= \int_0^1 \prod_{i=1}^k p^{x_i}(1-p)^{1-x_i}f(p) dp \\
	&= \int_0^1 p^{\sum_i x_i}(1-p)^{1-\sum_i x_i}f(p) dp \\
	&= \int_0^1 p^t(1-p)^{1-t}f(p) dp
\end{aligned} 
$$ 
where $t = \sum_i x_i$. \QED 

(b) From (a),
  $$ P(X_1=x_1,\ldots, X_n=x_n) = \int_0^1 p^t(1-p)^{1-t}f(p) dp  $$
where $t = \sum_i^n x_i$.

On the other hand,  $$\prod^n_i P(X_i=x_i) = \prod^n_i \int_0^1 P(X_i=x_i | P=p)f(p)dp = \prod^n_i \int_0^1 p^{x_i}(1-p)^{1-x_i}dp $$
Therefore $P(X_1=x_1,\ldots, X_n=x_n) \neq \prod^n_i P(X_i=x_i)$. \QED

\section*{Exercise 5.5}
let $Y = \sum_i X_i$ then $\bar{X} =Y/n$. Suppose we have $f_Y(y)$,  then $$f_{\bar{X}}(\bar{x}) = f_Y(y) = f_Y(n \bar{x}) \left|\frac{dy}{d\bar{x}}\right| = n f_Y(n\bar{x})
$$

\section*{Exercise 5.6}
*Book has typos, it should be 5.2.9 instead of 5.2.3.

(a) Let $Z= X+Y$, $V=X$, then 
  $$ f_{V, Z}(v, z) = f_{X, Y}(v, z-v) \left| \frac{\partial (X, Y)}{\partial(V, Z)}\right| =f_{X, Y}(v, z-v)  \begin{vmatrix}1 & 0 \\ -1 & 1 \end{vmatrix} = f_{X, Y}(v, z-v) $$
Therefore $$f_Z(z) = \int_v f_{X, Y}(v, z-v)dv =\int_v f_{X}(v) f_{Y}(z-v)dv $$

(b) Let $Z= XY$, $V = X$, then
$$ f_{V, Z}(v, z) = f_{X, Y}(v, z/v) \left| \frac{\partial (X, Y)}{\partial(V, Z)}\right| =f_{X, Y}(v, z/v)  \begin{vmatrix}1 & 0 \\ -z/v^2 & 1/v \end{vmatrix} = f_{X, Y}(v, z-v) \left| \frac{1}{v} \right| $$
Therefore $$f_Z(z) = \int_v f_{X, Y}(v, z/v)dv =\int_v f_{X}(v) f_{Y}(z/v)  \left| \frac{1}{v} \right|dv $$

(c) Let $Z = X/Y$, $V = X$, then 
$$  \left| \frac{\partial (X, Y)}{\partial(V, Z)}\right|= \begin{vmatrix}1 & 0 \\ 1/z & -v/z^2 \end{vmatrix} = \left| \frac{v}{z^2}\right| $$
Therefore 
$$ f_Z(z) = \int_v f_{V, Z}(v,z) dv = \int_v f_{X, Y}(v,v/z)\left| \frac{\partial (X, Y)}{\partial(V, Z)}\right|dv = \int_v f_{X}(v) f_{Y}(v/z) \left| \frac{v}{z^2}\right|dv$$

\section*{Exercise 5.7}
(a) Combining the terms on the right side and order the term by power of $w$, we get
$$ \begin{aligned}
	 \left( \frac{A}{\tau^2}- \frac{C}{\sigma^2} \right) w^3 &= 0 \\
	 \left( -\frac{2Az}{\tau^2} + \frac{B}{\tau^2} - \frac{D}{\sigma^2} \right)w^2 &= 0\\
	 \left( A + \frac{Az^2}{\tau^2} - \frac{2Bz}{\tau^2} - C \right)w &= 0 \\
	 B + \frac{Bz^2}{\tau^2} - D &= 1 
\end{aligned}
$$
We get linear equation of 
$$ \begin{pmatrix}
	\sigma^2 & 0 & -\tau^2 & 0 \\
	-2z\sigma^2 & \sigma^2 & 0 & -\tau^2 \\
	\tau^2 + z^2 & -2z & -\tau^2 & 0 \\
	0 & \tau^2 + z^2 & 0 & -\tau^2 \\
\end{pmatrix}
\begin{pmatrix}
	A \\ B \\ C \\ D 
\end{pmatrix}
= 
\begin{pmatrix}
	0 \\ 0 \\ 0 \\ \tau^2
\end{pmatrix}
$$
The determinant is $(-\sigma^2 + \tau^2 + z^2)^2 + 4z^2\sigma^2 \neq 0$. So $A,B,C,D$ exists.

(b) Skipping the trivial calculation.

\section*{Exercise 5.8}
(a) $$
\begin{aligned}
	(n-1)S^2 &= \sum_i (X_i - \bar{X})^2 \\
	    &= \sum_i \left(X_i - \frac{1}{n} \sum_j X_j\right)^2 \\
	    &= \sum_i \left(X_i - \frac{2}{n} X_i \sum_j X_j + \frac{1}{n^2} \left(\sum_j X_j\right)^2 \right) \\
	    &= \sum_i X_i - \frac{2}{n} \sum_i X_i \sum_j X_j + \frac{1}{n^2} \sum_i \left(\sum_j X_j\right)^2 \\
	    &= \sum_i X_i - \frac{2}{n} \sum_i \sum_j X_i X_j + \frac{1}{n} \sum_i \sum_j X_i X_j \\
\end{aligned}
$$
Multiply both side by $2n$, we get 
$$
\begin{aligned}
	2n(n-1)S^2 &= 2n \sum_i X_i - 2 \sum_i \sum_j X_i X_j \\
	&= n \sum_i X_i - 2 \sum_i \sum_j X_i X_j + n \sum_i X_i \\
	&= n \sum_i X_i - 2 \sum_i \sum_j X_i X_j + n \sum_j X_j \\
	&= \sum_j \sum_i X_i - 2 \sum_i \sum_j X_i X_j + \sum_i \sum_j X_j , \mbox{ (Note that } n=\sum_i 1=\sum_j 1 \mbox{ )} \\
	&=  \sum_i \sum_j (X_i - X_j)^2 \\
\end{aligned}
$$ 
\QED 

(b) Let $Y_i = X_i - \theta_1$. Then $\E Y_i = 0$, $\E Y_i^j = \theta_j$. $$\begin{aligned}
	 2N(N-1)S^2 &= \sum_i \sum_j (X_i - X_j)^2 \\
	 &= \sum_i \sum_j (Y_i - Y_j)^2 \\
	 &=  \sum_{i\neq j} (Y_i)^2 - Y_i Y_j + (Y_j)^2 \\
	 2N(N-1) \E S^2 &=\sum_{i\neq j} E(Y_i)^2 - 2\E Y_i \E Y_j + \E (Y_j)^2 \\
	   		 &= [E(Y_1)^2 - 2\E Y_1 \E Y_2 + E(Y_2)^2] \\
	   		 &= 2N(N-1) \theta_2 \\
	   \E S^2 &= \theta_2
\end{aligned}$$
For $i\neq j$, $i$ has $N$ choices and $j$ has $N-1$. Therefore there are $N(N-1)$ terms in the sum.

$$ \begin{aligned}
	(4N^2(N-1)^2)S^4 &=  \sum_i \sum_j (Y_i - Y_j)^2 \sum_m \sum_n (Y_m - Y_n)^2 \\
	&= \sum_{i\neq j}\sum_{m\neq n} (Y_i^2 -2 Y_i Y_j + Y_j^2)(Y_m^2 - 2Y_m Y_n + Y_n^2) \\
	&= \sum_{i\neq j, m\neq n} Y_i^2 Y_m^2 -2Y_i^2 Y_m Y_n + Y_i^2 Y_n^2 \\
	& -2Y_i Y_j Y_m^2 + 4Y_iY_jY_mY_n - 2Y_iY_jY_n^2 \\
	& + Y_j^2Y_m^2 - 2Y_j^2Y_mY_n + Y_j^2 Y_n^2 \\
\end{aligned}$$
We have terms of 3 patterns $Y_i^2 Y_m Y_n$, $Y_i^2Y_m^2$ and $Y_iY_jY_mY_n$. The rest are equivalent. Grouping them together, we have.

$Y_i^2Y_m^2$ can be split into $Y_i^4$ when $i=m$ and $Y_i^2Y_m^2$ when $m\neq i$

$Y_iY_jY_mY_n$ will not vanish only when $i =m, j = n$ or $i = n, j = m$. It can be written as  $2Y_i^2Y_m^2$ when $i=m$ (Times 2 to account for both cases)


$Y_i^2 Y_m Y_n$ will vanish when we take the expected value since $m \neq n$ and $\E Y_i=0$ 
$$\begin{aligned}
	\E (4N^2(N-1)^2)S^4 &= \E \sum_{i\neq j, m\neq n} 4 Y_i^2 Y_m^2 - 8 Y_i^2 Y_m Y_n + 4 Y_iY_jY_mY_n \\
	&=\E\left[ \sum_{i\neq j, m\neq n} 4 Y_i^2 Y_m^2  + 4 Y_iY_jY_mY_n \right] \\
	&=\E \left[4 \sum_{i\neq j, m\neq n, i=m} Y_i^4 + 4 \sum_{i\neq j, m\neq n, i\neq m}   Y_i^2 Y_m^2 + 8 \sum_{i\neq j, m\neq n, i=m, j=n} Y_i^2 Y_m^2 \right]
\end{aligned}
$$
The first term $Y_i^4$, $i=m$ has $N$ choices. Then $j$ and $n$ both have $N-1$ choices. So it has $N(N-1)^2$ terms.

The second term $ Y_i^2 Y_m^2$, $i$ has $N$ choices. Then $m$ has $N-1$ choices. $j \neq i$ so $j$ has $N-1$ choices. $m\neq n$ so $n$ has $N-1$ choices. So it has $N(N-1)^3$ terms.

The third term $Y_i^2 Y_m^2$, Since $i=m$, there are $N$ choices. $j=n$ there are $N-1$ choices since $j\neq i$.  Therefore there are $N(N-1)$ terms.


$$
\begin{aligned}
		(4N^2(N-1)^2)\E S^4 &= 4N(N-1)^2  \E Y_1^4 + 4 N(N-1)^3 \E Y_1^2 \E Y_2^2 + 8 N(N-1) \E Y_1^2 \E Y_2^2  \\
		(4N^2(N-1)^2)\E S^4 &=4N(N-1)^2\theta_4 + 4N(N-1)[(N-1)^2 + 2] \theta_2^2 \\
		\E S^4 &=\frac{1}{N}\theta_4 + \frac{N^2 - 2N + 3}{N(N-1)} \theta_2^2 \\
\end{aligned}
$$

Therefore $$\Var S^4 = ES^4 -(ES^2) = \frac{1}{N}\theta_4 + \frac{N^2 - 2N + 3}{N(N-1)} \theta_2^2 - \theta_2^2 =  \frac{1}{N} \left(\theta_4 - \frac{N-3}{N-1} \theta_2^2 \right)
$$
\QED

(c)
 Let $Y_i = X_i - \theta_1$. Then $\E Y_i = 0$, $\E Y_i^j = \theta_j$
 $$\begin{aligned}
	\Cov(\bar{X}, S^2) &= \Cov(\bar{Y}, S^2) \\
	 &= \E(\bar{Y}, S^2) - \E \bar{Y} \E S^2 \\
	 &= \E(\bar{Y}, S^2) \\
	 &= \frac{1}{N} \frac{1}{2N(N-1)} \E \sum_{i\neq j, k} Y_k (Y_i - Y_j)^2 \\
	 &= \frac{1}{2N^2(N-1)} \E \sum_{i\neq j, k} Y_kY_i^2 - 2Y_k Y_iY_j + Y_k Y_j^2 \\
\end{aligned}$$
Note that $Y_k Y_iY_j$ vanishes because $i \neq j$ so the expected value of one of them will be 0. By the same argument,  $Y_kY_i^2$ will not be 0 if $k=i$. Therefore 
 $$\begin{aligned}
	\Cov(\bar{X}, S^2) &=\frac{1}{2N^2(N-1)} \E \sum_{i\neq j, k=i} 2 Y_i^3 \\
	&= \frac{1}{2N^2(N-1)} 2N(N-1) \E Y_i^3 \\
	&= \frac{1}{N } \theta_3 \\
\end{aligned}$$
$\Cov(\bar{X}, S^2) = 0 $ when $\theta_3 = 0$.
\QED

\section*{Exercise 5.9}
Using induction, when $n = 2$, 
$$ \begin{aligned}
	(a_1^2 + a_2^2)(b_1^2+b_2^2) - (a_1b_1+a_2b_2)^2 &= a_1^2b_2^2 + a_2^2b_1^2 - 2a_1a_2b_1b_2 \\ 
	&= a_1^2b_2^2 -  a_1a_2b_1b_2  + ( a_2^2b_1^2 - a_1a_2b_1b_2) \\
	&= a_1b_2(a_1b_2 - a_2b_1) + a_2b_1(a_2b_1 - a_1b_2) \\
	&= (a_1b_2 - a_2b_1)^2 
\end{aligned}$$
The identity holds. Now suppose $n=k$ holds and consider $n=k+1$. Let $t_{ij} = (a_ib_j - a_jb_i)^2$. Then Right hand side is becomes $\sum_{i=1}^{k}\sum_{j=i+1}^{k+1} t_{ij}$. To find the extra term compared to the sum in $n=k$, write the entry in a matrix.
$$
\begin{pmatrix}
	t_{1,1} & \cdots & t_{1, k} & t_{1, k+1}\\
	\vdots &    & \vdots & \vdots \\
	t_{k-1, 1} &  \ddots  & t_{k-1, k} & \vdots \\
	t_{k, 1} &  \cdots  & t_{k,k}  & t_{k, k+1}\\
	t_{k+1, 1} &  \cdots  & t_{k+1,k}  & t_{k+1, k+1}\\
\end{pmatrix}
$$
Since we sum over $j > i$ so we sum the matrix above the diagonal. The difference between $k$ and $k+1$ for the left hand side is the last column above the diagonal which is $\sum_{i=1}^{k} t_{i, k+1}$. 

Expanding the right hand side
$$\begin{aligned}
	RHS = \sum_{i=1}^{k}\sum_{j=i+1}^{k+1} t_{ij} &= \sum_{i=1}^{k-1}\sum_{j=i+1}^k t_{ij} + \sum_{i=1}^{k} t_{i, k+1} \\
	&= \sum_i^k \sum_j^k a_i^2  b_j^2 - \sum_i^k \sum_j^k a_ib_ia_jb_j + \sum_{i=1}^{k} (a_ib_{k+1} - a_{k+1}b_i)^2
\end{aligned}$$

Expanding the left hand side
$$\begin{aligned}
	LHS &= (\sum_i^k a_i^2 + a_{k+1}^2)(\sum_j^k b_j^2 + b_{k+1}^2) - \left(\sum_i^k a_ib_i + a_{k+1}b_{k+1}\right)^2 \\
	 &= \sum_i^k \sum_j^k a_i^2  b_j^2 + a_{k+1}^2 \sum_i^k b_i^2 + b_{k+1}^2 \sum_i^k a_i^2 - 2a_{k+1}b_{k+1} \sum_i^k a_ib_i - \sum_i^k \sum_j^k a_ib_ia_jb_j  \\
	 &= \sum_i^k \sum_j^k a_i^2  b_j^2 +  
	     \sum_i^k \left[(a_{k+1}^2 b_i^2 -a_{k+1}b_{k+1} a_ib_i)  + (b_{k+1}^2 a_i^2 - a_{k+1}b_{k+1} a_ib_i) \right] 
	     - \sum_i^k \sum_j^k a_ib_ia_jb_j \\
	 &= \sum_i^k \sum_j^k a_i^2 b_j^2+
	     \sum_i^k \left[a_{k+1}b_i(a_{k+1}b_i -b_{k+1} a_i) + b_{k+1}a_i(b_{k+1}a_i-a_{k+1}b_i)\right]
	     -\sum_i^k \sum_j^k a_ib_ia_jb_j \\
	 &= \sum_i^k \sum_j^k a_i^2  b_j^2 - \sum_i^k \sum_j^k a_ib_ia_jb_j + \sum_{i=1}^{k} (a_ib_{k+1} - a_{k+1}b_i)^2 \\
	 &= RHS
\end{aligned}
$$
Therefore the identity holds. 

We don't actually need this identity to prove the proposition.
The correlation coefficient is defined as $\rho_{xy}= \frac{\Cov(X, Y)}{\sigma_x \sigma_y}$. If data points $(x_i, y_i)$ lies on a line, then $Y = aX + b$. We have 
$$ \begin{aligned}
	\rho_{xy} &= \frac{\Cov(X, Y)}{\sigma_x \sigma_y}  \\
	 		  &= \frac{\Cov{X, aX + b}}{\sqrt{\Var X \Var (aX+b)}} \\
	 		  &= \frac{a \Var X }{\sqrt{a^2 (\Var X)^2}} \\
	 		  &= 1
\end{aligned}
$$

Now suppose $\rho_{xy}=1$. Since each data point has equal weight, so $p=\frac{1}{n}$. Then we have 
$$ \begin{aligned}
	\Cov(X, Y) &= \sigma_x \sigma_y \\
	\left( \sum_{i}(x_i -\bar{x})(y_i -\bar{y}) \right)^2 &= 
	\left( \sum_{i}(x_i -\bar{x})^2 \right) \left( \sum_{i}(y_i -\bar{y})^2 \right)
\end{aligned}$$

Note the left hand side is less or equal to the right side by Cauchy-Schwarz's Inequality. It is equal only when $(x_i -\bar{x})(y_j-\bar{y}) = (x_j -\bar{x})(y_i -\bar{y})$ for any $i,j$.
Therefore 
$$  \frac{y_i -\bar{y}}{x_i -\bar{x}} = \mbox{constant} $$
Which is the definition of linearity.
\QED


\end{document}