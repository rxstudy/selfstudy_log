\documentclass[12pt]{article}

\include{pythonlisting}
\usepackage{fullpage}
\usepackage{times}
\usepackage[normalem]{ulem}
\usepackage{multirow}
\usepackage{fancyhdr,graphicx,amsmath,amssymb, mathtools, scrextend, titlesec, enumitem}\usepackage[pdftex]{hyperref}
\usepackage[ruled,vlined]{algorithm2e}
\usepackage{parskip}
\usepackage{listings}
\usepackage{amsmath}
\usepackage{physics}
\usepackage{bbm}
\usepackage{titling}
\usepackage{bm}
\usepackage{geometry}
\geometry{top=1cm, left=1cm, bottom=1.5cm, right=1.5cm, margin=1cm}
\newcommand{\varvec}[2][n]{#2_1,\ldots, #2_#1}
\newcommand{\Var}{\mathrm{Var}}
\newcommand{\Bias}{\mathrm{Bias}}
\newcommand{\Cov}{\mathrm{Cov}}
\newcommand{\E}{{\rm I\kern-.3em E}}
\newcommand{\Binomial}{\mathrm{Binomial}}
\newcommand{\Bernoulli}{\mathrm{Bernoulli}}
\newcommand{\Poisson}{\mathrm{Poisson}}
\newcommand{\Normal}{\mathcal{N}}
\newcommand{\one}{\mathbbm{1}}
\newcommand{\Beta}{\text{Beta}}
\newcommand{\BetaPDF}{\text{Beta}}
\newcommand{\GammaPDF}{\text{Gamma}}
\newcommand{\Uniform}{\mathrm{Uniform}}
\newcommand{\QED}{\newline \mbox{} \hfill $\blacksquare$}
\newcommand{\Real}{\mathbb{R}}
\newcommand{\Sgn}{\mbox{Sgn}}
\renewcommand*{\arraystretch}{1.4}
\newtheorem{theorem}{Result}
\newtheorem{definition}{Definition}
\lstset{frame=tb,
	language=Python,
	aboveskip=3mm,
	belowskip=3mm,
	showstringspaces=false,
	columns=flexible,
	basicstyle={\small\ttfamily},
	numbers=none,
	stringstyle=\color{mauve},
	breaklines=true,
	breakatwhitespace=true,
	tabsize=3
}

\title{Chapter 4: Multiple Random Variables}
\begin{document}
\maketitle

\section*{Exercise 4.1}
(a) Since $f_{X,Y}(x,y)$ is constant. $X^2+Y^2<1$ is circle of radius 1. Therefore $P(X^2 + Y^2 < 1) = \pi/4$

(b) $2X-Y = 0$ divides the unit square into two region of equal area and $f$ is constant. Therefore $P(2X-Y > 0) = 1/2$

(c) $P(|X+Y|<2) = P (-2 < X + Y < 2)$. The area covers the entire square. Therefore $P(|X+Y|<2) = 1$.

\section*{Exercise 4.4}
(a) Since $\int_0^1 \int_0^2 f(x,y)dxdy  = \int_0^1 \int_0^2 C(x+2y) dx dy = 4C = 1$
So $C = 1/4$.

(b) $f(x) = \int^1_0 f(x,y)dy = (1/4) (xy + y^2)|^1_0 = \frac{x+1}{4}, \ \ x \in (0,2)$

(c) 
For $(x,y) \in (0,2)\times (0,1)$:
$$F(x, y) = P(X <x, Y <y) = \int_{-\infty}^{x} \int_{\infty}^{y} \frac{t+2s}{4} dsdt = \int_{0}^{x} \int_{0}^{y} \frac{t+2s}{4} dsdt =  \frac{1}{8}(x^2y + 2xy^2) $$
For $(x,y) \in (0,2) \times [1, \infty)$:
$$F(x, y) = P(X <x, Y <y) = \int_{0}^{x} \int_{0}^{1} \frac{t+2s}{4} dsdt = \frac{1}{8}(x^2 + 2x) $$
For $(x,y) \in (-\infty,2] \times (0,1)$:
$$F(x, y) = P(X <x, Y <y) = \int_{0}^{2} \int_{0}^{y} \frac{t+2s}{4} dsdt = \frac{1}{2}(y + y^2) $$



(d) from (b), we have $f(x) = \frac{x+1}{4}$. And $z = \frac{9}{(x+1)^2}$ is monotonic for $x\in[0,2]$ with $z\in[1, 9]$. So we can take $x = \frac{\sqrt{z}}{3} - 1$. Then $$f(z) = f(x^{-1}(z))\left|\frac{dx}{dz} \right| =  \frac{3}{4}(z^{-1/2}) (\frac{3}{2}z^{-3/2}) = \frac{9}{8}z^{-2}$$

\section*{Exercise 4.5}
(a) The area for integration is $0 < x < 1$ and $0 < y < x^2$. 
$$ P(X > \sqrt{Y}) = \int_0^1 \int_{0}^{x^2} x+ y dy dx =\int_0^1 x^3 + \frac{x^4}{2} dx = 0.35 $$

(b) The area of integration is $0 < x < 1$ and $x^2 < y < x$.
$$ P(X^2 <Y < X) = \int_0^1 \int_{x^2}^{x} 2x dy dx = \int_0^1 2x^2 - 2x^3 dx = \frac{1}{6}$$

\section*{Exercise 4.6}
Let $X, Y$ be the time A and B arrive in time interval $[0,1]$. Since they are independent, $f(x,y) = f(x) f(y) = 1$ for $(x,y) \in [0,1]\times [0,1]$.

Let $T$ be the length of time A waits for B. Then $T = \max (Y - X, 0)$ because $T=0$ when $Y < X$. 
$$ P(T < t) = P(\max (Y - X, 0)) = P(Y - X < t, Y \geq X) + P(Y < X)$$ 
For term $P(Y - X < t, Y \geq X)$, The area of integration is the area between $y - x = t$ and $ y \geq x$ bounded by unit square. We can find the complement area which is an isosceles right triangle with side of $1-t$, which gives
$$ P(Y - X < t, Y \geq X) =   \frac{1}{2} - \frac{1}{2} (1-t)^2$$
$P(Y<X)$ is the lower half triangle of the unit square which has area of $\frac{1}{2}$
Therefore 
$$P(T < t) = P(Y - X < t, Y \geq X) + P(Y < X) = 1 - \frac{1}{2} (1-t)^2$$


\section*{Exercise 4.7}
We can formulate the problem as such: $X \in [0,30], Y \in [40,50]$, find $P(X+Y < 60)$. We want to find the intersection of $x+y=60$ with $[0,30] \times [40, 50]$. We get $(10,50), (20,40)$. Since the distributions are uniform, we can simply find area of the trapezoid and divide it by the total area. 
$$P(X+Y < 60) = \frac{10(10 + 20)0.5}{10(30)} = 150/300 = 0.5$$

\section*{Exercise 4.9}
For interval $[a,b] \times [c,d]$.
$$\begin{aligned} 
P(a \leq X \leq b) P(c \leq Y \leq d) &= [F_X(b)-F_X(a)] [F_Y(d)-F_Y(c)] \\ 
&= F_X(b)F_Y(d)-F_X(b)F_Y(c)-F_X(a)F_Y(d) + F_X(a)F_Y(c)  \\
&= F(b,d)-F(b,c)-F(a,d)+F(a,c)
\end{aligned}$$

If we define the regions $A_1 = [a,b]\times[c,d]$, $A_2=[a,b]\times(-\infty, c)$ , $A_3=(-\infty, a)\times(-\infty, c)$, $A_4 = (-\infty, a)\times [c,d]$. Then
   $$\begin{aligned}
   		F(b,d) &= P(A_1)+P(A_2)+P(A_3)+P(A_4) \\
   		F(b,c) &= P(A_3) + P(A_2) \\
   		F(a,d) &= P(A_3) + P(A_4) \\
   		F(a,c) &= P(A_3)
   \end{aligned}$$
Hence 
$$\begin{aligned}
P(a \leq X \leq b) P(c \leq Y \leq d) &= F(b,d)-F(b,c)-F(a,d)+F(a,c) \\
&= P(A_1)  \\
&= P([a,b]\times[c,d])  \\
&= P(X\in [a,b], Y\in [c,d])
\end{aligned}$$


\section*{Exercise 4.10}
(a) Summing up the columns and rows, we have $P(X=1) = 1/4$, $P(X=2)=1/2$ , $P(X=3)=1/4$. 
$P(Y=2)= P(Y=3) = P(Y=4) = 1/3$. 

(b) We can build up a table for independent $U,V$ just be multiplying the marginal probability.
\begin{center}
\begin{tabular}{ |c|ccc| } 
\hline
U/V & 1& 2&  3 \\
2 & 1/12 & 1/6 & 1/12 \\ 
3 & 1/12 & 1/6 & 1/12 \\ 
4 & 1/12 & 1/6 & 1/12 \\ 
\hline
\end{tabular}
\end{center}

\section*{Exercise 4.11}
$U$ and $V$ are dependent. Consider $P(V|U=n)$ and $P(V)$, knowing $U=n$ means there is only one toss of head in the first $n$ trials which means $P(V \leq n | U = n) = 0 \neq P(V \leq n)$.

\section*{Exercise 4.12}
Let $X$ and $Y$ be uniform(0,1). Then $f_{X,Y} = 1$ is a unit square on $[0,1]\times[0,1]$. By symmetry, we only need to consider the probability conditioned on $X > Y$ which is the lower half triangle of the unit square. Then the 3 segments are $1 - X, X - Y$ and $Y$. For the segments to be a triangle, denoted by event $T$, it must satisfy the sum of two sides is larger than the other side. Therefore
$$\begin{aligned}
 1 - X + X - Y &> Y \\
 1 - X + Y &> X - Y  \\
 X - Y + Y &> 1 - X \\
\end{aligned}$$
Simplifying the expression, we have event $T$ given $X > Y$ as the area of a region bounded by 
$$\begin{aligned}
 \frac{1}{2} &> Y \\
 Y &> X - \frac{1}{2}  \\
 X &> \frac{1}{2} \\
 X &> Y \\
\end{aligned}$$
Note that area of $X > Y$ is $\frac{1}{2}$ and the area of the region for $T$ is $\frac{1}{8}$.  So $P(T | X > Y)  = \frac{1/8}{1/2} = \frac{1}{4}$ and .

Finally, we have $P(T) = P(T | X > Y) P(X > Y) + P(T | X \leq Y) P( X \leq Y) = \frac{1}{4}$

\end{document}

