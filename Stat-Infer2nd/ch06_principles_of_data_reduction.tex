\documentclass[12pt]{article}

\include{pythonlisting}
\usepackage{fullpage}
\usepackage{times}
\usepackage[normalem]{ulem}
\usepackage{multirow}
\usepackage{fancyhdr,graphicx,amsmath,amssymb, mathtools, scrextend, titlesec, enumitem}\usepackage[pdftex]{hyperref}
\usepackage[ruled,vlined]{algorithm2e}
\usepackage{parskip}
\usepackage{listings}
\usepackage{amsmath}
\usepackage{physics}
\usepackage{bbm}
\usepackage{titling}
\usepackage{bm}
\usepackage{geometry}
\geometry{top=1cm, left=1cm, bottom=1.5cm, right=1.5cm, margin=1cm}
\newcommand{\varvec}[2][n]{#2_1,\ldots, #2_#1}
\newcommand{\Var}{\mathrm{Var}}
\newcommand{\Bias}{\mathrm{Bias}}
\newcommand{\Cov}{\mathrm{Cov}}
\newcommand{\E}{{\rm I\kern-.3em E}}
\newcommand{\Binomial}{\mathrm{Binomial}}
\newcommand{\Bernoulli}{\mathrm{Bernoulli}}
\newcommand{\Poisson}{\mathrm{Poisson}}
\newcommand{\Normal}{\mathcal{N}}
\newcommand{\one}{\mathbbm{1}}
\newcommand{\Beta}{\text{Beta}}
\newcommand{\BetaPDF}{\text{Beta}}
\newcommand{\GammaPDF}{\text{Gamma}}
\newcommand{\Uniform}{\mathrm{Uniform}}
\newcommand{\QED}{\newline \mbox{} \hfill $\blacksquare$}
\newcommand{\Real}{\mathbb{R}}
\newcommand{\Sgn}{\mbox{Sgn}}
\renewcommand*{\arraystretch}{1.4}
\newtheorem{theorem}{Result}
\newtheorem{definition}{Definition}
\lstset{frame=tb,
	language=Python,
	aboveskip=3mm,
	belowskip=3mm,
	showstringspaces=false,
	columns=flexible,
	basicstyle={\small\ttfamily},
	numbers=none,
	stringstyle=\color{mauve},
	breaklines=true,
	breakatwhitespace=true,
	tabsize=3
}

\title{Chapter 6: Principles of Data Reduction}
\begin{document}
\maketitle

\section*{Exercise 6.1}
Yes
$$
 \frac{p(x|0,\sigma)}{q(|X| | 0, \sigma)} = \frac{p(x|0,\sigma)}{p(x| 0, \sigma) + p(-x|0, \sigma)} = \frac{1}{2}
$$
Does not depend on the paramters.

\section*{Exercise 6.2}
The pdf for $X_i$ is $f_{X_i}(x|\theta) = \exp(i\theta - x)\one_{x\geq i\theta}$. Then 
$$\begin{aligned}
   f_X(x_i|\theta) &= \prod_{i=1}^n f_{X_i}(x_i|\theta) \\
    	&=  \exp\left(\sum_{i=1}^n i\theta - x_i\right) \prod_{i=1}^n \one_{x_i \geq i\theta} \\ 
    	&=  \exp\left(\frac{n(n+1)\theta}{2}\right) \exp\left(\sum_i x_i\right)  \prod_{i=1}^n \one_{\frac{x_i}{i} \geq \theta} \\
    	&=  \exp\left(\frac{n(n+1)\theta}{2}\right)   \one_{\min \frac{x_i}{i} \geq \theta}  \exp\left(\sum_i x_i\right)
\end{aligned}
$$
$g(T(x)|\theta) = g(\min \frac{x_i}{i}|\theta) =  \exp\left(\frac{n(n+1)\theta}{2}\right)   \one_{\min \frac{x_i}{i} \geq \theta}$ and $h(x) = \exp\left(\sum_i x_i\right)$. By factorization theorem, it is sufficient statistic.

\section*{Exercise 6.3}
Given the pdf 
$$ f(x | \mu, \sigma) = \frac{1}{\sigma} e^{-(x-\mu)/\sigma}, \ \ u<x <\infty, 0 < \sigma < \infty$$
We need to get rid of the dependency on $u$ in the range of $x$. With indicator, we can rewrite it as
$$ f(x | \mu, \sigma) = \frac{1}{\sigma} e^{-(x-\mu)/\sigma}\one_{\mu < x}, \ \ -\infty <x <\infty, 0 < \sigma < \infty$$
Then $$
\begin{aligned}
	f(\vb{x} | \mu, \sigma) &= \prod^n_{i=0} 	f(x_i | \mu, \sigma) \\
		&= \frac{1}{\sigma^n} \exp\left(-\frac{\sum_i{x_i} - n\mu}{\sigma}\right)   \prod^n_{i=0} \one_{\mu < x_i} \\
		&= \frac{1}{\sigma^n} \exp\left(-\frac{\sum_i{x_i} - n\mu}{\sigma}\right)   \one_{\mu <  x_{\min}}
\end{aligned}
$$
Define $T(\boldmath(x)) = (t_1, t_2) = (\sum_i x_i, x_{\min})$ and $h(\vb{x}) = 1$. By factorization theorem, it is a sufficient statistics.

\section*{Exercise 6.4}
The pdf is $$
\begin{aligned}
	f(x|\theta) &= \left[\prod_i^n h(x_i) \right] c^n(\theta) \exp \left( \sum_{i=1}^{n} \sum_{j=1}^k w_j(\theta)t_j(x_i) \right) \\
	&= H(x)C(\theta) \exp \left(  \sum_{j=1}^k w_j(\theta)\sum_{i=1}^{n} t_j(x_i) \right) \\
	&= H(x)C(\theta) \exp \left(  \sum_{j=1}^k w_j(\theta)T_j(x) \right) \\
	&= H(x)C(\theta)g(T(x)|\theta)
\end{aligned}
$$
Therefore by Factorization theorem,  $T(X) = [T_j(X)] =  \left( \sum_{i=1}^{n} t_j(x_i) \right) $ is sufficient statistics.

\section*{Exercise 6.8}
The pdf of sample $X$ is $f(x|\theta) = \prod_{i=1}^n f(x_i - \theta)$. By theorem 6.2.13, we take the ratio of $f(x|\theta)$ and $f(y|\theta)$,  $$ \frac{f(x|\theta)}{f(y|\theta)} = \frac{\prod_{i=1}^n f(x_i - \theta)}{\prod_{i=1}^n f(y_i - \theta)}$$
Note that the above expression is a constant of $\theta$ only when the terms cancel out which implies there exists an ordering $T$ such that $f(T(x_i) - \theta) = f(T(y_i) - \theta)$. Order statistics is such an ordering. Therefore it is the minimal statistics.

\section*{Exercise 6.9}
(a) the ratio is $$\frac{f(x|\theta)}{f(y|\theta)} = \exp\left( -\frac{1}{2} \sum_i (x_i - \theta)^2 + \frac{1}{2} \sum_i (y_i-\theta)^2 \right) =\exp\left( -\frac{1}{2} \sum_i (x_i^2 - y_i^2) + n \theta (\bar{x} - \bar{y}) \right)$$ 
For the ration to be constant function of $\theta$ iff $\bar{x} = \bar{y}$. So sample mean is the min sufficient statistics.

(b) the joint pdf is $f(x|\theta) = \exp( -\sum_i (x_i - \theta))$ Where $x_i > \theta, \theta \in \real$. We can rewrite this with indicator function, $$f(x|\theta) = I_{\min{x} > \theta } \exp( -\sum_i (x_i - \theta)), \mbox{  where  } x_i,\theta \in \real$$

So the ratio becomes 
$$
\frac{f(x|\theta)}{f(y|\theta)} = \frac{I_{\min{x} > \theta } \exp( -\sum_i (x_i - \theta))}{I_{\min{y} > \theta } \exp( -\sum_i (y_i - \theta))} =  \frac{I_{\min{x} > \theta } \exp( -\sum_i x_i) }{I_{\min{y} > \theta } \exp( -\sum_i y_i )}
$$

$\frac{I_{\min{x} > \theta }}{I_{\min{y} > \theta }}$ cancels out iff $\min x = \min y$ which is the minimal sufficient statistics.

(c) The ratio is the following 
$$
 \frac{f(x|\theta)}{f(y|\theta)} = \frac{\exp(\bar{y} - \bar{x})}{\prod_i (1 + \exp(x_i - \theta))^2/(1+ \exp(y_i - \theta))^2 }
$$
The only way for the ratio to be constant of $\theta$ is to have the bottom term cancels out which means for each $i$, there is $j$ such that $1 + \exp(x_i - \theta) = 1 + \exp(y_j - \theta)$. This can only happen when $T$ is order statistics of $x$ and $T(x) = T(y)$. Therefore order statistics is the minimal sufficient statistics.

(d) Same reasoning as (c),  order statistics is the minimal sufficient statistics.

(e) The ratio is $$
\frac{f(x|\theta)}{y|\theta} = \exp{- \sum_i |x_i - \theta| + \sum_i |y_i - \theta|}
$$
The expression in the exponent is the difference between the distance from $\theta$ to $x$ and that of $y$. Since $\theta$ can be any value, so the difference between $x$ and $y$ to $\theta$ cannot stay constant unless the data point in $x$ is the same as data point in $y$ which means ordered statistic is the minimal sufficient statistics.

\section*{Exercise 6.12}
(a.1)The distribution of $N$ is just $P(N=k) = p_k$ independent of $\theta$, therefore $N$ is ancillary statistics.

(a.2)Suppose $(X(x), N(x)) = (X(y), N(y))$, then $x$ and $y$ have the same successes and same length, so the ratio $f(x|\theta) / f(y|\theta)$ is constant 1.

Now Suppose $x=(x_1,\ldots, x_{n_1})$ and $y=(y_1,\ldots,y_{n_2})$ such that $f(k_1|
\theta) / f(k_2|\theta)$ is constant  (Here $k_1 = X(x)$ and $k_2 = X(y)$ are successes).  Then $$
  \frac{f(k_1, n_1|\theta)}{f(k_2, n_2|\theta)} = \frac{f(k_1|\theta, n_1)p(n_1)}{f(k_2 |\theta, n_2)p(n_2)} = \frac{{n_1 \choose k_1}}{{n_2 \choose k_2}} \theta^{k_1 - k_2} (1-\theta)^{n_1 -n_2 - (k_1-k_2)} \frac{p_{n_1}}{p_{n_2}} $$
For the expression to be constant, we get $k_1=k_2$ and $n_1 -n_2 - (k_1 -k_2) = 0 \Rightarrow n_1 = n_2$ So there $(X,N)$ is minimal statistics.

(b) Since the bias of the estimator $X/N$ is $\E [X/N - \theta]$. 
$$ 
\begin{aligned}
  \mbox{Bias}(X/N, \theta) &= \E[X/N - \theta] \\
        &= \E[X/N] - \theta \\
        &= \sum_{X, N} \frac{X}{N}P(X, N|\theta)  - \theta \\ 
        &= \sum_{n=1}^\infty \sum_{k=1}^n \frac{k}{n}P(X=k|N=n, \theta)P(N=n) - \theta \\
        &= \sum_{n=1}^\infty \sum_{k=1}^n \frac{k}{n} {n \choose k} \theta^k (1- \theta)^{n-k} p_n - \theta \\
        &= \sum_{n=1}^\infty  \frac{p_n}{n} \sum_{k=1}^n k {n \choose k} \theta^k (1- \theta)^{n-k} - \theta \\
        &= \sum_{n=1}^\infty  \frac{p_n}{n} \E_{\Binomial(n, \theta)}[X]  - \theta \\
        &=  \sum_{n=1}^\infty  \frac{p_n}{n} n\theta - \theta \\
        &= \theta - \theta = 0 
\end{aligned}
$$

\subsection*{Exercise 6.13}
Since $f(x, y|\alpha) = \alpha (xy)^{\alpha - 1}e^{-x^{\alpha} - y^{\alpha}}$

Let $$ \begin{cases}
	s = \log x / \log y \\
	t = \log x + \log y
\end{cases}
\Rightarrow
 \begin{cases}
	x = \exp(\frac{st}{s+1}) \\
	y = \exp(\frac{t}{s+1})
\end{cases}
$$
$$
\Rightarrow
\frac{\partial(x,y)}{\partial(s,t)} =
\begin{vmatrix}
	\frac{t}{(s+1)^2}e^{st/(s+1)}  &  \frac{s}{s+1}e^{st/(s+1)}  \\
	-\frac{t}{(s+1)^2}e^{t/(s+1)} & \frac{1}{s+1}e^{t/(s+1)} \\
\end{vmatrix}
= \frac{t}{(s+1)^2}e^t
$$
Then by change of variables, $$
f(s, t|\alpha) = f(x, y|\alpha)\left|\frac{\partial(x,y)}{\partial(s,t)}\right| = \alpha \frac{\alpha t}{(s+1)^2} e^{\alpha t} e^{-\exp(\frac{\alpha t s}{s+1}) - \exp(\frac{\alpha t} {s+1})} 
$$

We want to find $f(s|\alpha) = \int_t f(s, t|\alpha) dt$. If we let $u = \alpha t$, then $\alpha dt = du$. Therefore
$$
\begin{aligned}
	f(s|\alpha) &= \int_t f(s, t|\alpha) dt \\
	&= \int_t  \frac{\alpha t}{(s+1)^2} e^{\alpha t} \exp{-\exp(\frac{\alpha t s}{s+1}) - \exp(\frac{\alpha t} {s+1})} \alpha dt \\
	&= \int_u  \frac{u}{(s+1)^2} e^{u} \exp{-\exp(\frac{u s}{s+1}) - \exp(\frac{u} {s+1})} du \\
	&= f(s)
\end{aligned}
$$
$\alpha$ has vanished from the final expression, therefore the distribution of $s=\log x / \log y$ which does not depend on $\alpha$ is ancillary.

\end{document}